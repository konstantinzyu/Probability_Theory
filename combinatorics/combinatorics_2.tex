	\subsection{Элементарная комбинаторика}
	
	\subsubsection{Элементарные факты для подсчёта числа комбинаций}
	
	Центральной задачей комбинаторной математики можно считать задачу расположения объектов в соответствии со специальными правилами и нахождения числа способов, которыми это может быть сделано.
	Для нахождения числа способов полезны следующие результаты.
	
	\begin{lemma}%[о правиле суммы]
		Мощность объединения двух конечных множеств $A$ и $B$ равна
		$$|A\cup B| = |A| + |B| - |A\cap B|.$$
	\end{lemma}
	\textbf{Доказательство} проводится изображением диаграммы Эйлера-Венна и подсчётом элементов в её частях. $\Box$
	
	
	\begin{lemma}[формула включений--исключений]
		$$\left|\bigcup_{i=1}^{n}A_i \right| = \sum_{i} | A_i | - \sum_{i<j} | A_i \cap A_j | + \sum_{i<j<k} | A_i \cap A_j \cap A_k | - \ldots$$
		$$\ldots + (-1)^{n-1} | A_1 \cap A_2 \cap \ldots \cap A_n |.$$
	\end{lemma}
	\noindent\textbf{Доказательство} проводится по индукции с помощью предыдущей леммы, которая используется как база и шаг индукции одновременно. $\Box$
	
	\begin{lemma}[принцип умножения]
		Число пар, которые можно составить, когда первый элемент выбирается из множества $A$, а второй независимо от первого берется из множества $B$ равно $|A|\cdot|B|$.
	\end{lemma}
	\noindent\textbf{Доказательство.} Следует из формулы $|A\times B| = |A|\cdot|B|$. $\Box$
	
	
	\begin{lemma}%[о числе кортежей]
		Число кортежей $(a_1, \ldots, a_n)$ в которых независимо выбираются первый элемент $a_1$ из множества $A_1$, второй $a_2$ --- из множества $A_2$, \ldots, $a_n$ --- из множества, равно  $|A_1|\cdot|A_2|\cdot\ldots\cdot|A_n|$.
	\end{lemma}
	\noindent\textbf{Доказательство.} По индукции из предыдущей леммы. $\Box$
	
	\subsubsection{Классические комбинаторные величины}
	
	\defin{Перестановка}{permutation} --- это расположение $n$ различных объектов по различным местам.
	Число перестановок множества в $n$ элементов обозначаем $P_n$. Например, число способов разложить девять подписанных писем в девять подписанных конвертов --- это число перестановок $P_9$.
	
	
	\defin{Перестановка с повторениями}{permutation_repetitions} --- это расположение $n$ объектов, некоторые из которых могут быть тождественны друг другу, по различным местам.
	Число перестановок множества в $n$ элементов, где $n_1$ элементов 1-го типа, \ldots, $n_k$ элементов $k$-того типа (причём $n_1+\ldots+n_k=n$) обозначаем $P(n_1, \ldots, n_k)$. Например, число способов составить из букв слова <<математика>> всевозможные (абстрактные) слова --- это число перестановок с повторениями $P(2,3,2,1,1,1)$.
	
	Упорядоченная выборка (кортеж) в $m$ элементов из множества в $n$ элементов называется \defin{размещением}{partial_permutation}, неупорядоченная выборка (подмножество) из $n$ элементов по $m$ называется \defin{сочетанием}{combination}.
	Выборка может быть \underline{без повторений}, если элементы повторяться не могут, и может быть \underline{с повторениями}, если элементы в выборке повторяются. 
	Число размещений без повторений обозначается $A_n^m$, с повторениями --- $\widetilde{A}_n^m$.
	Число сочетаний без повторений обозначается $C_n^m$ или $\binom{n}{m}$, с повторениями --- $\widetilde{C}_n^m$ или $\left(\binom{n}{m}\right)$.
	
	Примеры:
	\begin{enumerate}
		\item Число способов расставить шесть книг из имеющихся восьми на книжкой полке --- это число размещений без повторения $A_{8}^{6}$.
		\item Число способов выбрать команду в пять человек из девяти данных игроков --- это число сочетаний без повторения $C_9^5$.
		\item Число способов купить семь ручек из продающихся ручек пяти фирм --- это число сочетаний с повторениями $\widetilde{C}_5^7$.
	\end{enumerate}
	
	
	\begin{theorem}%[о комбинаторных числах]
		Числа перестановок, размещений и сочетаний находятся для неотрицательных целых\ $n, n_i, m$\ по следующим формулам (для комбинаций без повторений дополнительно требуется $m\leqslant n$):
		
		%\vspace{2mm}
		\begin{tabular}[c]{|p{8em}|p{12em}|p{12em}|}
			\hline
			Комбинация & без повторений & с повторениями  \\ \hline
			перестановка    & $P_n=n!$ & $P(n_1, \ldots, n_k)=\frac{\left(n_1+\ldots+n_k\right)!}{n_1!\cdot\ldots\cdot n_k!}$ \\ \hline
			размещение    & $A_n^m=n^{[m]}={\frac{n!}{(n-m)!}}^{\phantom{M}}$ & $\widetilde{A}_n^m=n^m$ \\ \hline
			сочетание  & $C_n^m=\frac{n!}{m!(n-m)!}$      & $\widetilde{C}_n^m=C_{n+m-1}^m={\frac{(n+m-1)!}{m!(n-1)!}}^{\phantom{M}}$ \\ \hline
		\end{tabular}
	\end{theorem}
	\noindent\textbf{Доказательство.}
	
	Для $A_n^m$ применяем принцип умножения: $n\cdot(n-1)\cdot\ldots\cdot(n-m+1)=\frac{n!}{(n-m)!}.$ 
	
	Для $\widetilde{A}_n^m$ также работает принцип умножения: $n\cdot n\cdot\ldots\cdot n = n^m.$ 

	Для $P_n$ имеем $P_n=A^n_n=n!$.

	Для $C_n^m$ замечаем, что выбрать $m$ и упорядочить их можно $A_n^m$ способами, причём каждый набор упорядочен $m!$ раз, поэтому
	$C_n^m=\frac{A_n^m}{m!}=\frac{n!}{m!\,(n-m)!}.$
	
	Для $P(n_1, \ldots, n_k)$ аналогично предыдущему замечаем, что верно равенство $P_{n_1+\ldots+n_k} = P(n_1, \ldots, n_k) \cdot n_1! \cdot \ldots \cdot n_k!$, откуда $P(n_1, \ldots, n_k) = \frac{P_{n_1+\ldots+n_k}}{n_1! \cdot \ldots \cdot n_k!} = \frac{\left(n_1+\ldots+n_k\right)!}{n_1!\cdot\ldots\cdot n_k!}$.
	
	Для $\widetilde{C}_n^m$ рассуждаем следующим образом.
	Необходимо выбрать из набора из $n$ различных типов предметов ровно $m$ предметов, причём каждый тип можно выбрать несколько раз (т.е. с повторениями), и порядок выбора не важен.
	Представим каждую из выбранных $m$ «штук» как звёздочку ($\ast$), а разделители между типами — как палочки ($|$).
	Всего типов $n$, значит между этими типами нужно разместить $n-1$ разделитель.
	Расположение звёзд и палочек вдоль одной оси однозначно кодирует, сколько звёзд (предметов) взято каждого типа.
	Всего символов («звёзд» и «палочек») $= m + (n-1)$. Из них нужно выбрать $m$ позиций для звёзд (или, эквивалентно, $n-1$ позиций для палочек). Таким образом число всех конфигураций равно числу способов выбрать $m$ позиций из $(m + n - 1)$, т.е.
	$\widetilde{C}_n^m =  C_{\,n + m - 1}^{\,m}. \Box$
	
	\subsubsection{Свойства комбинаторных величин}
	
	
	\begin{theorem}[о свойствах числа сочетаний]
		$$C_n^m=P(m,n-m) \quad (m\leqslant n);$$
		$$C_n^m=C_n^{n-m} \quad (m\leqslant n);$$
		$$C_n^m=\tfrac{n}{m}\cdot C_{n-1}^{m-1} \quad (m\leqslant n);$$
		$$C_n^m+C_n^{m+1}=C_{n+1}^{m+1} \quad (m<n).$$
	\end{theorem}
\textbf{Доказательство} проводится выписыванием с помощью факториалов левых и правых частей равенств и их сравнением. $\Box$
	
	\begin{theorem}[о биноме Ньютона]
		$$(x+y)^n = \sum_{k=0}^n C_{n}^{k} x^{n - k} y^k =$$
		$$={C_n^0}x^n + {C_n^1}x^{n - 1}y + \ldots + {C_n^k}x^{n - k}y^k + \ldots + {C_n^n}y^n.$$
	\end{theorem}
\textbf{Доказательство.} Чтобы получить член $x^{n - k}y^k$ при раскрытии $n$ скобок $(x+y)$ нужно выбрать $k$ скобок, из которых возьмём $y$ (из остальных автоматически выбирается $x$). Следовательно, при раскрытии скобок будет ровно $C_n^k$ слагаемых вида $x^{n - k}y^k$. $\Box$
	
	В связи с этой теоремой числа сочетаний $C_n^m=\binom{n}{m}$ называют биномиальными коэффициентами.
	
	Бином Ньютона может быть обобщён до мультинома Ньютона --- возведения в степень суммы произвольного числа слагаемых:
	\begin{theorem}[Мультином Ньютона]
		$$(x_1 + x_2 + \cdots + x_k)^n =\sum\limits_{n_j \geqslant 0 \atop n_1 + n_2 + \cdots + n_k = n} P(n_1, \cdots, n_k) x_1^{n_1} \ldots x_k^{n_k}.$$
	\end{theorem}
	
	В связи с формулой из теоремы число $P(n_1,  \cdots, n_k)$ также называется мультиномиальным (полиномиальным) коэффициентом и обозначается  $\binom{n}{n_1, n_2, \ldots, n_k}$.
	\medskip
	
	При $k=2$ мультиномиальный коэффициент совпадает с биномиальным $\binom{n_1+n_2}{n_1, n_2}=\binom{n_1+n_2}{n_1}=\binom{n_1+n_2}{n_2}$.
	\bigskip
	
	\textbf{Урновая схема.}
	Пусть имеется $n$ урн, пронумерованных от $1$ до $n$. 
	Разместим в урны произвольным образом $m$ шаров, где $m\leqslant n$.
	Число способов разместить шары зависит от двух условий: (1) различимы ли шары или нет; (2) имеет ли место принцип исключения, который не позволяет положить второй шар в урну, уже содержащую один шар. 
	Это число способов в зависимости от комбинации условий находится как одно из известных комбинаторных чисел:
	\bigskip
	
	\begin{tabular}[c]{|p{8em}|p{12em}|p{12em}|}
		\hline
		{Комбинация} & {с принципом исключений} & {без принципа исключений}  \\ \hline
		{шары различимы}    & $A_n^m=n^{[m]}={\frac{n!}{(n-m)!}}^{\phantom{M}}$ & $\widetilde{A}_n^m=n^m$ \\ \hline
		{шары неразличимы}  & $C_n^m=\frac{n!}{m!(n-m)!}$\phantom{MMMM}    & $\widetilde{C}_n^m=C_{n+m-1}^m={\frac{(n+m-1)!}{m!(n-1)!}}^{\phantom{M}}$ \\ \hline
	\end{tabular}
	\bigskip
	
	\textbf{Число решений} $\left(x_1, \ldots, x_n\right)$ уравнения 
	$$x_1+ \ldots + x_n=k$$
	в неотрицательных целых $x_i$
	%, где $x_i$ есть число появлений $i$-го элемента в данном сочетании. Это число
	равно числу $\widetilde{C}_n^k$ сочетаний с повторениями из $n$ элементов по $k$.
	\bigskip
	
	\textbf{Задача о беспорядках.}
	Требуется найти число перестановок $(p_1, p_2, \ldots, p_n)$ множества $\{1, 2, \ldots , n\}$ таких что $p_i \neq i$ для всех $i$. Такие перестановки называются \defin{беспорядками}{derangement}.
	
	\begin{theorem}[о числе беспорядков]
		Число беспорядков на множестве из $n$ элементов равно
		$n!\left(1-\frac{1}{1!}+\frac{1}{2!}-\frac{1}{3!}+ ... +(-1)^n\frac{1}{n!}\right).$
	\end{theorem}
	\textbf{Доказательство} следует из формулы включений-исключений. $\Box$
