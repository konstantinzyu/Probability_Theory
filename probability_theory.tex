 \documentclass[12pt]{article}
%\documentclass[12pt]{amsart}

\pagestyle{plain}
\usepackage[margin=3cm]{geometry} 

\usepackage{amsmath,amssymb,amsfonts,enumerate,latexsym, amsthm,textcomp,wasysym,longtable}

% \usepackage{indentfirst}
\usepackage[matrix, arrow, curve]{xy} % Для коммутативных диаграмм

\usepackage[utf8]{inputenc}
\usepackage[russian]{babel}
\usepackage{verbatim}
\makeatletter
\def\@settitle{\begin{center}%
		\baselineskip14\p@\relax
		\bfseries
		\large \@title
	\end{center}%
}
\makeatother

\usepackage{cancel}
\usepackage{graphicx}
% \graphicspath{{pictures/}}
% \DeclareGraphicsExtensions{.pdf,.png,.jpg}
%\usepackage{russian}

%%%%%%%%%%%%%%%%%%%%%%%%%%%%%%%%%%%%%%%%%%%%%%%%%%%%%%%%%%%%
% % commands for making comments
\usepackage[dvipsnames]{xcolor}
\newcommand{\YP}[1]{\footnote{\textcolor{red}{YP: #1}}}
\newcommand{\yp}[1]{\leavevmode{\color{red}{#1}}}
% {\textcolor{orange}{#1}} 
\usepackage[normalem]{ulem}
%%%%%%%%%%%%%%%%%%%%%%%%%%%%%%%%%%%%%%%%%%%%%%%%%%%%%%%%%%%%

\usepackage{hyperref}
\usepackage{tikz-cd} % Ещё для коммутативных диаграмм


% \textheight=270mm
% \textwidth=190mm
% \voffset=-40mm
% \hoffset=-35mm
% \pagestyle{empty}
% 
% \\SLoppy



\emergencystretch=5pt

\newtheorem{theorem}{Теорема}
\newtheorem{proposition}{Предложение}
\newtheorem*{definition}{Определение}
\newtheorem{lemma}[theorem]{Лемма}
\newtheorem{corollary}[theorem]{Следствие}


\numberwithin{theorem}{section}
\numberwithin{proposition}{section}

\theoremstyle{definition}

\newenvironment{example}{\indent \textbf{Пример.}}{\indent}

\newtheorem*{remark*}{Замечание}

\newcommand{\Alt}{\mathfrak{A}}
\newcommand{\Sym}{\mathfrak{S}}
\newcommand{\Q}{\mathrm{Q}}
\newcommand{\D}{\mathrm{D}}
\newcommand{\Dic}{\mathrm{Dic}}
\newcommand{\rC}{\mathrm{C}}
\newcommand{\T}{\mathrm{T}}
\newcommand{\rO}{\mathrm{O}}
\newcommand{\I}{\mathrm{I}}
\newcommand{\CC}{\mathbb{C}}
\newcommand{\RR}{\mathbb{R}}
\newcommand{\FF}{\mathbb{F}}
\newcommand{\EE}{\mathbb{E}}
\newcommand{\KK}{\mathbb{K}}
\newcommand{\LL}{\mathbb{L}}

\newcommand{\calA}{\mathcal{A}}
\newcommand{\calB}{\mathcal{B}}
\newcommand{\calE}{\mathcal{E}}
\newcommand{\calP}{\mathcal{P}}

\newcommand{\Gal}{\operatorname{Gal}}
\newcommand{\Aut}{\operatorname{Aut}}
\newcommand{\Tor}{\operatorname{T}}
\newcommand{\AGL}{\operatorname{AGL}}
\newcommand{\GL}{\operatorname{GL}}
\newcommand{\Qrn}{\operatorname{Q}_8}
\newcommand{\SL}{\operatorname{SL}}
\newcommand{\PGL}{\operatorname{PGL}}
\newcommand{\PSL}{\operatorname{PSL}}
\newcommand{\PSU}{\operatorname{PSU}}
\newcommand{\SU}{\operatorname{SU}}
\newcommand{\SO}{\operatorname{SO}}
\newcommand{\diag}{\operatorname{diag}}
\newcommand{\projective}{\mathbb{P}}
\newcommand{\affine}{{\mathbb{A}}}
\newcommand{\characteristic}{\operatorname{char}}
\newcommand{\rank}{\operatorname{rank}}
\newcommand{\rd}{\operatorname{rd}}
\newcommand{\ed}{\operatorname{ed}}
\newcommand{\id}{\operatorname{id}}

\newcommand{\ab}[1]{#1^{\mathrm{ab}}}

\newcommand{\prob}{\operatorname{P}}
\newcommand{\defin}[2]{\hypertarget{#2}{{\color{red} #1}}}
\newcommand{\events}{\mathfrak{F}}
\newcommand{\expect}{\operatorname{E}}
\newcommand{\disp}{\operatorname{D}}
\newcommand{\cov}{\operatorname{Cov}}

% Цвета
\definecolor{linkcolor}{HTML}{0000FF} % цвет ссылок
\definecolor{urlcolor}{HTML}{0000FF} % цвет гиперссылок
\definecolor{citecolor}{HTML}{0000FF} % цвет ссылки на статью
\hypersetup{pdfstartview=FitH, linkcolor=linkcolor, urlcolor=urlcolor, citecolor=citecolor, colorlinks=true}

% Пробелы, отступы и выделения
\definecolor{todocolor}{HTML}{FF4500} % цвет todo
\newcommand{\TODO}[1]{\textcolor{todocolor}{НУЖНО: #1}}
\renewcommand\labelenumi{\rm (\arabic{enumi})}
\renewcommand\theenumi{\rm (\arabic{enumi})}
% Определение множества
\newcommand{\defineset}[2]{\left\{
	\left.
	#1
	\right\vert
	#2
	\right\}}

% Кусочное определение функции
\newcommand{\definefuntwo}[4]{
	\begin{cases}
		#1, & #2; \\
		#3, & #4.
	\end{cases}
}

\newcommand{\spmatrix}[4]{
	\left( \begin{smallmatrix}
		#1 & #2 \\
		#3 & #4
	\end{smallmatrix} \right)
}


\setcounter{section}{-1}

\title{Теория вероятностей}
\author{(Ещё не)алгебраист}


\begin{document}
	\maketitle
	
	\section*{Предисловие}
	
	Эти записки созданы с целью аккуратно формализовать и заполнить пробелы в лекциях Елены Борисовны Яровой.
	В разделе \ref{preparing} будут содержаться основные принятые в курсе обозначения, 
	а также сведения и определения из разных разделов математики, которыми автор будет пользоваться.
	Поскольку автор считает полезным взгляд на всякий раздел математики с точки зрения теории категорий и её приложений, 
	этот язык также будет упоминаться (тем не менее, не замещая собой прочие подходы).
	
	\tableofcontents
	
	
	\section{Предварительные сведения} \label{preparing}
	
	\subsection{Обозначения}
	
	%В работе приняты следующие обозначения:
	
	\begin{itemize}
		\item $ \Omega $ --- пространство элементарных исходов;
		\item $ \omega $ --- элементарный исход;
		\item $ \events $ --- $ \sigma $-алгебра событий;
		\item $ \prob $ --- вероятностная мера;
		\item $ \xi, \eta, \zeta $ --- случайные величины;
		\item $ \expect \xi $ --- математическое ожидание случайной величины $ \xi $;
		\item $ \disp \xi $ --- дисперсия случайной величины $ \xi $;
		\item $ \cov(\xi, \eta) $ --- ковариация случайных величин $ \xi $ и $ \eta $;
		\item $ \rho(\xi, \eta) $ --- корреляция случайных величин $ \xi $ и $ \eta $;
	\end{itemize}
	
	\subsection{Предварительные сведения из действительного анализа}
	
	\subsubsection{($ \sigma $-)Алгебры и меры}
	
	Пусть $ \Omega $ --- некоторое множество.
	
	Система множеств $ S $ называется полуцольком, если выполнены следующие аксиомы
	\begin{enumerate}
		\item $ \varnothing \in S $;
		\item $ A, B \in S: A \cap B \in S $;
		\item $ A, B \in S, A \subset B \ \exists n \in \mathbb{N} \exists C_1, \ldots, C_n \in S: 
		A = B \sqcup \bigsqcup\limits_{k = 1}^{n} C_k  $
	\end{enumerate}
	
	Система множеств (следует понимать как синоним термина <<семейство множеств>>) 
	$ R \subset 2^{\Omega} $ называется \defin{алгеброй с единицей $ \Omega $}{algebra}, если выполнены первые три из следующих аксиом и \defin{$ \sigma $-алгеброй с единицей $ \Omega $}{sigma-algebra}, если выполнены все четыре аксиомы.
	\begin{enumerate}
		\item $ \Omega \in R $;
		\item $ \forall \ A, B \in R: A \cup B, A \cap B \in R $;
		\item $ \forall \ A \in R: \ \Omega \setminus A := \overline{A} \in R $;
		\item $ \forall \ \{A_k\}_{k \in R} \subset R: \ \bigcup\limits_{k \in \mathbb{N}} A_k \in R $.
	\end{enumerate}
	 
	Далее, если не оговорено иное, все алгебры являются ($ \sigma $-)алгебрами с единицей $ \Omega $
	и будут называться <<($ \sigma $-)алгебрами>>.
	
	Будем называть функцию $ \mu \colon R \to \RR $ \defin{(конечной) мерой}{measure} на ($ \sigma $-)алгебре $ R $, 
	если $ \nu $ удовлетворяет аксиоме аддитивности
	$$ \forall A, B \in R, A \cap B = \varnothing \ \mu(A \sqcup B) = \mu(S) + \mu(B). $$
	Если дополнительно для любой последовательности попарно непересекающихся подмножеств $ \{A_k\}_{k \in \mathbb{N}} $,
	объединение которых есть элемент $ R $ (отметим, что это автоматически выполнено для $ \sigma $-алгебры)
	имеет место равенство 
	$$ \mu\left(\bigsqcup\limits_{k = 1}^{\infty} A_k \right) = \sum\limits_{k = 1}^{\infty} \mu(A_k), $$
	то мера $ \mu $ называется \defin{$ \sigma $-аддитивной}{sigma-measure} (аксиома $ \sigma $-аддитивности).
	Можно показать, что из этой аксиомы следует, что $ \mu(\varnothing) = 0 $ и поэтому из неё следует аксиома аддитивности.
	
	\begin{lemma} \label{algebra restriction}
		Пусть $ \events $ --- ($ \sigma $-)алгебра и $ A \in \events $.
		Тогда множество $$ \events \cap A := \defineset{B \cap A}{B \in \events} $$ является ($ \sigma $-)алгеброй
		с единицей $ A $.
	\end{lemma}
	
	\begin{proof}
		По построению $ \Omega \cap A = A $ содержится в $ \events \cap A $.
		
		Пусть теперь $ C_1 = B_1 \cap A, C_2 = B_2 \cap A \in \events \cap A $ --- два множества.
		Тогда $ C_1 \cap C_2 = (B_1 \cap B_2)  \cap A \in \events \cap A $, так как $ B_1 \cap B_2 \in \events $.
		Далее, $ C_1 \cup C_2 = (B_1 \cup B_2) \cap A \in \events \cap A $, так как $ B_1 \cup B_2 \in \events $.
		Окончательно, $ A \setminus C_1 = (A \setminus B_1) = (\Omega \setminus B_1) \cap A \in \events \cap A $,
		поскольку $ \Omega \setminus B_1 \in \events $.
		
		Предположим, что $ \events $ являлось $ \sigma $-алгеброй. Пусть $ \{C_k\} $ --- счётное семейство 
		элементов $ \events \cap A $ и $ C_k = B_k \cap A $.
		Тогда
		$$ \bigcup\limits_{i = 1}^{\infty} C_k = \bigcup\limits_{i = 1}^{\infty} (B_k \cap A) 
		= \left(\bigcup\limits_{i = 1}^{\infty} B_k \right) \cap A \in \events \cap A, $$
		принадлежность справедлива в силу того, что $ \bigcup\limits_{i = 1}^{\infty} B_k \in \events $.
	\end{proof}
	
	\subsubsection{Лебеговское продолжение меры}
	
	\subsubsection{Мера Лебега-Стилтьеса}
	
	\subsubsection{Измеримое отображение}
	
	\subsubsection{Интеграл Лебега}
	
	\subsubsection{Прямой образ меры (pushforward measure)}
	
	\subsection{Предварительные сведения из анализа Фурье}
	
	\subsection{Предварительные сведения из линейной алгебры}
	
	\subsubsection{Билинейные функции и квадратичные формы}
	
	Пусть $ \Bbbk $ --- некоторое поле (в нашем случае будут рассматриваться только поля вещественных чисел $ \RR $) и $ V $ --- векторное пространство над $ \Bbbk $.
	
	Отображение $ B \colon V \times V \to \Bbbk $ называется \defin{билинейной функцией}{bilinear}, если выполнены следующие аксиомы
	\begin{enumerate}
		\item $ \forall v,u,w \in V \ B(u + v, w) = B(u, w) + B(v, w) $;
		\item $ \forall v,u \in V, \lambda \in \Bbbk \ B(\lambda u, v) = \lambda B(u, v) $;
		\item $ \forall v,u,w \in V \ B(u, v + w) = B(u, w) + B(u, v) $;
		\item $ \forall v,u \in V, \lambda \in \Bbbk \ B(u, \lambda v) = \lambda B(u, v) $.
	\end{enumerate}
	
	Билинейная функция называется \defin{симметрической}{symmetric}, если дополнительно для любых $ u, v \in V $ выполнено $ B(u, v) = B(v, u) $.
	
	\begin{example}
		Пусть $ V = \Bbbk $ и $ B(a, b) = a \cdot b $, где $ \cdot $ --- умножение в поле $ \Bbbk $.
		Тогда $ B $ --- симметрическая билинейная функция.
	\end{example}
	
	\begin{example}
		Пусть в векторном пространстве $ V $
		фиксирован базис $ e_1, \ldots, e_n $. Тогда если $ B(x, y) = \sum\limits_{i = 1}^{n} x_iy_i $, где $ x = \sum\limits_{i = 1}^{n} x_ie_i $ и $ y = \sum\limits_{i = 1}^{n} y_ie_i $, то $ B $ --- также билинейная симметрическая форма.
	\end{example}
	
	\defin{Квадратичной формой}{quadratic} называется отображение $ Q \colon V \to \Bbbk $ такое, что для некоторой билинейной формы и любой вектора $ v \in V $ имеет место равенство $ Q(v) = B(v, v) $.
	Если $ B $ --- билинейная функция, то квадратичная форма $ Q $, заданная формулой $ Q(v) = B(v, v) $
	называется квадратичной формой соответствующей билинейной функции $ B $.
	Пусть $ \Bbbk = \RR $, $ Q $ --- квадратичная форма 
	и для любого ненулевого вектора $ v \in V $ выполнено неравенство $ Q(v) > 0 $.
	Тогда форма $ Q $ называется положительно определённой. 
	Если для любого $ v \in V $ выполнено неравенство $ Q(v) \geqslant 0 $,
	то форма $ Q $ называется неотрицательно определённой.

	Симметрическую билинейную форму с положительно определённой соответствующей квадратичной формой называют 
	\defin{скалярным произведением}{inner-product}. Вместо $ B(u,v) $ часто пишут $ (u, v) $ или $ \left<u, v\right> $.
	
	Примеры. Квадратичные формы, соответствующие билинейным функциям из примеров выше являются положительно определёнными.
	
	\begin{theorem}[Коши, Буняковский, Шварц] \label{Cauchy-real}
		Пусть $ V $ --- векторное пространство над полем $ \RR $ и $ B $ --- скалярное произведение на $ V $.
		Тогда дл любых двух векторов $ u, v \in V $ выполнено равенство
		$$ B(u, v)^2 \leqslant B(u,u)B(v,v), $$
		причём равенство достигается тогда и только тогда, когда $ u $ и $ v $ коллинеарны.
	\end{theorem}
	
	\begin{proof}
		Рассмотрим вектор $ u + tv $, где $ t \in \RR $ и значение квадратичной формы на нём.
		По билинейности, симметричности и положительной определённости имеем 
		$$ B(u + tv, u + tv) = B(u, u) + tB(u, v) + tB(v, u) + t^2B(v,v) = B(u,u) + 2tB(u,v) + t^2B(v,v) \geqslant 0, $$
		причём последнее равенство достигается тогда и только тогда, когда $ u + tv = 0 $.
		
		Многочлен второй степени принимает только неотрицательные (положительные) значения тогда и только тогда, когда его дискриминант меньше или равен 0 (меньше 0).
		Итого $$ D = 4B(u,v)^2 - 4B(u,u)B(v,v) \leqslant 0  \Leftrightarrow B(u,v)^2 \leqslant B(u,u)B(v,v) $$
		и $ D = 0 \Leftrightarrow B(u,v)^2 = B(u,u)B(v,v) $. Последнее равносильно тому, что многочлен имеет корень $ t $
		и $ u + tv = 0 $, то есть $ u $ и $ v $ пропорциональны.
	\end{proof}
	
	Заметьте, что доказательство этого неравенства в случае поля комплексных чисел требует добавления дополнительной <<поправки>> $ \lambda $.
	
	\subsubsection{Полуторалинейные функции}
	
	В этом подразделе будем рассматривать только векторные пространства над полем комплексных чисел.
	
	Отображение $ S \colon V \times V \to \Bbbk $ называется \defin{полуторалинейной функцией (по второму аргументу)}{sesquilinear}, если выполнены следующие аксиомы
	\begin{enumerate}
		\item $ \forall v,u,w \in V \ S(u + v, w) = S(u, w) + S(v, w) $;
		\item $ \forall v,u \in V, \lambda \in \Bbbk \ S(\lambda u, v) = \lambda S(u, v) $;
		\item $ \forall v,u,w \in V \ S(u, v + w) = S(u, w) + S(u, v) $;
		\item $ \forall v,u \in V, \lambda \in \Bbbk \ S(u, \lambda v) = \overline{\lambda} S(u, v) $,
		где надчёркивание означает комплексное сопряжение.
	\end{enumerate}
	
	Полуторалинейная функция называется \defin{эрмитовой}{hermitian}, если для любых векторов $ u $ и $ v $ дополнительно выполнено равенство
	$ S(u, v) = \overline{S(v, u)} $.
	
	Эрмитова функция называется \defin{скалярным произведением}{C-inner-product}, если для любого ненулевого вектора $ v $
	выполнено неравенство $ S(v, v) > 0 $.
	
	
	\begin{theorem}[Коши, Буняковский, Шварц] \label{Cauchy-complex}
		Пусть $ V $ --- векторное пространство над полем $ \CC $ и $ S $ --- скалярное произведение на $ V $.
		Тогда для любых двух векторов $ u, v \in V $ выполнено равенство
		$$ S(u, v)\overline{S(u,v)} \leqslant S(u,u)S(v,v), $$
		причём равенство достигается тогда и только тогда, когда $ u $ и $ v $ коллинеарны.
	\end{theorem}
	
	\begin{proof}
		Если $ S(u, v) = 0 $, то неравенство выполнено. При таком условии $ u $ и $ v $ пропорциональны тогда и только тогда,
		когда один из этих векторов равен 0. Последнее в свою очередь равносильно тому, что правая часть неравенства обращается в нуль. Далее будем считать, что $ S(u,v) \neq 0 $.
		
		Рассмотрим вектор $ u + t\lambda v $, где $ t \in \RR $ и $ \lambda = S(u,v) $.
		Поскольку $ S $ --- скалярное произведение и из условий наложенных на $ 
		\lambda $, то 
		$$ S(u + t\lambda v, u + t\lambda v) 
		= S(u, u) + t\overline{\lambda}S(u, v) + t\lambda S(v, u) + t^2\lambda\overline{\lambda}S(v,v) = $$ 
		$$ = S(u,u) + 2tS(u,v)S(v,u) + t^2S(u,v)S(v,u)S(v,v) \leqslant 0 $$
		причём последнее равенство достигается тогда и только тогда, когда $ u + t\lambda v = 0 $.
		
		Многочлен второй степени принимает только неотрицательные (положительные) значения тогда и только тогда, когда его дискриминант меньше или равен 0 (меньше 0).
		Итого $$ D = 4S(u,v)^2S(v,u)^2 - 4S(u,u)S(v,v)S(u,v)S(v,u) \leqslant 0  \Leftrightarrow S(u,v)S(v,u) \leqslant S(u,u)S(v,v) $$
		и $ D = 0 \Leftrightarrow S(u,v)^2 = S(u,u)S(v,v) $. Последнее равносильно тому, что многочлен имеет корень $ t_0 $
		и $ u + t_0S(u, v)v = 0 $, то есть $ u $ и $ v $ пропорциональны.
	\end{proof}
	
	\subsection{Теория категорий и взгляд на измеримые пространства с её точки зрения}
	
	\subsubsection{Категория измеримых пространств}
	
	\subsubsection{Прямой образ $ \sigma $-алгебры}
	
	\subsubsection{Обратный образ $ \sigma $-алгебры}
	
	\subsubsection{Функтор борелевской $ \sigma $-алгебры}
	
	\section{Вероятностное пространство, случайные события}
	
	Пусть $ \Omega $ --- некоторое множество, $ \mathfrak{F} $ --- $ \sigma $-алгебра с единицей $ \Omega $
	и $ \prob $ --- $ \sigma $-аддитивная мера на $ \mathfrak{F} $, удовлетворяющая свойству $ \prob(\Omega) = 1 $. 
	Тогда тройка $ (\Omega, \mathfrak{F}, \prob) $ называется \defin{вероятностным пространством}{prob-space}.
	Множество $ \Omega $ называется \defin{пространством элементарных событий (исходов)}{space},
	элементы $ \sigma $-алгебры $ \mathfrak{F} $ называются \defin{событиями}{event}.
	
	Вероятностное пространство называется \defin{дискретным}{discr}, если множество $ \Omega $ не более, чем счётно.
	
	Для кратности, если множество $ \{\omega\} $ является событием, вместо $ \prob({\omega}) $ будем писать $ \prob(\omega) $.
	
	Примеры. Пусть $ \Omega = \{1,2,3,4,5,6\} $ --- числа, возникающие при броске игральной кости.
	Будем считать, что все элементарные исходы равновероятны, 
	то есть $ \prob(1) = \prob(2) = \prob(3) = \prob(4) = \prob(5) = \prob(6) = \tfrac{1}{6} $.
	Тогда вероятность события $ A = \{2,4,6\} $ --- <<>выпало чётное число> 
	равна $ \prob(A) = \prob(2) + \prob(4) + \prob(6) = \tfrac{1}{6} + \tfrac{1}{6} + \tfrac{1}{6} = \tfrac{1}{2} $.
	
	Рассмотренный пример мотивирует нас ввести параллельные определения для дискретного пространства.
	\defin{Дискретным вероятностным пространством}{discr-2} мы будем называть  пару $ (\Omega, \prob) $, 
	где $ \Omega = \{\omega_k\}_{k \in \mathbb{N}} $ --- не более чем счётное множество 
	(также называемое \defin{пространством элементарных исходов}{space-discr}),
	а $ \prob \colon \Omega \to \RR $ --- неотрицательная функция, удовлетворяющая свойству
	$ \sum\limits_{k \in \mathbb{N}} \prob(\omega_k) = 1 $.	
	Говорят, что в этом случае на $ \Omega $ \defin{заданы вероятности элементарных событий}{prob-defined} и что функция $ \prob $ 
	\defin{задаёт на $ \Omega $ распределение вероятностей}{disrtib-discr}.
	\defin{Событиями}{event-discr} называются подмножества $ \Omega $. 
	\defin{Вероятностью события}{prob-discr} $ A \subset \Omega $ называется величина
	$$ \prob(A) = \sum\limits_{\omega \in A} P(\omega), $$
	которую мы также будем обозначать буквой $ \prob $. 
	Последнее данное определение корректно, поскольку ряд в правой части сходится абсолютно.
	
	\begin{proposition} \label{}
		Пусть $ (\Omega, \prob) $ --- дискретное вероятностное пространство в смысле \hyperlink{discr-2}{последнего определения}.
		Пусть $ \prob \colon 2^{\Omega} \to \mathbb{R} $ --- функция, сопоставляющая событию его вероятность.
		Тогда тройка $ (\Omega, 2^{\Omega}, \prob) $ является вероятностным пространством в смысле \hyperlink{prob-space}{исходного определения}.
	\end{proposition}
	
	\begin{proof}
		
		Множество $ 2^{\Omega} $ является $ \sigma $-алгеброй, поэтому достаточно проверить, что функция $ \prob $
		удовлетворяет аксиомам вероятностной меры.
		
		Из определения $ \prob $ имеем
		$$ \prob(\Omega) = \sum\limits_{i = 1}^{+\infty} \prob(\omega_i) = 1. $$
		Пусть $ A, B \subset \Omega $ и $ A \cap B = \varnothing $.
		Положим $ A = \{\omega_{i}\}_{i \in I_A} $, $ B = \{\omega_{i}\}_{i \in I_B} $ и 
		$ A \sqcup B = \{\omega_i\}_{i \in I_{A \sqcup B}} $. 
		Поскольку $ A $ и $ B $ не пересекаются, то $ I_A \sqcup I_B =  I_{A \sqcup B} $.
		Тогда, так как ряды в формуле ниже сходятся абсолютно, имеем
		$$ \prob(A \sqcup B) = \sum\limits_{i \in I_{A \sqcup B}} \omega_i = 
		\sum\limits_{i \in I_A} \omega_{i} + \sum\limits_{i \in I_B} \omega_{i}
		= \prob(A) + \prob(B). $$
		
		Пусть теперь $ \{A_k\}_{k \in \mathbb{N}} $ --- счётное семейство непересекающихся подмножеств множества $ \Omega $.
		Положим $ A_k = \{\omega_{i}\}_{i \in I_k} $, $ A = \bigsqcup\limits_{k \in I} A_k $.
		Снова, поскольку $ A_k $ попарно не пересекаются, то $ I = \bigsqcup\limits_{k \in \mathbb{N}} I_k $.
		Поскольку все ряды ниже сходятся абсолютно, то выполнены равенства
		$$ \prob(A) = \sum\limits_{i \in I} \prob(\omega_i) 
		= \sum\limits_{k \in \mathbb{N}} \sum\limits_{i \in I_k} \prob(\omega_i) 
		= \sum\limits_{k \in \mathbb{N}} \prob(A_k). $$
	\end{proof}
	
	Пусть $ A, B \in \events $ --- события. Введём основные операции над событиями 
	и приведём их классические наименования и обозначения в теории вероятностей. 
	
	Событие $ \Omega \setminus A $ называется
	\defin{дополнением к событию $ A $}{event-compl}
	и обозначается $ \overline{A} $ (<<событие $ A $ не произошло>>).
	
	Событие $ A \cup B $ называется
	\defin{суммой событий $ A $ и $ B $}{event-sum}
	и обозначается $ A + B $  (<<произошло событие $ A $ или $ B $>>). 
	В курсе лекций это обозначение использовалось для случаев, когда $ A \cap B = \varnothing $.
	
	Событие $ A \cap B $ называется 
	\defin{произведением событий $ A $ и $ B $}{event-product} 
	и обозначается $ AB $ (<<произошло и событие $ A $ и событие $ B $>>).
	
	События $ \Omega $ и $ \varnothing $ называются \defin{достоверным}{} и \defin{невозможным}{}, соответственно.
	
	Если $ AB = \varnothing $, то события $ A $ и $ B $ называются \defin{несовместными}{}.
	(<<события $ A $ и $ B $ не происходят одновременно>>).
	
	\begin{proposition}[Начальные свойства вероятностной меры]
		Пусть $ A, B, A_k \in \events $ --- события.
		Тогда имеет место следующее:
		\begin{enumerate}
			\item $ \prob(\overline{A}) = 1 - \prob(A); $ \label{prob-prop1}
			\item если $ A \subset B $, то $ \prob(B \setminus A) = \prob(B) - \prob(A); $ \label{prob-prop2}
			\item если $ A \subset B $, то $ \prob(A) \leqslant \prob(B); $ \label{prob-prop2.5}
			\item $ \prob(A \cup B) = \prob(A) + \prob(B) - \prob(AB); $ \label{prob-prop3}
			\item $ \prob(A \cup B) \leqslant \prob(A) + \prob(B) $; \label{prob-prop4}
			\item $ \prob(\bigcup\limits_{k = 1}^{n} A_k) = \sum\limits_{k = 1}^{n} 
			(-1)^{k - 1}\sum\limits_{i_1 < i_2 < \ldots < i_k} \prob(A_{i_1} \ldots A_{i_k}); $ \label{prob-prop5}
			\item $ \prob\left(\bigcup\limits_{k = 1}^{+\infty} A_k \right) \leqslant \sum\limits_{k = 1}^{+\infty} \prob(A_k) $ 
			(это свойство называется \defin{субаддитивностью}{subadditivity}). \label{prob-prop6}
		\end{enumerate}
	\end{proposition}
	
	\begin{proof}
		Равенство \ref{prob-prop1} следует из цепочки 
		$$ 1 = \prob(\Omega) = \prob(A \sqcup \overline{A}) = \prob(A) + \prob(\overline{A}). $$
		
		Равенство \ref{prob-prop2} --- из цепочки 
		$$ \prob(B) = \prob(A \cup (B \setminus A)) = \prob(A) + \prob(B \setminus A). $$
		Неравенство \ref{prob-prop2.5} следует из этого равенства и неотрицательности вероятности.
		
		Равенство \ref{prob-prop3} --- из цепочки
		$$ \prob(A \cup B) = \prob((A \setminus B) \sqcup (A \cap B) \sqcup (B \setminus A)) = $$ 
		$$ = \prob(A \setminus B) + \prob(A \cap B) + \prob(B \setminus A) + \prob(A \cap B) - \prob(A \cap B) = $$
		$$ = \prob((A \setminus B) \sqcup (A \cap B)) + \prob((B \setminus A) \sqcup (A \cap B)) - \prob(A \cap B) = $$
		$$ = \prob(A) + \prob(B) - \prob (A \cap B). $$
		Неравенство \ref{prob-prop4} немедленно следует из равенства \ref{prob-prop3}.
		
		Докажем \ref{prob-prop5} по индукции.
		
		База $ n = 2 $ была доказана в пункте 3.
		
		Докажем шаг.
		Положим $ B = \bigcup\limits_{k = 1}^{n - 1} A_k $.
		По базе индукции $$ \prob(B \cup A_{n}) = \prob(B) + \prob(A_n) - \prob(B A_n). $$
		Далее, положим $ B_k = A_kA_n $. Тогда $ BA_n = \bigcup\limits_{k = 1}^{n - 1} B_k $.
		По индукционному предположению вероятность $ \prob(B \cup A_n) $ равна
		$$ \sum\limits_{k = 1}^{n - 1} 
		(-1)^{k - 1}\sum\limits_{i_1 < i_2 < \ldots < i_k} \prob(A_{i_1} \ldots A_{i_k})
		+ \prob(A_n)
		- \left(\sum\limits_{k = 1}^{n - 1} 
		(-1)^{k - 1}\sum\limits_{i_1 < i_2 < \ldots < i_k} \prob(A_{i_1} \ldots A_{i_k} A_{n})\right) = $$
		$$ = \sum\limits_{k = 1}^{n} 
		(-1)^{k - 1}\sum\limits_{i_1 < i_2 < \ldots < i_k} \prob(A_{i_1} \ldots A_{i_k}). $$
		
		Докажем неравенство $ \ref{prob-prop6} $.
		Положим $ B_k = A_k \setminus \bigcup\limits_{i = 1}^{k - 1} A_i $.
		Тогда $ \bigcup\limits_{k = 1}^{+\infty} B_k = \bigcup\limits_{k = 1}^{+\infty} A_k $,
		причём  $ B_k $ попарно не пересекаются и $ \prob(B_k) \leqslant \prob(A_k) $ по $ \ref{prob-prop2} $.
		Тогда по $ \sigma $-аддитивности имеем 
		$$ \bigcup\limits_{k = 1}^{+\infty} A_k = \bigcup\limits_{k = 1}^{+\infty} B_k 
		= \sum\limits_{k = 1}^{+\infty} \prob(B_k) \leqslant \sum\limits_{k = 1}^{+\infty} \prob(A_k). $$
	\end{proof}
	
	\section{Условные вероятности, формула Байеса, независимость событий}
	
	\subsection{Условная вероятность}
	
	В задачах бывает полезно рассмотреть вероятность того, что произойдёт некоторое событие $ B $ при условии, 
	что произойдёт событие $ A $. Пусть $ \prob(A) > 0 $. Тогда вероятность $ \prob(B \mid A) = \tfrac{\prob(AB)}{\prob(A)} $
	называется \defin{условной вероятностью события $ B $ при условии того, 
	что событие $ A $ произойдёт с вероятностью $ \prob(A) > 0 $}{conditional}.
	Вероятность $ \prob(B) $ также иногда называется \defin{априорной вероятностью события $ B $}{apriori}.
		
	\begin{proposition}
		Пусть $ (\Omega, \events, \prob) $ --- вероятностное пространство.
		Пусть $ {A \in \events} $ --- событие, удовлетворяющее условию $ \prob(A) > 0 $.
		Тогда тройка $$ (\Omega, \events, \left.\prob\right|_{A}), $$
		где $ \left.\prob\right|_{A}(B) := \prob(B \mid A) = \tfrac{\prob(AB)}{\prob(A)} $,
		является вероятностным пространством.
	\end{proposition}
	
	\begin{proof}
		Достаточно проверить аксиомы вероятностной меры 
		(аксиомы $ \sigma $-аддитивной меры и равенство $ \left.\prob\right|_{A}(\Omega) = 1 $).
		
		Так как обе величины $ \prob(AB) $ и $ \prob(A) $ неотрицательны (а последняя и вовсе положительна),
		то $ \prob(B \mid A) \leqslant 0 $.
		
		Справедливость упомянутого равенства выводится из определения условной вероятности: 
		$$ \left.\prob\right|_{A}(\Omega) = \tfrac{\prob(A\Omega)}{\prob(A)} = \tfrac{\prob(A)}{\prob(A)} = 1. $$
		
		Пусть $ \{B_k\}_{k \in \mathbb{N}} $ --- 
		счётная последовательность попарно не пересекающихся элементов алгебры $ \events $. 
		Тогда элементы последовательности $ \{B_k \cap A\}_{k \in \mathbb{N}} $ также попарно не пересекаются.
		Тогда
		$$ \left.\prob\right|_{A}\left(\bigsqcup\limits_{k = 1}^{+\infty} B_k \right)
		= \tfrac{1}{\prob(A)}\prob\left(A \cap \bigsqcup\limits_{k = 1}^{+\infty} B_k \right)
		= \tfrac{1}{\prob(A)}\prob\left(\bigsqcup\limits_{k = 1}^{+\infty} AB_k \right)
		= \sum\limits_{k = 1}^{+\infty} \tfrac{\prob(AB_k)}{\prob(A)} 
		= \sum\limits_{k = 1}^{+\infty} \left.\prob\right|_{A}(B_k). $$
	\end{proof}
	
	\begin{corollary}
		Пусть $ A \in \events $ --- событие, вероятность которого больше $ 0 $, $ B_1, B_2 \in \events $.
		Тогда справедливы следующие свойства
		\begin{enumerate}
			\item если $ B_1 \supset A $, то $ \prob(B_1 \mid A) = 1; $
			\item $ \prob(B_1 \cup B_2 \mid A) = \prob(B_1 \mid A) + \prob(B_2 \mid A) - \prob(B_1B_2 \mid A); $
			\item если $ B_1 $ и $ B_2 $ несовместны, то $ \prob(B_1 + B_2 \mid A) = \prob(B_1 \mid A) + \prob(B_2 \mid A). $
		\end{enumerate}
	\end{corollary}
	
	\subsection{Формула полной вероятности и формула Байеса}
	
	Теперь мы покажем, как связаны условные вероятности с вероятностями произведений событий, 
	как можно вычислять вероятность события, зная его условные вероятности для несовместных событий (формула полной вероятности)
	и как можно вычислить условную вероятность <<с переставленными причиной и следствием>> (формула Байеса).
	
	\begin{lemma} \label{cond prob}
		Пусть $ A, B \in \events $ --- события и $ \prob(A), \prob(B) > 0 $.
		Тогда имеют место равенства
		$$ \prob(AB) = \prob(A \mid B)\prob(B) = \prob(B \mid A)\prob(A), $$
		$$ \prob(A \mid B) = \tfrac{\prob(B \mid A)\prob(A)}{\prob(B)}. $$
	\end{lemma}
	
	\begin{proof}
		Первое равенство немедленно следует из определения условной вероятности, второе -- немедленно из первого
		и предположения, что $ \prob(B) > 0 $.
	\end{proof}
	
	Второе равенство, доказанное в лемме иногда (особенно в школьных программах), называют формулой Байеса.
	Ниже, пользуясь этим простым свойством, мы докажем более общую формулу и в дальнейшем будем называть формулой Байеса её.
	
	\begin{theorem}[Формула произведения вероятностей] \label{multiplication law}
		Пусть $ A_1, \ldots, A_n \in \events $ --- события.
		Если вероятности событий $ A_2A_3\ldots A_n, \ldots, A_{n - 1}A_n, A_n $ не равны нулю, то имеет место формула
		$$ \prob(A_1A_2\ldots A_n) = \prob(A_1 \mid A_2 \ldots A_n)\prob(A_2 \mid A_3 \ldots A_n) 
		\ldots \prob(A_{n - 1}\mid A_n)\prob(A_n). $$
		Если вероятности событий $ A_1A_2\ldots A_n, A_1A_2\ldots A_{n - 1}, \ldots, A_1 $ не равны нулю, то имеет место формула
		$$ \prob(A_1A_2\ldots A_n) = \prob(A_n \mid A_1\ldots A_{n-1})\prob(A_{n - 1} \mid A_1 \ldots A_{n - 2}) 
		\ldots \prob(A_{2}\mid A_1)\prob(A_1). $$
	\end{theorem}
	
	\begin{proof}
		Докажем первое утверждение индукцией по $ n $, второе получается из первого перестановкой индексов в обратном порядке.
		
		База: $ n = 2 $ есть определение условной вероятности.
		
		Докажем шаг индукции. Пусть для $ n - 1 $ утверждение выполнено.
		Положим $ B = A_2A_3\ldots A_n $.
		Тогда по базе индукции (здесь мы пользуемся тем, что $ \prob(B) > 0 $) 
		и затем по индукционному предположению (а здесь всеми остальными условиями) имеем
		$$ \prob(A_1B) = \prob(A_1 \mid B)\prob (B) = \prob(A_1 \mid A_2A_3 \ldots A_n)\prob(A_2 \mid A_3 \ldots A_n)
		\ldots \prob(A_{n - 1}\mid A_n)\prob(A_n). $$
	\end{proof}
	
	Набор событий $ A_1, \ldots, A_n \in \events $ называется \defin{разбиением пространства $ \Omega $}{exclusive-exhaustive}
	(или просто <<разбиение $ \Omega $>>),
	если $ \prob(A_i) > 0 $ для каждого $ i $, $ A_i $ попарно несовместны ($ A_iA_j = \varnothing $ при $ i \neq j $)
	и $ {A_1 + A_2 + \ldots + A_n = \Omega} $.
	
	\begin{theorem}[Формула полной вероятности] \label{law of total probability}
		Пусть $ A_1, \ldots, A_n \in \events $ --- разбиение $ \Omega $. 
		Тогда для всякого события $ B $ имеет место равенство формула полной вероятности
		$$ \prob(B) = \sum\limits_{i = k}^{n} \prob(B \mid A_k)\prob(A_k). $$
	\end{theorem}
	
	\begin{proof}
		Так как события $ A_k $ попарно несовместны, то события $ A_kB $ также попарно несовместны.
		Имеем цепочку равенств
		$$ \prob(B) = \prob(B\Omega) = \prob(B(A_1 + \ldots + A_n))
		= \prob(BA_1 + \ldots + BA_n) \overset{\text{несовместность}}{=} $$ 
	    $$ \overset{\text{несовместность}}{=} \sum\limits_{k = 1}^{n} \prob(BA_k)
		= \sum\limits_{k = 1}^{n} \prob(B\mid A_k)\prob(A_k). $$
	\end{proof}
	
	Формула полной вероятности остаётся справедливой, если отказаться от требования $ A_1 + A_2 + \ldots + A_n = \Omega $
	и заменить его на условие $ B \subset A_1 + \ldots + A_n $ (сохраняя требования попарной несовместности событий $ A_i $ 
	и $ \prob(A_i) > 0 $).
	
	\begin{theorem}[Формула Байеса] \label{Bayes}
		Пусть события $ A_1, \ldots, A_n \in \events $ образуют разбиение $ \Omega $,
		пусть $ B \in \events $ --- ещё одно событие и $ \prob(B) > 0 $.
		Тогда справедлива формула Байеса
		$$ \prob(A_i \mid B) = \tfrac{\prob(B \mid A_i)\prob(A_i)}{\sum\limits_{k = 1}^{n} \prob(B \mid A_k)\prob(A_k)}. $$
	\end{theorem}
	
	\begin{proof}
		По лемме \ref{cond prob} (<<простейшая формула Байеса>>) имеем равенство
		$$ \prob(A_i \mid B) = \tfrac{\prob(B \mid A_i)\prob(B)}{\prob(B)}. $$
		По формуле полной вероятности имеем
		$$ \prob(B) = \sum\limits_{i = k}^{n} \prob(B \mid A_k)\prob(A_k), $$
		откуда следует искомая формула.
	\end{proof}
	
	\subsection{Независимость событий}
	
	Интуиция говорит нам, что события $ A $ и $ B $ <<независимы>>, когда от того с какой вероятностью произойдёт событие $ A $
	не зависит вероятность того, что произойдёт событие $ B $ и наоборот.
	Математически это выражается формулами $ \prob(B \mid A) = \prob(B) $ и $ \prob(A \mid B) = \prob(A) $.
	Чтобы не ограничиваться случаями, когда вероятности событий больше 0, мы определим независимость
	следствием формул выше. События $ A $ и $ B $ называются \defin{независимыми}{independent}, если справедливо равенство
	$ \prob(AB) = \prob(A)\prob(B) $.
	
	\begin{proposition}[Свойства независимости] \label{indep-prop}
		Имеют место следующие утверждения:
		\begin{enumerate}
			\item если $ \prob(B) > 0 $, 
			то независимость $ A $ и $ B $ равносильна равенству $ \prob(A \mid B) = \prob(A) $; \label{indep-prop-1}
			\item если $ A $ и $ B $ независимы, то $ \overline{A} $ и $ B $ независимы; \label{indep-prop-2}
			\item если события $ B_1 $ и $ B_2 $ несовместны, $ A $ и $ B_1 $ независимы, а также $ A $ и $ B_2 $ независимы, 
			то $ A $ и $ B_1 + B_2 $ независимы. \label{indep-prop-3}
		\end{enumerate}
	\end{proposition}
	
	\begin{proof}
		Проверим $ \ref{indep-prop-1} $.
		Если $ \prob(B) > 0 $, то по независимости имеем 
		$$ \prob(A \mid B) = \tfrac{\prob(AB)}{\prob(B)} = \tfrac{\prob(A)\prob(B)}{\prob(B)} = \prob(A). $$
		Обратно, если $ \prob(A \mid B) = \prob(A) $, то $ \tfrac{\prob(AB)}{\prob(B)} = \prob(A) $, откуда
		$ \prob(AB) = \prob(A)\prob(B) $.
		
		Для доказательства $ \ref{indep-prop-2} $ выпишем цепочку равенств
		$$ \prob(\overline{A}B) = \prob((\Omega \setminus A)B) = \prob(B \setminus AB) 
		= \prob(B) - \prob(AB) \overset{\text{независимость}}{=} $$ 
		$$ \overset{\text{независимость}}{=} \prob(B) - \prob(A)\prob(B) = \prob(B)(1 - \prob(A)) = \prob(B)\prob(\overline{A}). $$
		
		Наконец, для доказательства $ \ref{indep-prop-3} $ заметим, что события $ B_1A $ и $ B_2A $ несовместны.
		Тогда
		$$ \prob((B_1 + B_2)A) = \prob(B_1A + B_2A) = \prob(B_1A) + \prob(B_2A) 
		= \prob(B_1)\prob(A) + \prob(B_2)\prob(A)= $$ 
		$$ = (\prob(B_1) + \prob(B_2))\prob(A) = \prob(B_1 + B_2)\prob(A). $$
	\end{proof}

	Теперь определим независимость для набора событий.
	Пусть $ B_1, \ldots, B_n \in \events $ --- события.
	Будем говорить, что они \defin{попарно независимы}{pairwise-independent}, если для всяких двух индексов $ i \neq j $ выполнено равенство
	$ \prob(B_iB_j) = \prob(B_i)\prob(B_j) $ (то есть $ B_i $ и $ B_j $ независимы). 
	Будем называть эти события \defin{независимыми}{global-independent}, если для всякого набора индексов
	$ i_1 < \ldots < i_k $ (здесь $ 2 \leqslant k \leqslant n $) имеет место равенство
	$$ \prob\left(\bigcap\limits_{s = 1}^{k} B_{i_s}\right) = \prod_{s = 1}^{k} \prob(B_{i_s}). $$
	
	\begin{proposition}
		Если события $ B_1, \ldots, B_n $ независимы, то они попарно независимы.
	\end{proposition}
	
	\begin{example}
		Вообще говоря из попарной независимости не следует независимость, что демонстрируется следующим примером.
		Рассмотрим тетраэдр, три грани которого покрашены в красный, зелёный и синий цвета, соответственно,
		а последняя разбита на три треугольника, покрашенных в те же цвета. 
		Пусть вероятности выпадения граней равны $ \tfrac{1}{4} $. покажем, что события <<выпала грань с цветом $ A $>>,
		где $ A $ --- цвет попарно независимы, но не являются таковыми в совокупности. 
		Формально ситуация выглядит следующим образом $ \Omega = \{\omega_{R}, \omega_{G}, \omega_{B}, \omega_{RGB}\} $
		--- элементарное событие --- выпала грань с данной раскраской.
		По условию $ \prob(\omega_{R}) = \prob(\omega_{G}) = \prob(\omega_{B}) = \prob(\omega_{RGB}) = \tfrac{1}{4} $.
		Обозначим через $ R = \{\omega_R, \omega_{RGB}\} $  $ (G = \{\omega_G, \omega_{RGB}\}, B = \{\omega_B, \omega_{RGB}\}) $
		события <<выпала грань с красным (зелёным, синим) цветом>>, соответственно.
		Тогда $ \prob(R) = \prob(G) = \prob(B) = \tfrac{1}{2} $,
		$ \prob(RG) = \prob(GB) = \prob(BR) = \tfrac{1}{4} = \prob(R)\prob(G) = \prob(G)\prob(B) = \prob(B)\prob(R) $,
		но $ \prob(RGB) = \prob(\omega_{RGB}) = \tfrac{1}{4} \neq \prob(R)\prob(G)\prob(B) = \tfrac{1}{8} $.
	\end{example}

	\begin{example}
		Пользуясь примером выше, можно показать, что условие несовместности в пункте $ \ref{indep-prop-3} $ 
		предложения \ref{indep-prop} нельзя опустить.
		Положим $ A = R $ и $ B_1 = G $ и $ B_2 = B $. Тогда $ \prob((B_1 \cup B_2)A) = \prob(\omega_{RGB}) = \tfrac{1}{4} $,
		но $ \prob(A) = \tfrac{1}{2} $, $ \prob(B_1 + B_2) = \tfrac{3}{4} $ и $ \tfrac{1}{2} \cdot \tfrac{3}{4} \neq \tfrac{1}{4} $.
		Таким образом, события $ B_1 + B_2 $ и $ A $ не являются независимыми.
	\end{example}
	
	\begin{example}
		Покажем, что из условия независимости нельзя удалить ни одно из равенств.
		Более того, мы докажем, что для всякого натурального $ n $ 
		и семейства наборов индексов $ S_{J} = \{(i_{1,j}, \ldots, i_{k_j,j})\}_{j \in J} $
		можно построить пример вероятностного пространства $ (\Omega, \events, \prob) $ 
		и событий $ A_1, \ldots, A_n \in \events $ 
		для которых множество наборов, на которых выполнены равенства
		$$ \prob\left(\bigcap\limits_{s = 1}^{k} B_{i_s}\right) = \prod_{s = 1}^{k} \prob(B_{i_s}) $$
		в точности совпадает с $ J $.
		
		Построим пример для дискретного вероятностного пространства.
		Положим $ \Omega = \defineset{(\varepsilon_1, \ldots, \varepsilon_n)}{\varepsilon_i \in \{0, 1\}} $
		--- множество всех кортежей из нулей и единиц длины $ n $,
		$ \prob((\varepsilon_1, \ldots, \varepsilon_n)) = p_{(\varepsilon_1, \ldots, \varepsilon_n)} $
		--- будущее распределение вероятностей.
		Также положим 
		$ A_k = \defineset{(\varepsilon_1, \ldots, \varepsilon_n)}{\varepsilon_i \in \{0, 1\}, \varepsilon_k = 1} $.
		Рассмотрим отображение 
		$ \varphi \colon \mathbb{R}^{2^n} \to \mathbb{R}^{2^n} $,
		заданное в некоторых фиксированных базисах этих пространств по правилу 
		$$ \varphi\colon 
		\left( \begin{matrix}
			\ldots \\
			p_{(\varepsilon_1, \ldots, \varepsilon_n)} \\
			\ldots
		\end{matrix} \right)
		\mapsto 
		\left( \begin{matrix}
			\ldots \\
			\prob(A_{i_1}\ldots A_{i_k}) \\
			\ldots
		\end{matrix} \right), $$
		где для $ k = 0 $ предполагается, что в матрице стоит $ \prob(\Omega) $.
		Поскольку $ \prob(A_{i_1}\ldots A_{i_k}) = 
		\prob(\defineset{(\varepsilon_1, \ldots, \varepsilon_n)}{\varepsilon_i \in \{0, 1\}, 
			\varepsilon_{i_s} = 1, 1 \leqslant s \leqslant k})
		= \sum\limits_{\varepsilon_{i_s} = 1, 1 \leqslant s \leqslant k} p_{(\varepsilon_1, \ldots, \varepsilon_n)} $,
		то $ \varphi $ --- линейное отображение.
		Можно показать, что $ \varphi $ сюръективно (проверьте с помощью элементарных преобразований, 
		что его матрица имеет ранг $ 2^n $) и, следовательно, биективно.
		Таким образом, достаточно подобрать значения вероятностей все возможных произведений $ A_i $ так, чтобы
		вероятности $ p_{(\varepsilon_1, \ldots, \varepsilon_n)} $ были неотрицательны, в сумме давали 1 ($ \prob(\Omega) = 1 $)
		и при этом выполнялись в точности все желаемые равенства на вероятности произведений событий $ A_i $.
		Положим $ \prob(A_i) = \tfrac{1}{2^{2n}} $, $ \prob(\Omega) = 1 $.
		Если $ (i_1, \ldots, i_k) \in S_{J} $, то положим $ \prob(A_{i_1}\ldots A_{i_k}) = \tfrac{1}{2^{2kn}} $.
		Иначе положим $ \prob(A_{i_1}\ldots A_{i_k}) = \tfrac{1}{2^{2kn + 1}} $.
		Проверим, что имеют место неравенство
		$ \tfrac{1}{2^{2kn + 2}} \leqslant p_{(\varepsilon_1, \ldots, \varepsilon_n)} \leqslant \tfrac{1}{2^{2kn}} $,
		для кортежей с $ k > 0 $ числом единиц.
		Для кортежа $ (1, \ldots, 1) $ неравенство выполнено по построению.
		Докажем неравенства для оставшихся кортежей с данным условием индукцией по числу нулей в кортеже.
		Пусть в текущем кортеже $ (\varepsilon_1, \ldots, \varepsilon_n) $ присутствует $ n - k \geqslant 1 $ нулей.
		Прибавим к $ p_{(\varepsilon_1, \ldots, \varepsilon_n)} $ все остальные значения вероятностей элементарных исходов
		--- кортежей, в которых некоторые нули из данного кортежа заменены на единицы.
		Тогда $ p_{(\varepsilon_1, \ldots, \varepsilon_n)} \leqslant \tfrac{1}{2^{2kn}} $,
		так как по предположению индукции все остальные слагаемые положительны, а сумма не превосходит $ \tfrac{1}{2^{2kn}} $.
		С другой стороны, 
		$ p_{(\varepsilon_1, \ldots, \varepsilon_n)} \geqslant \tfrac{1}{2^{2kn + 1}} - \tfrac{2^k - 1}{2^{2(k + 1)n}}
		= \tfrac{1}{2^{2kn + 1}} - \tfrac{1}{2^{2kn + 2 + (2n - k - 2)}} $.
		Так как $ n - k - 1 \leqslant 0 $ и $ n - 1 \leqslant 0 $, 
		то последнее слагаемое по модулю не превосходит $ \tfrac{1}{2^{2kn + 2}} $.
		Тогда $ p_{(\varepsilon_1, \ldots, \varepsilon_n)} $ не меньше $ \tfrac{1}{2kn + 2} $.
		Остаётся убедиться в том, что $ p_{(0, \ldots, 0)} 
		= 1 - \sum\limits_{(\varepsilon_1, \ldots, \varepsilon_n) \neq (0, \ldots, 0)} p_{(\varepsilon_1, \ldots, \varepsilon_n)}
		\geqslant 1 - \tfrac{2^n - 1}{2^{2n}} > 0 $.
		
	\end{example}

	\section{Случайные величины, их распределения, функции распределения и плотности}
	
	\TODO{вписать все определения}
	
	\TODO{ввести функцию распределения}
	
	\TODO{определить распределение как прямой образ вероятностной меры}
	
	\TODO{доказать, что прямой образ вероятностной меры и мера Лебега-Стилтьеса, порождённая функцией распределения совпадают}
	
	\TODO{ввести понятие абсолютно непрерывной случайной величины и её плотности}
	
	\section{Классические примеры распределений} 
	
	\TODO{вписать описания для всех классических распределений}
	
	Дискретные распределения.
	
	\subsection{Распределение константы}
	
	\subsection{Распределение Бернулли}
	
	\subsection{Дискретное равномерное распределение}
	
	\subsection{Биномиальное распределение}
	
	\subsection{Распределение Пуассона}
	
	\subsection{Геометрическое распределение}
	
	\subsection{Гипергеометрическое распределение}
	
	\subsection{Отрицательное биномиальное распределение}
	
	Абсолютно непрерывные случайные величины
	
	\subsection{Равномерное распределение}
	
	\subsection{Экспоненциальное (показательное) распределение}
	
	\subsection{Нормальное распределение (распределение Гаусса)}
	
	\subsection{Распределение Коши}
	
	\section{Численные характеристики случайных величин}
	
	\TODO{записать определения и свойства, описать ковариацию как скалярное произведение}
	
	\TODO{доказать формулы для вычисленя матожидания через интегралы Лебега, Лебега-Стилтьеса и интеграл Римана для абсолютно непрерывной случайно величины}
	
	
	
	\section{Сходимости случайных величин}
	
	\TODO{записать определения всех сходимостей и вывод одних сходимостей из других}
	
	\section{Производящие функции}
	
	\TODO{записать определение}
	
	\section{Характеристические функции}
	
	\begin{theorem}[Бохнер, Хинчин]
		
	\end{theorem}
	
	\section{Предельные теоремы}
	
	\TODO{дописать ниже доказательства теорем}
	
	\subsection{Неравенства}
	
	\subsection{Закон больших чисел}
	
	\begin{theorem}[Закон больших чисел в форме Бернулли]
		
	\end{theorem}
	
	\begin{theorem}[Закон больших чисел в форме Чебышёва]
		
	\end{theorem}
	
	\begin{theorem}[Усиленный закон больших чисел]
		
	\end{theorem}
	
	\begin{theorem}[Закон больших чисел в форме Хинчина]
		content
	\end{theorem}
	
	\subsection{Теорема Муавра-Лапласа}
	
	\begin{theorem}[Теорема Пуассона]
		
	\end{theorem}
	
	\begin{theorem}[Формула Стирлинга]
		
	\end{theorem}
	
	\begin{theorem}[Муавр, Лапласа]
		
	\end{theorem}
	
	\subsection{Закон нуля или единицы}
	
	\begin{lemma}[Борель, Кантелли]
		
	\end{lemma}
	
	\begin{lemma}[Борель, Кантелли]
		
	\end{lemma}
	
	\begin{theorem}[Закон нуля или единицы Колмогорова]
		
	\end{theorem}
	
	\subsection{Закон повторного логарифма}
	
	\begin{theorem}[Закон повторного логарифма]
		
	\end{theorem}
	
	\subsection{Закон арксинуса}
	
	\begin{theorem}[Закон арксинуса]
		
	\end{theorem}
	
	\subsection{Правило трёх сигм}
	
	\begin{theorem}[Правило трёх сигм]
		
	\end{theorem}
	
	\subsection{Центральная предельная теорема}
	
	\begin{theorem}[Центральная предельная теорема]
		
	\end{theorem}
	
	\begin{theorem}[Оценка Берри-Эссена]
		
	\end{theorem}
	
	\section{Совместные распределения случайных величин}
	
	d
	
	\section{Свёртки случайных величин}
	
	d
	
	\section{Указатель терминов}
	
	d
	
	\section{Указатель теорем}
	
	d
	
	\begin{thebibliography}{2}
		
		\bibitem{KolmogorovFomin} Колмогоров А.Н., Фомин С.В. {\it Элементы теории функций и функционального анализа}, Физматлит, 2004, 572с.
		
		\bibitem{DiyachenoUliyanov} Дьяченко~М.~И., Ульянов~П.~Л. {\it Мера и интеграл}, Факториал, 1998, 160с.
	
		\bibitem{Borovkov} Боровков А. А. {\it Теория вероятностей}, Физматлит, 1986, 432с.
	
	\end{thebibliography}
	
	
\end{document}

