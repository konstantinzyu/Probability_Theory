\documentclass[12pt]{article}
%\documentclass[12pt]{amsart}

\pagestyle{plain}
\usepackage[margin=3cm]{geometry} 

\usepackage{amsmath,amssymb,amsfonts,enumerate,latexsym, amsthm,textcomp,wasysym,longtable,dsfont}
\usepackage{cmap}

\usepackage{pgfplots}
\pgfplotsset{width=10cm,compat=1.9}

% \usepackage{indentfirst}
\usepackage[matrix, arrow, curve]{xy} % Для коммутативных диаграмм

\usepackage[utf8]{inputenc}
\usepackage[russian]{babel}
\usepackage{verbatim}
\makeatletter
\def\@settitle{\begin{center}%
		\baselineskip14\p@\relax
		\bfseries
		\large \@title
	\end{center}%
}
\makeatother

\usepackage{cancel}
\usepackage{graphicx}
% \graphicspath{{pictures/}}
% \DeclareGraphicsExtensions{.pdf,.png,.jpg}
%\usepackage{russian}

%%%%%%%%%%%%%%%%%%%%%%%%%%%%%%%%%%%%%%%%%%%%%%%%%%%%%%%%%%%%
% % commands for making comments
\usepackage[dvipsnames]{xcolor}
\newcommand{\YP}[1]{\footnote{\textcolor{red}{YP: #1}}}
\newcommand{\yp}[1]{\leavevmode{\color{red}{#1}}}
% {\textcolor{orange}{#1}} 
\usepackage[normalem]{ulem}
%%%%%%%%%%%%%%%%%%%%%%%%%%%%%%%%%%%%%%%%%%%%%%%%%%%%%%%%%%%%

\usepackage{hyperref}
\usepackage{tikz,tikz-cd} % Ещё для коммутативных диаграмм


% \textheight=270mm
% \textwidth=190mm
% \voffset=-40mm
% \hoffset=-35mm
% \pagestyle{empty}
% 
% \\SLoppy



\emergencystretch=5pt

\newtheorem{theorem}{Теорема}
\newtheorem{proposition}[theorem]{Предложение}
\newtheorem*{definition}{Определение}
\newtheorem{lemma}[theorem]{Лемма}
\newtheorem{corollary}[theorem]{Следствие}


\numberwithin{theorem}{section}

\theoremstyle{definition}

\newenvironment{example}{\indent \textbf{Пример.}}{\indent}

%\newcommand{\defin}[2]{\hypertarget{#2}{{\color{red} #1}} \label{def:#2}}
\newcommand{\defin}[2]{\hypertarget{#2}{{\color{red} #1}}}


\newtheorem*{remark*}{Замечание}

\newcommand{\Alt}{\mathfrak{A}}
\newcommand{\Sym}{\mathfrak{S}}
\newcommand{\Q}{\mathrm{Q}}
\newcommand{\D}{\mathrm{D}}
\newcommand{\Dic}{\mathrm{Dic}}
\newcommand{\rC}{\mathrm{C}}
\newcommand{\T}{\mathrm{T}}
\newcommand{\rO}{\mathrm{O}}
\newcommand{\I}{\mathrm{I}}
\newcommand{\CC}{\mathbb{C}}
\newcommand{\RR}{\mathbb{R}}
\newcommand{\FF}{\mathbb{F}}
\newcommand{\EE}{\mathbb{E}}
\newcommand{\KK}{\mathbb{K}}
\newcommand{\LL}{\mathbb{L}}

\newcommand{\calA}{\mathcal{A}}
\newcommand{\calB}{\mathcal{B}}
\newcommand{\calM}{\mathcal{M}}
\newcommand{\calN}{\mathcal{N}}
\newcommand{\calE}{\mathcal{E}}
\newcommand{\calP}{\mathcal{P}}

\newcommand{\Gal}{\operatorname{Gal}}
\newcommand{\Aut}{\operatorname{Aut}}
\newcommand{\Tor}{\operatorname{T}}
\newcommand{\AGL}{\operatorname{AGL}}
\newcommand{\GL}{\operatorname{GL}}
\newcommand{\Qrn}{\operatorname{Q}_8}
\newcommand{\SL}{\operatorname{SL}}
\newcommand{\PGL}{\operatorname{PGL}}
\newcommand{\PSL}{\operatorname{PSL}}
\newcommand{\PSU}{\operatorname{PSU}}
\newcommand{\SU}{\operatorname{SU}}
\newcommand{\SO}{\operatorname{SO}}
\newcommand{\diag}{\operatorname{diag}}
\newcommand{\projective}{\mathbb{P}}
\newcommand{\affine}{{\mathbb{A}}}
\newcommand{\characteristic}{\operatorname{char}}
\newcommand{\rank}{\operatorname{rank}}
\newcommand{\rd}{\operatorname{rd}}
\newcommand{\ed}{\operatorname{ed}}
\newcommand{\id}{\operatorname{id}}
\newcommand{\pr}{\operatorname{pr}}
\newcommand{\ab}[1]{#1^{\mathrm{ab}}}

\newcommand{\prob}{\operatorname{P}}
\newcommand{\events}{\mathfrak{F}}
\newcommand{\expect}{\operatorname{E}}
\newcommand{\disp}{\operatorname{D}}
\newcommand{\cov}{\operatorname{Cov}}

\newcommand{\Bor}{\operatorname{Bor}}
\newcommand{\Top}{\mathfrak{Top}}
\newcommand{\Meas}{\mathfrak{Meas}}

\newcommand{\Hom}{\operatorname{Hom}}
\newcommand{\Image}{\operatorname{Im}}
\newcommand{\ind}{\mathds{1}}
\newcommand{\diff}{\mathrm{d}}



% Цвета
\definecolor{linkcolor}{HTML}{0000FF} % цвет ссылок
\definecolor{urlcolor}{HTML}{0000FF} % цвет гиперссылок
\definecolor{citecolor}{HTML}{0000FF} % цвет ссылки на статью
\hypersetup{pdfstartview=FitH, linkcolor=linkcolor, urlcolor=urlcolor, citecolor=citecolor, colorlinks=true}

% Пробелы, отступы и выделения
\definecolor{todocolor}{HTML}{FF4500} % цвет todo
\newcommand{\TODO}[1]{\textcolor{todocolor}{НУЖНО: #1}}
\renewcommand\labelenumi{\rm (\arabic{enumi})}
\renewcommand\theenumi{\rm (\arabic{enumi})}
% Определение множества
\newcommand{\defineset}[2]{\left\{
	\left.
	#1
	\right\vert
	#2
	\right\}}

% Кусочное определение функции
\newcommand{\definefuntwo}[4]{
	\begin{cases}
		#1, & #2; \\
		#3, & #4.
	\end{cases}
}

\newcommand{\definefunfour}[8]{
	\begin{cases}
		#1, & #2; \\
		#3, & #4; \\
		#5, & #6; \\
		#7, & #8.
	\end{cases}
}

\newcommand{\spmatrix}[4]{
	\left( \begin{smallmatrix}
		#1 & #2 \\
		#3 & #4
	\end{smallmatrix} \right)
}

\pgfmathdeclarefunction{erf}{1}{%
	\pgfmathparse{%
		sign(#1) * (1 - exp(-#1*#1*(4/pi + 0.147*#1*#1)) / (1 + 0.147*#1*#1)^1.5)
	}%
}


\setcounter{section}{-1}

\title{Теория вероятностей}
\author{(Ещё не)алгебраист}


\begin{document}
	\maketitle
	
	\section*{Предисловие}
	
	Эти записки созданы с целью аккуратно формализовать и заполнить пробелы в лекциях Елены Борисовны Яровой.
	В разделе \ref{preparing} будут содержаться основные принятые в курсе обозначения, 
	а также сведения и определения из разных разделов математики, которыми автор будет пользоваться.
	Поскольку автор считает полезным взгляд на всякий раздел математики с точки зрения теории категорий и её приложений, 
	этот язык также будет упоминаться (тем не менее, не замещая собой прочие подходы).
	
	\tableofcontents
	
	
	\section{Предварительные сведения} \label{preparing}
	
	\subsection{Обозначения}
	
	%В работе приняты следующие обозначения:
	
	\begin{itemize}
		\item $ \Omega $ --- пространство элементарных исходов;
		\item $ \omega $ --- элементарный исход;
		\item $ \events $ --- $ \sigma $-алгебра событий;
		\item $ \prob $ --- вероятностная мера;
		\item $ \xi, \eta, \zeta $ --- случайные величины;
		\item $ \expect \xi $ --- математическое ожидание случайной величины $ \xi $;
		\item $ \disp \xi $ --- дисперсия случайной величины $ \xi $;
		\item $ \cov(\xi, \eta) $ --- ковариация случайных величин $ \xi $ и $ \eta $;
		\item $ \rho(\xi, \eta) $ --- корреляция случайных величин $ \xi $ и $ \eta $;
	\end{itemize}
	
	\subsection{Предварительные сведения из действительного анализа}
	
	\subsubsection{Системы множеств и структуры на них}
	
	Система множеств (следует понимать как синоним термина <<семейство множеств>>) $ S $ 
	называется \defin{полуцольком}{semiring}, если она удовлетворяет следующим аксиомам:
	\begin{enumerate}
		\item $ \varnothing \in S $;
		\item $ \forall A, B \in S: A \cap B \in S $;
		\item $ \forall A, B \in S, A \subset B \ \exists n \in \mathbb{N} \ \exists C_1, \ldots, C_n \in S: 
		A = B \sqcup \bigsqcup\limits_{k = 1}^{n} C_k  $.
	\end{enumerate}
	
	Множество $ \Omega \in U $ называется \defin{единицей системы множеств $ U $}{unit},
	если всякий элемент $ A \in U $ является подмножеством $ \Omega $.
	
	Система множеств $ R $ называется \defin{кольцом}{ring}, если она удовлетворяет следующим аксиомам:
	\begin{enumerate}
		\item $ \forall \ A, B \in R: A \cap B \in R $;
		\item $ \forall \ A, B \in R: A \vartriangle B \in R $.
	\end{enumerate}
	
	Следующее утверждение проверяется непосредственно, исходя из теоретико-множественных тождеств,
	но его доказательство приведено, например, в книге \cite{DiyachenoUliyanov}.
	
	\begin{proposition}
		Пусть $ R $ --- кольцо. Тогда $ R $ является полукольцом.
		Кроме того, для любых элементов $ A, B \in R $ в $ R $ также содержатся их объединение $ A \cup B $
		и разность $ A \setminus B $. 
	\end{proposition}
	
	Кольцо называется \defin{$ \sigma $-кольцом}{sigma-ring}, если 
	для любого счётного набора его элементов $ \{A_k\}_{k \in R} \subset R $ их объединение содержится в $ R $
	($ \bigcup\limits_{k \in \mathbb{N}} A_k \in R $)
	и \defin{$ \delta $-кольцом}{delta-ring},
	если для любого счётного набора его элементов $ \{A_k\}_{k \in R} \subset R $ их пересечение содержится в $ R $.
	
	Кольцо с единицей $ \Omega $ называется \defin{алгеброй (подмножеств множества $ \Omega $)}{unit}.
	
	В книгах по теории вероятностей понятие алгебры часто вводится с использование другого равносильного набора аксиом,
	что выражает следующее
	
	\begin{proposition}[Определение алгебры в традиции теории вероятностей]
		Система множеств $ R $ является алгеброй подмножеств множества $ \Omega $
		тогда и только тогда, когда $ R $ удовлетворяет следующим аксиомам
		\begin{enumerate}
			\item $ \Omega \in R $;
			\item $ \forall \ A, B \in R: A \cup B, A \cap B \in R $;
			\item $ \forall \ A \in R: \ \Omega \setminus A := \overline{A} \in R $.
		\end{enumerate}
	\end{proposition}
	Мы снова опускаем доказательство, сводящееся к тождествам теории множеств.
	
	Если алгебра является $ \sigma $-кольцом или $ \delta $-кольцом, 
	то её называют \defin{$ \sigma $-алгеброй}{sigma-algebra} или \defin{$ \delta $-алгеброй}{delta-algebra}, соответственно.
	
	\begin{proposition}
		Имеет место следующее:
		\begin{enumerate}
			\item Всякое $ \sigma $-кольцо является $ \delta $-кольцом,
			обратное вообще говоря не верно.
			\item Всякая $ \sigma $-алгебра является $ \delta $-алгеброй
			и наоборот.
		\end{enumerate}
	\end{proposition}
	 
	\begin{lemma} \label{ring restriction}
		Пусть $ R $ --- $ ( $$ \sigma $-$ ) $кольцо и $ A \in R $.
		Тогда множество $$ R \cap A := \defineset{B \cap A}{B \in R} $$ является $ ( $$ \sigma $-$ ) $алгеброй
		подмножеств $ A $. Также $ R \cap A \subset R $.
	\end{lemma}
	
	\begin{proof}
		По построению $ \Omega \cap A = A $ содержится в $ R \cap A $
		и всякий элемент $ R \cap A $ есть подмножество $ A $.
		Так как кольцо замкнуто относительно пересечений, то $ R \cap A \subset R $.
		
		Пусть теперь $ C_1 = B_1 \cap A, C_2 = B_2 \cap A \in R \cap A $ --- два множества.
		Тогда $ C_1 \cap C_2 = (B_1 \cap B_2)  \cap A \in R \cap A $, так как $ B_1 \cap B_2 \in R $.
		Далее, $ C_1 \cup C_2 = (B_1 \cup B_2) \cap A \in R \cap A $, так как $ B_1 \cup B_2 \in R $.
		Окончательно, $ A \setminus C_1 = (A \setminus B_1) = (\Omega \setminus B_1) \cap A \in R \cap A $,
		поскольку $ \Omega \setminus B_1 \in R $.
		
		Предположим, что $ R $ являлось $ \sigma $-алгеброй. Пусть $ \{C_k\} $ --- счётное семейство 
		элементов $ R \cap A $ и $ C_k = B_k \cap A $.
		Тогда
		$$ \bigcup\limits_{i = 1}^{\infty} C_k = \bigcup\limits_{i = 1}^{\infty} (B_k \cap A) 
		= \left(\bigcup\limits_{i = 1}^{\infty} B_k \right) \cap A \in R \cap A, $$
		принадлежность справедлива в силу того, что $ \bigcup\limits_{i = 1}^{\infty} B_k \in R $.
	\end{proof}
	
	Можно показать, 
	что кольцо множеств является кольцом в алгебраическом смысле этого слова с операциями сложения $ \vartriangle $
	и умножения $ \cap $, а алгебра множеств является булевой алгеброй (в частности, $ \FF_2 $-алгеброй).
	
	
	\subsubsection{Минимальное кольцо и минимальная алгебра}
	
	Следующее утверждение сводится к проверке аксиом кольца или алгебры,
	но его доказательство также можно прочитать в книге $ \cite{DiyachenoUliyanov} $.
	
	\begin{proposition} \label{intersection of rings}
		Пусть $ \{R_{\alpha}\}_{\alpha \in \mathcal{A}} $ --- семейство $ ( $$ \sigma $-,$ \delta $-$ ) $колец множеств.
		Тогда  система $ R = \bigcap\limits_{\alpha \in \mathcal{A}} R_{\alpha} $ является $ ( $$ \sigma $-,$ \delta $-$ ) $кольцом.
		Кроме того, если все кольца $ R_{\alpha} $ являются $ ( $$ \sigma $-$ ) $алгебрами подмножеств множества $ \Omega $
		$ ( $то есть у них есть общая единица$ ) $, то $ R $ также является $ ( $$ \sigma $-$ ) $алгеброй подмножеств множества $ \Omega $.
	\end{proposition}
	
	\begin{theorem}[О минимальной ($ \sigma $-)алгебре] \label{min ring exists}
		Пусть $ U $ --- система множеств. Тогда существует как минимум одно $ ( $$ \sigma $-$ ) $кольцо, содержащее $ U $.
		Пересечение всех таких $ ( $$ \sigma $-$ ) $колец $ R(U) $ $ (R_{\sigma}(U)) $ само является $ ( $$ \sigma $-$ ) $кольцом. 
		Всякое $ ( $$ \sigma $-$ ) $кольцо, содержащее $ U $, содержит и $ R(U) $. 
		Если $ \Omega \in U $ --- единица $ U $, 
		то $ R(U) $ $ (R_{\sigma}(U)) $ является $ ( $$ \sigma $-$ ) $алгеброй подмножеств множества $ \Omega $.
	\end{theorem}
	
	\begin{proof}
		В качестве ($ \sigma $-)кольца, содержащего $ U $ можно взять булеан $ 2^{\Sigma} $, 
		где множество $ \Sigma $ определено как объединение $ \bigcup\limits_{A \in U} A $.
		
		Пересечение всех таких ($ \sigma $-)колец существует, поскольку имеется хотя бы одно кольцо и
		по предложению \ref{intersection of rings} это пересечение само является ($ \sigma $-)кольцом.
		
		Пусть ($ \sigma $-)кольцо $ R' $ содержит множество $ U $. 
		Тогда по построению $ R(U) $ ($ R_{\sigma}(U) $) содержится в пересечении $ 2^{\Sigma} \cap R' $,
		откуда $ R(U) \subset R' $ ($ R_{\sigma}(U) \subset R' $).
		
		Если $ \Omega $ --- единица $ U $, то по построению $ \Sigma = \Omega $.
		Для всякого ($ \sigma $-)кольца $ R' $, содержащего $ U $ имеем $ \Omega \in R $ и по лемме \ref{ring restriction}
		система множеств $ R' \cap \Omega \subset R' $ является ($ \sigma $-)алгеброй подмножеств $ \Omega $.
		Так как $ \Omega $ являлось единицей $ U $, то $ U $ содержится в $ R' \cap \Omega $.
		Следовательно, достаточно рассматривать пересечение только ($ \sigma $-)алгебр подмножеств множества $ \Omega $,
		содержащих $ U $. По предложению \ref{intersection of rings} их пересечение является ($ \sigma $-)алгеброй подмножеств $ \Omega $.
	\end{proof}
	
	Опираясь на предложение $ \ref{min ring exists} $ дадим определение. 
	Для системы множеств $ U $ пересечение всех колец, содержащих $ U $ называется \defin{минимальным кольцом,
	порождённым $ U $}{min-ring} и обозначается $ R(U) $.
	
	\begin{proposition} \label{min ring description}
		Пусть $ S $ --- полукольцо и $ R(S) $ --- минимальное кольцо, порождённое $ S $.
		Тогда $ R(S) $ допускает следующие описания
		\begin{itemize}
			\item $ R(S) = \defineset{A_1 \cup \ldots \cup A_n}{n \in \mathbb{N}, \{A_i\}_{i = 1}^{n} \subset S}; $
			\item $ R(S) = \defineset{A_1 \sqcup \ldots \sqcup A_n}{n \in \mathbb{N}, \{A_i\}_{i = 1}^{n} \subset S}. $
		\end{itemize}
	\end{proposition}
	
	\begin{proof}
		\TODO{дописать доказательство или сослаться}
	\end{proof}
	
	\subsubsection{Мера на полукольце и её продолжение на минимальное кольцо}
	
	Пусть $ S $ --- некоторое полукольцо.
	Будем называть неотрицательную функцию $ m \colon S \to \RR $ \defin{мерой}{measure} на полукольце $ S $, 
	если $ m $ удовлетворяет аксиоме аддитивности
	$$ \forall A, B \in S, A \cap B = \varnothing, A \cup B \in S: m(A \sqcup B) = m(A) + m(B). $$
	Если дополнительно для любой последовательности попарно непересекающихся подмножеств $ \{A_k\}_{k \in \mathbb{N}} $,
	объединение которых есть элемент из $ S $ (отметим, что это автоматически выполнено, если $ S $ является $ \sigma $-кольцом)
	имеет место равенство
	$$ m\left(\bigsqcup\limits_{k = 1}^{\infty} A_k \right) = \sum\limits_{k = 1}^{\infty} m(A_k), $$
	то мера $ m $ называется \defin{$ \sigma $-аддитивной}{sigma-measure} (аксиома $ \sigma $-аддитивности).
	Подразумевается, что ряд в правой части сходится для любой последовательности попарно не пересекающихся $ \{A_k\} $, объединение которых лежит в $ S $.
	Можно показать, что из этой аксиомы следует, что $ m(\varnothing) = 0 $ и поэтому из неё следует аксиома аддитивности.
	
	Пользуясь предложением \ref{min ring description} введём функцию $ \nu \colon R(S) \to \RR $
	по правилу $ \nu(A_1 \sqcup \ldots \sqcup A_n) = \sum\limits_{i = 1}^{n} m(A_i) $, где $ A_i \in S $.
	Следующее предложение позволяет назвать $ \nu $ \defin{продолжением меры $ m $ с полукольца $ S $ на его минимальное кольцо}
	{extension-to-min-ring}.
	
	\begin{proposition}
		Справедливо следующее
		\begin{enumerate}
			\item функция $ \nu $ определена корректно, то есть значение $ \nu $ не зависит от выбора представления
			$ A_1 \sqcup \ldots \sqcup A_n; $
			\item функция $ \nu $ является мерой на кольце $ R(S); $
			\item ограничение функции $ \nu $ на полукольцо $ S $ совпадает с $ m; $
			\item если мера $ m $ была $ \sigma $-аддитивной, то функция $ \nu $ также является $ \sigma $-аддитивной.
		\end{enumerate}
	\end{proposition}
	
	\begin{proof}
		См. доказательство в \cite[Глава 1, Теорема 2.2]{DiyachenoUliyanov}.
	\end{proof}
	
	\begin{lemma}
		Пусть $ S $ --- полукольцо и $ m \colon S \to \mathbb{R} $ --- мера на $ S $.
		Тогда $ m $ удовлетворяет следующим свойствам
		\begin{enumerate}
			\item если для $ A, B \in S $ выполнено $ A \subset B $, то $ m(A) \leqslant m(B) $;
			\item если $ A, A_1, \ldots, A_n \in S $ и $ A \subset \bigcup\limits_{i = 1}^{n} A_i $, 
			то $$ m(A) \leqslant \sum\limits_{i = 1}^{n} m(A_i); $$
			\item если $ A_1, \ldots, A_n \subset S $ ---  попарно не пересекающиеся множества
			и $ \bigsqcup\limits_{i = 1}^{n} A_i \subset A \in S $, то 
			$$ \sum\limits_{i = 1}^{n} m(A_i) \leqslant m(A); $$
			\item если $ \{A_i\}_{i = 1}^{+\infty} \subset S $ ---  попарно не пересекающиеся множества
			и $ \bigsqcup\limits_{i = 1}^{+\infty} A_i \subset A \in S $, то 
			$$ \sum\limits_{i = 1}^{+\infty} m(A_i) \leqslant m(A). $$
		\end{enumerate}
	\end{lemma}
	
	\begin{proof}
		\TODO{дописать доказательство или сослаться}
	\end{proof}
	
	Последнее свойство, доказанное в лемме, называют \defin{полуаддитивностью}{semi-additivity}.
	
	\subsubsection{Лебеговское продолжение меры}
	
	Далее будем рассматривать полукольцо $ S $ с единицей $ \Omega $
	и $ \sigma $-аддитивной мерой $ m $.
	Пусть $ \nu \colon R(S) \to \RR $ --- продолжение этой меры на минимальное кольцо.
	
	Введём функцию \defin{внешней меры}{outer-measure} $ \mu^* \colon 2^{\Omega} \to \RR $, заданную по правилу
	$$ \mu^*(A) := \inf\limits_{A \subset \bigsqcup\limits_{i = 1}^{+\infty} B_i, B_i \in S} 
	\sum\limits_{i = 1}^{+\infty} m(B_i). $$
	
	\begin{proposition}
		Для всякого $ A \subset \Omega $ в определении внешней меры можно заменить дизъюнктные объединения на произвольные:
		$$ \mu^*(A) = \inf\limits_{A \subset \bigcup\limits_{i = 1}^{+\infty} B_i, B_i \in S} 
		\sum\limits_{i = 1}^{+\infty} m(B_i). $$
	\end{proposition}
	
	\begin{proof}
		\TODO{дописать доказательство или сослаться}
	\end{proof}
	
	Множество $ A \subset \Omega $ называется \defin{измеримым}{measurable-set}, если для любого $ \varepsilon > 0 $
	найдётся множество $ B \in R(S) $ такое, что $ \mu^*(A \vartriangle B) < \varepsilon $.
	Если $ A $ измеримо, то его \defin{мерой}{measure of set} называется значение $ \mu(A) := \mu^*(A) $.
	Обозначим через $ \calM $ системы всех измеримых подмножеств единицы $ \Omega $.
	
	\begin{lemma}
		Пусть $ \{A_i\}_{i = 1}^{+\infty} \subset \calM $ --- последовательность множеств,
		$ A \in M $
		и $ A \subset \bigcup\limits_{i = 1}^{n} A_i $.
		Тогда $$ \mu^*(A) \leqslant \sum\limits_{i = 1}^{n} \mu^*(A_i); $$
	\end{lemma}
	
	\begin{proof}
		\TODO{дописать доказательство или сослаться}
	\end{proof}
	
	\begin{theorem}
		Система измеримых множеств $ \calM $ является алгеброй.
	\end{theorem}
	
	\begin{proof}
		\TODO{дописать доказательство или сослаться}
	\end{proof}
	
	\begin{theorem}
		Функция $ \mu $ на алгебре множеств $ \calM $ является мерой.
	\end{theorem}
	
	\begin{proof}
		\TODO{дописать доказательство или сослаться}
	\end{proof}
	
	\begin{theorem}
		Алгебра измеримых множеств $ \calM $ является $ \sigma $-алгеброй.
	\end{theorem}
	
	\begin{proof}
		\TODO{дописать доказательство или сослаться}
	\end{proof}
	
	\begin{theorem}
		Мера $ \mu $ на $ \sigma $-алгебре измеримых множеств $ \calM $ является $ \sigma $-аддитивной.
	\end{theorem}
	
	\begin{proof}
		\TODO{дописать доказательство или сослаться}
	\end{proof}
	
	Ограничение внешней меры $ \mu^* $ на $ \sigma $-алгебру измеримых подмножеств $ \calM $ мы будем называть 
	\defin{лебеговским продолжением меры $ m $}{lebesgue-extention}.
	
	%\subsubsection{$ \sigma $-конечная мера}
	
	\subsubsection{Единственность продолжения меры на минимальную $ \sigma $-алгебру}
	
	Прежде, чем приступить к доказательству теорема Каратеодори мы докажем лемму, 
	характеризующую минимальную $ \sigma $-алгебру,
	порождённую данной алгеброй.
	
	\begin{lemma} \label{min sigma algebra monotonic characterization}
		Пусть $ \calA $ --- некоторая алгебра.
		Тогда существует наименьшая по включению система множеств $ Mon(\calA) $,
		удовлетворяющая свойствам
		 \begin{enumerate}
		 	\item $ \calA \subset Mon(\calA) ;$
		 	\item если имеется последовательность множеств $ \{A_i\} \subset Mon(\calA) $
		 	и либо $ A_1 \subset A_2 \subset \ldots $, либо $ A_1 \supset A_2 \supset \ldots $,
		 	то либо $ \bigcup\limits_{i = 1}^{+\infty} A_i \in Mon(\calA) $,
		 	либо $ \bigcap\limits_{i = 1}^{+\infty} A_i \in Mon(\calA) $, соответственно
		 	(мы будем называть это свойство \defin{монотонностью}{monotonic-class}).
		 \end{enumerate}
		 Более того, система множеств $ Mon(\calA) $ 
		 совпадает с минимальной $ \sigma $-алгеброй $ R_{\sigma}(\calA) $,
		 порождённой $ \calA $.
	\end{lemma}
	
	\begin{proof}
		Из определения $ \sigma $-алгебры следует, что $ R_{\sigma}(\calA) $ удовлетворяет свойствам из условия.
		Кроме того, пересечение произвольного семейства систем множеств, удовлетворяющих данным условиям
		снова будет удовлетворять им. Поэтому взяв пересечение всех таких систем множеств,
		мы получим наименьшую по включению систему $ Mon(\calA) $.
		Так как всегда можно ограничиться системами множеств с той же единицей, что была в $ \calA $,
		то $ Mon(\calA) $ обладает единицей.
		
		Из сказанного выше следует, что $ Mon(\calA) $ содержится в $ \sigma $-алгебре $ R_{\sigma}(\calA) $.
		Мы докажем, что система $ Mon(\calA) $ сама является $ \sigma $-алгеброй.
		Отсюда по минимальности (теорему \ref{min ring exists}) будет следовать обратное включение.
		
		Для множества $ B \in Mon(\calA) $ обозначим через 
		$$ L(B) 
		= \defineset
		{A \in Mon(\calA)}
		{A \setminus B, B \setminus A, A \cup B \in Mon(\calA)} $$ 
		систему множеств из $ Mon(\calA) $, 
		разности и объединение которых с $ B $ снова лежат в $ Mon(\calA) $.
		По построению $ A \in L(B) $ равносильно тому, что $ B \in L(A) $.
		
		Рассмотрим случай, когда система множеств $ L(B) $ непуста и в ней 
		содержится вложенную последовательность $ \{A_i\} \subset L(B) $
		то есть либо $ A_1 \subset A_2 \subset \ldots $, либо $ A_1 \supset A_2 \supset \ldots $.
		Тогда из того, что $ Mon(\calA) $ удовлетворяет свойству монотонности,
		то $ \bigcup\limits_{i = 1}^{+\infty} A_i \in L(B) $,
		либо $ \bigcap\limits_{i = 1}^{+\infty} A_i \in L(B) $, соответственно.
		Таким образом, система множеств $ L(B) $ тоже удовлетворяет свойству монотонности.
		
		Пусть теперь $ A \in \calA $. Тогда $ \calA \subset L(A) $, то есть $ L(A) $ удовлетворяет обоим свойствам из условия. 
		Отсюда по минимальности $ Mon(\calA) \subset L(A) $.
		Тогда для любого $ B \in Mon(\calA) $ имеем $ \calA \subset L(B) $.
		Поэтому, аналогично, $ Mon(\calA) \subset L(B) $,
		то есть $ Mon(\calA) = L(B) $ для любого $ B \in Mon(\calA) $.
		Таким образом, система множеств $ Mon(\calA) $ является алгеброй.
		
		Из монотонности следует, что $ Mon(\calA) $ является $ \sigma $-алгеброй.
	\end{proof}
	
	Мы приведём доказательство теоремы Каратеодори для случая конечной меры, он отметим, 
	что она остаётся верной и для $ \sigma $-конечной меры.
	
	\begin{theorem}[Каратеодори] \label{uniqueness of extention to the minimal sigma algebra}
		Пусть $ \calA $ --- $ \sigma $-алгебра подмножеств множества $ \Omega $
		и $ \nu \colon \calA \to \RR $ --- конечная мера на $ \calA $.
		Тогда существует единственная $ \sigma $-аддитивная мера $ \mu $ 
		на минимальной $ \sigma $-алгебре $ R_{\sigma}(\calA) $,
		ограничение которой на $ \calA $ совпадает с $ \nu $.
	\end{theorem}
	
	\begin{proof}
		Для доказательства существования рассмотрим лебеговское продолжение меры $ \nu $ на $ \sigma $-алгебру измеримых подмножеств $ \calM $.
		По минимальности (теореме \ref{min ring exists}) имеем включение $ R_{\sigma}(\calA) \subset \calM $.
		Тогда ограничение лебеговского продолжения на $ R_{\sigma}(\calA) $ будет являться $ \sigma $-аддитивной мерой.
		
		Докажем единственность. Предположим, что $ \mu_1 $ и $ \mu_2 $ --- два продолжения меры $ \nu $.
		Положим $ \calA_{=} = \defineset{B \in R_{\sigma}(\calA)}{\mu_1(B) = \mu_2(B)} \subset R_{\sigma}(\calA) $ 
		--- система множеств, на которых меры $ \mu_1 $ и $ \mu_2 $ совпадают.
	
		Промерим, что $ \calA_{=} $ удовлетворяет условиям леммы \ref{min sigma algebra monotonic characterization}.
		По предположению о совпадении ограничений $ \mu_1 $ и $ \mu_2 $ на алгебре $ \calA $ с мерой $ m $ 
		имеем включение $ \calA \subset \calA_{=} $.
		Остаётся проверить только второе условие.
		
		Пусть имеется вложенная последовательность $ \{A_i\} \subset \calA_{=} $,
		то есть либо $ A_1 \subset A_2 \subset \ldots $, либо $ A_1 \supset A_2 \supset \ldots $.
		По непрерывности мер в обоих случаях имеем равенства
		$$ \mu_1\left(\bigcup\limits_{i = 1}^{+\infty} A_i\right)
		= \lim\limits_{i \to +\infty} \mu_1(A_i) 
		= \lim\limits_{i \to +\infty} \mu_2(A_i)
		= \mu_2\left(\bigcup\limits_{i = 1}^{+\infty} A_i\right), $$
		$$ \mu_1\left(\bigcap\limits_{i = 1}^{+\infty} A_i\right)
		= \lim\limits_{i \to +\infty} \mu_1(A_i) 
		= \lim\limits_{i \to +\infty} \mu_2(A_i)
		= \mu_2\left(\bigcap\limits_{i = 1}^{+\infty} A_i\right). $$
		Таким образом, $ \calA_{=} $ удовлетворяет условию монотонности и по лемме \ref{min sigma algebra monotonic characterization} содержит (поскольку мы не проверили минимальность лемма даёт только такой результате) 
		$ \sigma $-алгебру $ R_{\sigma}(\calA) $.
		
		Из всего сказанного следует равенство $ \calA_{=} = R_{\sigma}(\calA) $.
	\end{proof}
	
	\subsubsection{Непрерывность и полнота меры}
	
	Мера $ \mu $ на кольце $ R $ называется \defin{непрерывной}{continuous-measure},
	если для любой последовательности вложенных подмножеств $ \{A_i\}_{i = 1}^{+\infty} \subset R $,
	$ A_1 \supset A_2 \supset A_3 \supset \ldots $, пересечения которых $ A = \bigcap\limits_{i = 1}^{\infty} A_i $
	лежит в $ R $,
	имеет место равенство
	$$ \lim\limits_{i \to +\infty} \mu(A_i) = \mu(A) = \mu\left(\bigcap\limits_{i = 1}^{\infty} A_i\right). $$
	
	\begin{proposition}
		Конечная мера $ \mu $ на кольце $ R $ непрерывна тогда и только тогда, когда она $ \sigma $-аддитивна.
	\end{proposition}
	
	\begin{proof}
		\TODO{дописать доказательство}
	\end{proof}
	
	\begin{corollary}
		Пусть задана мера $ \mu $ на кольце $ R $
		и последовательность вложенных подмножеств $ \{A_i\}_{i = 1}^{+\infty} \subset R $,
		$ A_1 \supset A_2 \supset A_3 \supset \ldots $, пересечение которых $ A = \bigcap\limits_{i = 1}^{\infty} A_i $
		лежит в $ R $ и, кроме того, мера $ \mu(A_1) < +\infty $. Тогда
		имеет место равенство
		$$ \lim\limits_{i \to +\infty} \mu(A_i) = \mu(A) = \mu\left(\bigcap\limits_{i = 1}^{\infty} A_i\right). $$
	\end{corollary}
	
	\begin{proof}
		\TODO{дописать доказательство}
	\end{proof}
	
	\begin{proposition}
		Пусть задана конечная $ \sigma $-аддитивная мера $ \mu $ на кольце $ R $.
		Тогда для любой последовательности вложенных подмножеств $ \{A_i\}_{i = 1}^{+\infty} \subset R $,
		$ A_1 \subset A_2 \subset A_3 \subset \ldots $, объединение которых $ A = \bigcup\limits_{i = 1}^{\infty} A_i $
		лежит в $ R $,
		имеет место равенство
		$$ \lim\limits_{i \to +\infty} \mu(A_i) = \mu(A) = \mu\left(\bigcup\limits_{i = 1}^{\infty} A_i\right). $$
	\end{proposition}
	
	\begin{proof}
		\TODO{дописать доказательство}
	\end{proof}
	
	Мера $ \mu $ на кольце $ R $ называется \defin{полной}{complete measure}, если для любого множества $ A \in R $ 
	из равенства нулю меры $ \mu(A) = 0 $ вытекает, что всякое подмножество $ B \subset A $ лежит в $ R $.
	Из свойств меры в этом случае $ \mu(B) = 0 $.
	
	\subsubsection{Мера Лебега-Стилтьеса}
	
	Пусть $ f \colon \RR \to \RR $ --- неубывающая ограниченная функция. Можно рассматривать и случай, когда функция не ограничена и для неё развить теорию, используя $ \sigma $-конечную меры, но мы не будем сталкиваться с такими случаями в дальнейшем и поэтому остановимся на рассмотрении ограниченной функции $ f $.
	Рассмотрим полукольцо $ S $ полуинтервалов вида $ (a, b] $ или $ (a; +\infty), (-\infty; b] $
	и введём на нём функцию $ m_f $ по правилу $ m_f((a, b]) = f(b) - f(a) $,
	$ m_f((a; +\infty)) = \lim\limits_{x \to +\infty} f(x) - f(a) $,
	$ m_f((-\infty; b]) = f(b) - \lim\limits_{x \to -\infty} f(x) $
	и $ \mu_f(\RR) = \lim\limits_{x \to +\infty} f(x) - \lim\limits_{x \to -\infty} f(x) $.
	
	\begin{proposition}
		Функция $ m_f $ является мерой на полукольце $ S $.
		Функция $ f $ непрерывна справа тогда и только тогда, когда мера $ m_f $ является $ \sigma $-аддитивной.
	\end{proposition}
	
	\begin{proof}
		Докажем, что $ m_f $ удовлетворяет аксиоме аддитивности.
		Пусть $ (a, b] = \bigsqcup\limits_{i = 1}^{n} (a_i, b_i] $.
		Можно считать, что $ a = a_1 < b_1 = a_2 < b_2 < \ldots < b_n = b $.
		Тогда 
		$$ m_F((a,b]) = f(b) - f(a) = \sum\limits_{i = 1}^{n} (f(b_i) - f(a_i)) 
		= \sum\limits_{i = 1}^{n} m_F((a_i, b_i]). $$
		Доказательство для бесконечных интервалов аналогично, нужно лишь заменить числа $ a $ или $ b $
		на соответствующие пределы.
	
		Предположим, что $ f $ непрерывна справа.
		Пусть $ (a, b] = \bigsqcup\limits_{i = 1}^{+\infty} (a_i, b_i] $.
		По \hyperlink{semiadditivity}{полуаддитивности}
		$$ \sum\limits_{i = 1}^{+\infty} m_f((a_i, b_i]) \leqslant m_f((a, b]). $$
		Фиксируем $ \varepsilon > 0 $.
		По непрерывности справа найдём такое $ a' \in (a, b] $,
		что $ F(a') - F(a) < \tfrac{\varepsilon}{2} $
		и такие $ b_i' > b_i $, чтобы $ F(b_i') - F(b_i) < \tfrac{\varepsilon}{2^{i + 1}} $.
		Тогда имеем
		$$ [a',b] \subset (a,b] = \bigsqcup\limits_{i = 1}^{+\infty} (a_i, b_i] \subset \bigsqcup\limits_{i = 1}^{+\infty} (a_i, b_i'). $$
		Поскольку отрезок $ [a', b] $ компактен, то из его покрытия $ \{(a_i, b_i')\} $
		можно выбрать конечное подпокрытие $ \{(a_{i_k}, b_{i_k}')\}_{k = 1}^{n} $.
		Тогда $$ m_f((a,b]) \leqslant m_f((a',b]) + \tfrac{\varepsilon}{2} 
		\leqslant \sum\limits_{k = 1}^{n} m_f((a_{i_k}, b_{i_k}']) + \tfrac{\varepsilon}{2}
		\leqslant \sum\limits_{i = 1}^{+\infty} m_f((a_{i}, b_{i}']) + \varepsilon. $$
		Отсюда следует равенство $ \sum\limits_{i = 1}^{+\infty} m_f((a_i, b_i]) = m_f((a, b]) $.
		
		Случай с бесконечными концами сводится к случаю конечного полуинтервала
		путём выбора точки $ c $ такой, что $ F(c) - F(-\infty) < \varepsilon $
		или $ F(+\infty) - F(c) < \varepsilon $ и рассмотрения пересечения полуинтервалов разбиения с $ (-\infty; c] $
		или $ (c; +\infty) $.
	\end{proof}
	
	\defin{Мерой Лебега-Стилтьеса (порождённой функцией $ f $)}{Lebesgue-Stieltjes-measure} 
	мы назовём лебеговское продолжение меры $ m_f $ и обозначим её через $ \mu_f $.
	
	\subsubsection{Измеримые функции}
	
	Пусть $ \events $ --- $ \sigma $-алгебра подмножеств $ \Omega $.
	Назовём отображение $ f \colon \Omega \to \RR $ \defin{измеримой функцией}{measurable-function},
	если для любого \hyperlink{borel-sets}{борелевского множества} $ B \subset \RR $
	его прообраз является элементом $ \events $ (то есть $ f $ является \hyperlink{morphism-of-measurable-spaces}{измеримым отображением} 
	$ f \colon (\Omega, \events) \to (\RR, \calB) $).
	Можно показать (см. лемму \ref{measurability of map}), что данное требование равносильно тому, что прообраз любого интервала
	или тому, что прообраз любого бесконечного полуинтервала $ (-\infty, b] $ измерим.
	Будем говорить, что функция $ f $, определённая на подмножестве $ A \subset \Omega $,
	\defin{измерима на $ A $}{locally-measurable}, если также для любого борелевского множества $ B $ его прообраз $ f^{-1}(B) \subset A $ лежит в $ \events $ 
	(это равносильно тому, что ограничение $ \left.f\right|_{A} $ является измеримым отображением
	$ \left.f\right|_{A} \colon (A, \events \cap A) \to (\RR, \calB) $).
	
	Далее будем считать, что на $ \sigma $-алгебре $ \events $ задана $ \sigma $-аддитивная мера $ \mu $
	и элементы $ \events $ мы будем называет измеримыми множествами.
	
	\begin{proposition} \label{composition with continuous}
		Пусть функция $ f $ измерима и $ g $ --- непрерывная на $ \Image f $ функция.
		Тогда композиция $ g \circ f $ измерима.
	\end{proposition}
	
	\begin{proof}
		Пусть $ U $ --- открытое множество.
		Тогда $ g^{-1}(U) $ открыто и, следовательно, $ f^{-1}(g^{-1}(U)) $ измеримо.
	\end{proof}
	
	\begin{proposition} \label{measurable functions are good}
		Пусть $ f, g $ --- измеримые функции.
		Тогда множество $ A_{f \leqslant g} = \defineset{x \in \Omega}{f(x) \leqslant g(x)} $
		измеримо.
		Функции $ a + f, af, |f|, f^2, f + g $ и $ fg $, где $ a $ --- константа, измеримы.
		Если функция $ g $ не принимает значения 0, то функции $ \tfrac{1}{g} $ и $ \tfrac{f}{g} $
		измеримы.
	\end{proposition}
	
	\begin{proof}
		Как было замечено в определении измеримой функции,
		для проверки измеримости функции $ h $ достаточно доказать, что 
		для любой константы $ t $ множество $ A_{h \leqslant t} $ измеримо. 
		
		Первое утверждение следует из представления
		$$ A_{f \leqslant g} = \bigcup\limits_{q \in \mathbb{Q}} \defineset{x \in \Omega}{f(x) < q < g(x)}
		= \bigcup\limits_{q \in \mathbb{Q}} \left(\defineset{x \in \Omega}{f(x) < q} 
		\cap \defineset{x \in \Omega}{q < g(x)} \right), $$
		где все множества справа измеримы, так как измеримы функции $ f $ и $ g $.
		
		Измеримость функций $ a + f af, |f|, f^2, \tfrac{1}{g} $ следует из предложения \ref{composition with continuous}.
		
		По доказанному выше функция $ a - g $ измерима, 
		Тогда также по доказанному выше множество $ A_{f \leqslant a - g} $ измеримо и, 
		поэтому измерима функция $ f + g $.
		
		Измеримость функции $ fg $ следует из представления $ fg = \tfrac{1}{4}(f + g)^2 - \tfrac{1}{4}(f - g)^2 $
		и доказанного выше. Из этого следует измеримость функции $ \tfrac{f}{g} $.
	\end{proof}
	
	%\begin{proposition}
	%	Пусть $ \{f_n\} $ --- последовательность простые функций на $ \Omega $.
	%	Тогда $ \sup\limits_{n \in \mathbb{N}} f_n, \inf\limits_{n \in \mathbb{N}} f_n,
	%	\limsup\limits_{n \to +\infty} f_n, \liminf\limits_{n \to +\infty} f_n $
	%	являются измеримыми. Множество определения функции $ \lim\limits_{n \to +\infty} f_n $
	%	измеримо и она сама измерима на нём.
	%\end{proposition}
	
	%\begin{proof}
	%	Имеем $ \defineset{x \in \Omega}{\sup f_n(x) < a} 
	%	= \bigcap_{n \in \mathbb{N}} \defineset{x \in \Omega}{f_n(x) < a}$ --- измеримо.
	%	Для инфимума выполнено так как $ \inf f_n = -\sup (-f_n) $.
	%	Для верхнего и нижнего пределов выполнено так как
	%	$ \limsup\limits_{n \to +\infty} f_n = \inf\limits_{n \to +\infty} \sup\limits_{k \geqslant n} f_k $
	%	и $ \liminf\limits_{n \to +\infty} f_n = \sup\limits_{n \to +\infty} \inf\limits_{k \geqslant n} f_k $
	%	
	%	Множество определения функции $ \lim\limits_{n \to +\infty} f_n $
	%	в точности равно множеству, на котором совпадают
	%\end{proof}

	
	
	\defin{Простой функцией}{simple-function} мы будем называть измеримую функцию $ f $ такую, что
	множество значений $ f $ конечно и мера множества, на котором $ f $ принимает ненулевые значения конечна.
	
	\defin{Обобщённой простой функцией}{deneralized-simple-function} мы будем называть произвольную измеримую функцию, 
	принимающую не более чем счётное число значений. Всякая простая функция является обобщённой простой функцией.
	
	\defin{Индикатором множества $ A $}{indicator} мы будем называть функцию $ \ind_{A} $, заданную по правилу
	$$ \ind_A(x) = \definefuntwo{1}{x \in A}{0}{x \notin A}. $$
	
	\begin{proposition} \label{simple function description}
		Верно следующее.
		\begin{enumerate}
			\item Индикатор $ \ind_A $ измеримого множества $ A $ является простой функцией. 
			\label{simple function description | indicator}
			\item Всякая $ ( $обобщённая$ ) $ простая функция 
			единственным образом может быть представления в виде конечной линейной комбинации $ ( $ряда$ ) $
			с попарно различными коэффициентами индикаторов попрано непересекающихся измеримых множеств, 
			объединение которых равно $ \Omega $.
			\label{simple function description | decomposition}
			\item Всякая $ ( $обобщённая$ ) $ простая функция 
			единственным образом может быть представления в виде линейной комбинации
			$ ( $ряда$ ) $ с попарно различными ненулевыми коэффициентами 
			индикаторов попрано непересекающихся измеримых множеств.
			\label{simple function description | non-zero decomposition}
			\item Всякая простая функция является обобщённой простой и её представления из пунктов выше
			как для простой и как для обобщённой простой функции совпадают.
			\label{simple function description | simple and generalized simple}
			\item Ряд из индикаторов с коэффициентами $ ( $не обязательно различными$ ) $ 
			 попарно непересекающихся измеримых множеств является обобщённой простой функций.
			\label{simple function description | construction}
		\end{enumerate}
	\end{proposition}
	
	\begin{proof}
		Индикатор множества принимает всего два значения --- 0 и 1.
		Так как $ A $ измеримо, то прообраз любого борелевского множества относительно индикатора
		есть одно из четырёх измеримых множеств: $ \varnothing, A, \Omega \setminus A, \Omega $.
		Следовательно, $ \ind_{A} $ --- простая функция.
		
		Пусть $ \Image f = \{c_i\}_{i = 1}^{n (+\infty)} $ --- образ функции $ f $.
		Множества $ A_i = f^{-1}(\{c_i\}) $ измеримы, так как $ f $ измерима, 
		попарно не пересекаются и покрывают всё $ \Omega $.
		Тогда $$ f = \sum\limits_{i = 1}^{n (+\infty)} c_i\ind_{A_i}. $$
		Пусть имеется другое представление $ f $ в конечной суммы (ряда), 
		удовлетворяющее условию пункта \ref{simple function description | decomposition}.
		Пусть множество $ B $, индикатор которого входит в это представление не совпадает ни с одним из множеств $ A_i $.
		Если $ B $ строго содержится в некотором $ A_i $, то найдётся её одно множество $ C $ из второго разбиения, 
		пересекающееся с $ A_i $. Тогда функция $ f $ принимает на $ B $ и на $ C $ равные значения, что противоречит условию.
		Иначе $ B $ пересекается как минимум с двумя множествами $ A_i $ 
		и функция $ f $ принимает на $ B $ не менее двух различных значений, что снова противоречит условию.
		Следовательно, данное представление единственно.
		
		Чтобы получить представление для пункта \ref{simple function description | non-zero decomposition}
		удалим из построенной суммы (ряда) слагаемое, соответствующее $ c_i = 0 $.
		Доказательство единственности аналогично.
		
		Построенное выше представление для простой функции содержит только конечное число ненулевых слагаемых,
		поэтому выполнен пункт \ref{simple function description | simple and generalized simple}.
		
		Функция, построенная в пункте \ref{simple function description | construction} принимает счётное число различных значений,
		причём каждое конкретное значение --- на измеримом множестве.
		Тогда прообраз всякого борелевского множества является объединением (не более, чем счётным) этих множеств
		(и, возможно, дополнения до их объединения) и поэтому измерим.
	\end{proof}
	
	\begin{proposition}
		Сумму двух $ ( $обобщённых$ ) $ простых функций является $ ( $обобщённой$ ) $ простой функцией.
		Функция, пропорциональная $ ( $обобщённой$ ) $ простой функции, сама является таковой.
		Модуль $ ( $обобщённой$ ) $ простой функции, является таковой функцией.
	\end{proposition}

	\begin{proof}
		Сумма двух измеримых функций принимающих конечное (не более, чем счётное) число значений снова является
		измеримой функцией по предложению \ref{measurable functions are good} 
		и снова принимает конечное (не более, чем счётное) число значений.
		
		Функция, пропорциональная измеримой, измерима и модуль измеримой функции измерим по предложению 
		\ref{measurable functions are good}. Модуль функции, принимающей не более, чем счётное число значений
		и эта функция умноженная на константу также принимают не более, чем счётное число значений.
	\end{proof}

	\begin{proposition} \label{uniformly converging to measurable function}
		Для всякой измеримой функции $ f $ существует последовательность обобщённых простых функций $ \{f_n\} $,
		равномерно сходящаяся к $ f $.
	\end{proposition}
	
	\begin{proof}
		Поскольку $ f $ измерима, то для всяких натурального $ n $ и целого $ k $
		множество $ A_{n, k} = f^{-1}([\tfrac{k}{2^n}, \tfrac{k + 1}{2^n})) $ измеримо
		и для фиксированного $ n $ и различных $ k $ эти множества образуют разбиение $ \Omega $.
		Положим 
		$$ f_n(x) = \sum\limits_{k \in \mathbb{Z}} \tfrac{k}{2^n}\ind_{A_{n, k}}. $$
		Тогда $ f_n $ удовлетворяет определению обобщённой простой функции.
		Кроме того, для всякого $ n $ и $ x \in \Omega $ 
		выполнены неравенства $ 0 \leqslant f(x) - f_n(x) \leqslant \tfrac{1}{2^n} $.
		Поэтому последовательность $ \{f_n\} $ сходится к $ f $ равномерно.
	\end{proof}
	
	\subsubsection{Интеграл Лебега}
	
	Пусть $ f = \sum\limits_{i = 1}^{n}c_i\ind_{A_i}, c_i \neq 0 $ --- простая функция в представлении из предложения
	$ \ref{simple function description} $.
	\defin{Интегралом простой функции $ f $ на пространстве $ \Omega $ по мере $ \mu $}{simple-function-integral} 
	мы назовём сумму $ \sum\limits_{i = 1}^{n} c_i\cdot\mu(A_i) $ и обозначим через $ \int\limits_{\Omega} f\diff\mu $.
	
	Обобщённую простую функцию $ f = \sum\limits_{i = 1}^{+\infty}c_i\ind_{A_i} $ (в представлении из предложения
	$ \ref{simple function description} $) назовём 
	\defin{интегрируемой по Лебегу на пространстве $ \Omega $ по мере $ \mu $}{generalized-simple-function-integrability},
	если для $ c_i \neq 0 $ меры $ \mu(A_i) $ конечны и 
	ряд $ \sum\limits_{i = 1}^{+\infty} c_i\cdot\mu(A_i) $ сходится абсолютно (предполагается, что $ 0 \cdot \mu(A) = 0 $, даже если мера $ \mu(A) $ бесконечна).
	В этом случае обозначим через $ \int\limits_{\Omega} f\diff\mu $ сумму ряда и назовём её
	\defin{интегралом обобщённой простой функции $ f $ на пространстве $ \Omega $ по мере $ \mu $}{generalized-simple-function-integral}. 
	
	Из определения простой функции следует, что она является интегрируемой на пространстве $ \Omega $ по мере $ \mu $
	обобщённой простой функцией.
	
	\begin{lemma} \label{integrability of generalized simple functions}
		Пусть $ f, g $ ---  обобщённые простые функции.
		Тогда
		\begin{enumerate}
			\item если $ f, g $ --- интегрируемые на $ \Omega $ по мере $ \mu $, 
			то функция $ f + g $ интегрируема и имеет место формула
			$$ \int\limits_{\Omega} f\diff\mu + \int\limits_{\Omega} g\diff\mu = \int\limits_{\Omega} (f + g)\diff\mu; $$
			\item интегрируемость функции $ f $ равносильна интегрируемости функции $ |f| $ 
			и в случае интегрируемости справедливо неравенство
			$$ \left|\int\limits_{\Omega} f\diff\mu\right| \leqslant \int\limits_{\Omega} |f|\diff\mu; $$
			\item если $ |f| \leqslant g $ и $ g $ интегрируема, то $ |f| $ интегрируема и выполнено неравенство
			$$ \int\limits_{\Omega} |f|\diff\mu \leqslant \int\limits_{\Omega} g\diff\mu. $$
		\end{enumerate}
	\end{lemma}
	
	\begin{proof}
		Первые два свойства следуют из свойств сходимости абсолютно сходящихся рядов.
		Третье --- из признака абсолютной сходимости.
	\end{proof}
	
	\begin{lemma} \label{finite measure and generalized simple functions}
		Пусть мера $ \mu $ конечна на $ \Omega $. Тогда выполнены следующие утверждения.
		\begin{enumerate}
			\item Всякая ограниченная обобщённая простая функция интегрируема на $ \Omega $ по $ \mu $.
			\item Интеграл $ \int\limits_{\Omega} \ind_{A}\diff\mu $ от индикатора $ \ind_{A} $ равен мере $ A $.
		\end{enumerate}
	\end{lemma}
	
	\begin{proof}
		Пусть константа $ C $ ограничивает функцию $ f $. Так как мера $ \Omega $ конечна, 
		то функция $ C\cdot \ind_{\Omega} $ интегрируема, откуда по лемме \ref{integrability of generalized simple functions} следует, что $ f $ интегрируема.
		Второе утверждение выполнено из построения интеграла обобщённой простой функции
	\end{proof}
	
	\begin{theorem} \label{integrability}
		Пусть мера $ \mu $ конечна на $ \Omega $.
		Пусть $ f $ --- измеримая функция и существует последовательность интегрируемых обобщённых простых функций $ \{f_n\} $,
		равномерно сходящаяся к $ f $. Тогда
		\begin{enumerate}
			\item существует предел $ \int\limits_{\Omega} f_n \diff\mu =: I; $
			\item для любой другой последовательности интегрируемых обобщённых простых функций $ \{g_n\} $,
			равномерно сходящихся к $ f $ предел $ \int\limits_{\Omega} g_n \diff\mu $ также равен $ I; $
			\item если $ f $ --- обобщённая простая функция, то $ f $ интегрируема и $ \int\limits_{\Omega} f\diff\mu = I. $
		\end{enumerate}
	\end{theorem}
	
	\begin{proof}
		Фиксируем $ \varepsilon > 0 $.
		Так как функции $ f_n $ равномерно сходятся к $ f $, то начиная с некоторого $ N $
		для $ n, m > N $ и для любого $ x \in \Omega $ выполнено неравенство $ |f_n(x) - f_m(x)| < \varepsilon $.
		Тогда по леммам \ref{finite measure and generalized simple functions} и \ref{finite measure and generalized simple functions} имеем
		$$ \left|\int\limits_{\Omega} f_n\diff\mu - \int\limits_{\Omega} f_m\diff\mu\right| \leqslant \int\limits_{\Omega} |f_n - f_m|\diff\mu \leqslant \varepsilon \mu(X). $$
		По критерию Коши существует предел $ \int\limits_{\Omega} f_n \diff\mu =: I. $
		
		Заметим, что если $ f = 0 $, то для $ n > N(\varepsilon) $ выполнено равенство
		$$ \left|\int\limits_{\Omega} f_n\diff\mu\right| < \varepsilon\mu(X). $$
		Поэтому в данном случае $ I = 0 $. Поскольку, если последовательности $ f_n $ и $ g_n $ сходятся равномерно
		к $ f $, то последовательность $ f_n - g_n $ сходится равномерно к нулевой функции,
		то предел $ I $ будет одинаковым для обеих последовательностей.
		
		Пусть $ f $ сама является обобщённой простой функцией.
		Найдётся индекс $ n $ такой, что для любого $ x \in \Omega $
		выполнено неравенство $ |f(x) - f_n(x)| < 1 $.
		По леммам \ref{finite measure and generalized simple functions}
		и \ref{integrability of generalized simple functions}
		функция $ f - f_n $ интегрируема.
		Так как $ f_n $ интегрируема, то снова по лемме \ref{integrability of generalized simple functions} функция $ f = (f - f_n) + f_n $ интегрируема.
		Теперь возьмём в качестве последовательности $ f_n $ постоянную последовательность из функции $ f $.
		Тогда предел этой последовательности равен одновременно $ \int\limits_{\Omega} f\diff\mu $ и $ I $.
	\end{proof}
	
	Измеримая функция $ f $ на $ \Omega $ с $ \mu(\Omega) < +\infty $ называется \defin{интегрируемой}{integrable},
	если существует последовательность интегрируемых обобщённых простых функций $ \{f_n\} $,
	равномерно сходящаяся к $ f $. \defin{Интегралом (Лебега) $ f $ на $ \Omega $ по мере $ \mu $}{integral} 
	называется предел $ \int\limits_{\Omega} f_n \diff\mu $. В дальнейшем мы будем обозначать его через $ \int\limits_{\Omega} f\diff\mu $.
	
	\begin{corollary} \label{any sequence}
		Путь мера $ \mu $ конечна на $ \Omega $.
		Пусть измеримая функция $ f $ интегрируема на $ \Omega $ по $ \mu $
		и существует последовательность обобщённых простых функций $ \{f_n\} $, равномерно сходящаяся к $ f $.
		Тогда начиная с некоторого номера $ N $ все функции в этой последовательности интегрируемы.
	\end{corollary}
	
	\begin{proof}
		Рассмотрим некоторую последовательность интегрируемых обобщённых простых функций $ \{f_n\} $,
		равномерно сходящуюся к $ f $.
		Тогда последовательность простых функций $ \{f_n - g_n\} $ равомерно сходится к тождественно нулевой функции.
		Следовательно, начиная с некоторого индекса $ N $ для всех $ n > N $ и $ x \in \Omega $ выполнено неравенство
		$ |f_n(x) - g_n(x)| < 1 $.
		Тогда по леммам $ \ref{simple function description} $ и $ \ref{integrability of generalized simple functions} $ функция $ g_n - f_n $ интегрируема. Тогда интегрируема и функция $ g_n = (g_n - f_n) + f_n $.
	\end{proof}
	
	\subsubsection{Прямой образ меры (pushforward measure)}
	
	Пусть $ (X, \calA_X) $ и $ (Y, \calA_Y) $ --- \hyperlink{measurable-space}{измеримые пространства},
	$ \mu $ --- мера на $ \sigma $-алгебре $ \calA_X $ и $ f \colon (X, \calA_X) \to (Y, \calA_Y) $
	--- \hyperlink{morphism-of-measurable-spaces}{измеримое отображение}.
	Тогда \defin{прямым образом меры $ \mu $ при отображении $ f $}{pushforward-measure} называется функция $ f_{*}\mu $,
	заданная по правилу $ f_*\mu(A) = \mu(f^{-1}(A)) $ для любого $ A \in \calA_Y $.
	
	\begin{proposition}
		Прямой образ ($ \sigma $-аддитивной) меры $ f_*\mu $ является ($ \sigma $-аддитивной) мерой на алгебре $ \calA_Y $.
	\end{proposition}
	
	\begin{proof}
		Пусть $ \{A_k\} $ --- последовательность попрано не пересекающихся множеств из $ \calA_Y $.
		Тогда множества из последовательности $ \{f^{-1}(A_k)\} $ также попарно не пересекаются. 
		Из определения прямого образа меры $ f_{*}\mu $ имеем цепочку равенств
		$$ f_*\mu\left(\bigsqcup\limits_{k = 1}^{+\infty} A_k\right) 
		= \mu\left(f^{-1}\left(\bigsqcup\limits_{k = 1}^{+\infty} A_k\right)\right)
		= \mu\left(\bigsqcup\limits_{k = 1}^{+\infty} f^{-1}(A_k)\right)
		= \sum\limits_{k = 1}^{+\infty} \mu(f^{-1})(A_k)
		= \sum\limits_{k = 1}^{+\infty} f_{*}\mu(A_k). $$
		
		Доказательство аддитивности проводится аналогично.
	\end{proof}
	
	\begin{theorem} \label{pushforward measure main property}
		Пусть мера $ \mu $ конечна на $ X $.
		Рассмотрим измеримую функцию $ g \colon (Y, \calA_Y) \to (\RR, \calB) $.
		Тогда $ g $ интегрируема на $ Y $ по прямому образу меры $ f_*\mu $
		тогда и только тогда, когда композиция $ g \circ f $ интегрируема на $ X $ по мере $ \mu $.
		В случае интегрируемости равны интегралы
		$$ \int\limits_{X} (g \circ f)\diff\mu = \int\limits_{Y} g\diff f_*\mu. $$
	\end{theorem}
	
	\begin{proof}
		Из определения обобщённых простых функций $ g $ и $ g \circ f $ одновременно являются или не являются простыми обобщёнными функциями.
		
		Для обобщённой простой функции истинность утверждения следует из того, что
		интегралы $ \int\limits_{X} (g \circ f)\diff\mu $ и $ \int\limits_{Y} g\diff f_*\mu $
		равны сумме одного и того же ряда 
		(поскольку $ f_*\mu(g^{-1}(\{a\})) = \mu(f^{-1}(g^{-1}(\{a\}))) = \mu((g \circ f)^{-1}(\{a\})) $
		для $ a \in \RR $).
		
		Если существует последовательность интегрируемых обобщённых простых функций $ \{g_n\} $ на $ Y $,
		равномерно сходящаяся к $ g $, 
		то последовательность $ \{g_n \circ f\} $ 
		будет являться последовательностю интегрируемых обобщённых простых функций на $ X $,
		равномерно сходящейся к $ g \circ f $.
		Следовательно, из интегрируемости $ g $ следует интегрируемость $ g \circ f $.
		
		Обратно, пусть $ g \circ f $ интегрируема. 
		По предложению $ \ref{uniformly converging to measurable function} $
		существует последовательность обобщённых простых функций $ \{g_n\} $ на $ Y $,
		равномерно сходящаяся к $ g $. 
		Тогда последовательность обобщённых простых функций $ \{g_n \circ f\} $ равномерно сходится к $ g \circ f $.
		По следствию \ref{any sequence} начиная с некоторого номера $ N $ функции $ g_n \circ f $ являются интегрируемыми.
		По доказанному выше, это означает, что для всех $ n > N $ функции $ g_n $ интегрируемы,
		поэтому $ g $ интегрируема.
		
		Пусть теперь $ g $ и $ g \circ f $ интегрируемы.
		Фиксируем последовательность интегрируемых обобщённых простых функций $ \{g_n\} $ на $ Y $,
		равномерно сходящуюся к $ g $.
		Имеем цепочку равенств
		$$ \int\limits_{X} (g \circ f)\diff\mu = \lim\limits_{n \to +\infty} \int\limits_{X} (g_n \circ f)\diff\mu
		= \lim\limits_{n \to +\infty} \int\limits_{Y} g_n \diff f_{*}\mu = \lim\limits_{n \to +\infty} \int\limits_{Y} g\diff f_{*}\mu. $$
	\end{proof}
	
	\subsection{Теория категорий и взгляд на измеримые пространства с её точки зрения}
	
	Теория категорий в её лучшем проявлении выражает собой формализацию понятия <<математическая конструкция>>
	через понятия объектов, морфизмов, функторов, естественных преобразований,
	а также формализует интуицию в виде универсальных свойств, сопряжённости функторов, эквивалентности категорий и так далее.
	
	В этом подразделе будет предполагаться, что вы знакомы с определением категории 
	(например, основанном на теории множеств) и знакомы с понятиями объекта, морфизма между объектами и функтором из одной категории в другую. Остальные определения по возможности будут приведены здесь.
	
	\subsubsection{Категория измеримых пространств}
	
	Пусть $ \calA_X $ --- $ \sigma $-алгебра подмножеств множества $ X $.
	Пара $ (X, \calA_X) $ называется \defin{измеримым пространством}{measurable-space}.
	
	Пусть $ (X, \calA_X) $ и $ (Y, \calA_Y) $ --- пара измеримых пространств.
	\defin{Морфизмом измеримых пространств (измеримым отображением)}{morphism-of-measurable-spaces} 
	$ f \colon (X, \calA_X) \to (Y, \calA_Y) $ называется отображение
	множеств $ f \colon X \to Y $ такое, что для любого $ U \in \calA_Y $ выполнено $ f^{-1}(U) \in \calA_X $,
	где $ f^{-1} $ --- это полный прообраз.
	
	В категории измеримых пространств $ \Meas $ объектами являются измеримые пространства, а морфизмами
	--- морфизмы измеримых пространств.
	
	\subsection{Произведение и копроизведение к категории измеримых пространств}
	
	Пусть $ (X, \calA_X) $ и $ (Y, \calA_Y) $ --- измеримые пространства.
	Построим по ним новые измеримые пространства следующим образом.
	
	\defin{Произведением измеримых пространств $ (X, \calA_X) $ и $ (Y, \calA_Y) $}{product-of-measurable-spaces}
	мы назовём измеримое пространство $ (X \times Y, \calA_X \otimes \calA_Y) $,
	где $ \calA_X \otimes \calA_Y $ есть минимальная $ \sigma $-алгебра, порождённая полукольцом
	всех возможных декартовых произведений $ A_X \times A_Y $,
	где $ A_X \in \calA_X $ и $ A_Y \in \calA_Y $.
	
	\begin{proposition}
		Пусть $ (X, \calA_X) $ и $ (Y, \calA_Y) $ --- измеримые пространства,
		$ (X \times Y, \calA_X \otimes \calA_Y) $ --- их произведение.
		Тогда
		\begin{enumerate}
			\item отображения проекций $ \pr_X \colon X \times Y \to X $
			и $ \pr_Y \colon X \times Y \to Y $ являются морфизмами измеримых пространств.
			\item пространство $ (X \times Y, \calA_X \otimes \calA_Y) $ удовлетворяет универсальному свойству произведения,
			то есть для всякого измеримого пространства $ (Z, \calA_Z) $ 
			и всяких двух морфизмов $ f \colon (Z, \calA_Z) \to (X, \calA_X) $,
			$ g \colon (Z, \calA_Z) \to (Y, \calA_Y) $
			существует единственный морфизм $ f \times g \colon (Z, \calA_Z) \to (X \times Y, \calA_X \otimes \calA_Y) $,
			удовлетворяющий условию $ \pr_X \circ (f \times g) = f $ и $ \pr_Y \circ (f \times g) = g $.
		\end{enumerate}
		$$ \begin{tikzcd}
			& (Z, \calA_Z) \ar{dl}[above left]{f} \ar{dr}{g} \ar[dashed]{dd}[description]{f \times g} & \\
			(X, \calA_X) &  & (Y, \calA_Y) \\
			& (X \times Y, \calA_X \otimes \calA_Y) \ar{ul}{\pr_X} \ar{ur}[below right]{\pr_Y} &
		\end{tikzcd} $$
	\end{proposition}
	
	
	\subsubsection{Прямой образ $ \sigma $-алгебры}
	
	Пусть $ (X, \calA_X) $ --- измеримое пространство и $ f \colon X \to Y $ --- отображение множеств.
	Существует способ естественным образом построить $ \sigma $-алгебру подмножеств $ Y $.
	
	Положим $ f_{*}(\calA_X) = \defineset{B \subset Y}{f^{-1}(B) \in \calA_X} $ --- все подмножества $ Y $,
	полный прообраз которых лежит в $ \calA_X $.
	
	\begin{proposition} \label{direct image of sigma algebra}
		Конструкция $ f_{*} $ обладает следующими свойствами.
		\begin{enumerate}
			\item Система множеств $ f_{*}(\calA_X) $ является $ \sigma $-алгеброй подмножеств множества $ Y $.
			\item Отображение $ f $ является морфизмом измеримых пространств $ f \colon (X, \calA_X) \to (Y, f_{*}(\calA_X)) $.
			\item Если $ \calA_Y $ --- $ \sigma $-алгебра подмножеств $ Y $,
			то отображение $ f $ является морфизмом измеримых пространств $ f \colon (X, \calA_X) \to (Y, \calA_Y) $
			тогда и только тогда, когда $ \calA_Y \subset f_{*}(\calA_X) $.
			\item Пусть имеются отображения множеств $ f \colon X \to Y, g \colon X \to Z $ и $ h \colon Y \to Z $
			такие, что $ h \circ f = g $, то $ h $ является морфизмом измеримых пространств
			$ h \colon (Y, f_{*}(\calA_X)) \to (Z, g_{*}(\calA_X)). $
		\end{enumerate}
		$$ \begin{tikzcd}
			 & Y \ar{dd}{h} \\
			X \ar{ru}{f} \ar{rd}{g} &  \\
			& Z, \\
		\end{tikzcd} \
		\begin{tikzcd}
			& (Y, f_{*}(\calA_X)) \ar{dd}{h} \\
			(X, \calA_X) \ar{ru}{f} \ar{rd}{g} &  \\
			& (Z, g_{*}(\calA_X)). \\
		\end{tikzcd} $$
	\end{proposition}
	
	\begin{proof}
		Первый пункт проверяется непосредственно и следует из теоретико-множественных тождеств.
		Второй пункт --- из построения $ f_{*}(\calA_X) $, третий --- из определения морфизма измеримых пространств.
		
		Докажем четвёртый пункт. Пусть $ U \in g_*(\calA_X) $. Тогда $ g^{-1}(U) = f^{-1}(h^{-1}(U)) \in \calA_X $.
		По определению прямого образа $ f_*(\calA_X) $ множество $ h^{-1}(U) $ содержится в нём.
		Следовательно, $ h $ является морфизмом измеримых пространств.
	\end{proof}
	
	\subsubsection{Обратный образ $ \sigma $-алгебры}
	
	Пусть $ (Y, \calA_Y) $ --- измеримое пространство и $ f \colon X \to Y $ --- отображение множеств.
	Существует способ естественным образом построить $ \sigma $-алгебру подмножеств $ X $.
	
	Положим $ f^{*}(\calA_Y) = \defineset{f^{-1}(B)}{B \in \calA_Y} $ --- полные прообразы всех элементов $ \sigma $-алгебры $ \calA_Y $.
	
	\begin{proposition} \label{inverse image of sigma algebra}
		Конструкция $ f^{*} $ обладает следующими свойствами.
		\begin{enumerate}
			\item Система множеств $ f^{*}(\calA_Y) $ является $ \sigma $-алгеброй подмножеств множества $ X $.
			\item Отображение $ f $ является морфизмом измеримых пространств $ f \colon (X, f^{*}(\calA_Y)) \to (Y, \calA_Y) $.
			\item Если $ \calA_X $ --- $ \sigma $-алгебра подмножеств $ X $,
			то отображение $ f $ является морфизмом измеримых пространств $ f \colon (X, \calA_X) \to (Y, \calA_Y) $
			тогда и только тогда, когда $ f^{*}(\calA_Y) \subset \calA_X $.
			\item Пусть имеются отображения множеств $ f \colon X \to Y, g \colon Z \to Y $ и $ h \colon X \to Z $
			такие, что $ g \circ h = f $, то $ h $ является морфизмом измеримых пространств
			$ h \colon (X, f^{*}(\calA_Y)) \to (Z, g^{*}(\calA_Y)). $
		\end{enumerate}
		$$ \begin{tikzcd}
			X \ar{dd}{h} \ar{dr}{f} & \\
			& Y, \\
			Z \ar{ur}{g} & \\
		\end{tikzcd} \
		\begin{tikzcd}
			(X, f^{*}(\calA_Y)) \ar{dd}{h} \ar{dr}{f} & \\
			 & (Y, \calA_Y).  \\
			(Z, g^{*}(\calA_Y)) \ar{ur}{g} &  \\
		\end{tikzcd} $$
	\end{proposition}
	
	\begin{proof}
		Первый пункт проверяется непосредственно и следует из теоретико-множественных тождеств.
		Второй пункт --- из построения $ f^{*}(\calA_X) $, третий --- из определения морфизма измеримых пространств.
		
		Докажем четвёртый пункт. Пусть $ U \in g^*(\calA_Y) $. Тогда для некоторого $ V \in \calA_Y $ выполнено $ g^{-1}(V) = U $. Далее, $ h^{-1}(U) = h^{-1}(g^{-1}(V)) = f^{-1}(V) $.
		По определению обратного образа $ f^*(\calA_X) $ множество $ h^{-1}(U) $ содержится в нём.
		Следовательно, $ h $ является морфизмом измеримых пространств.
	\end{proof}
	
	\subsubsection{Связь между минимальной $ \sigma $-алгеброй, прямым и обратным образами $ \sigma $-алгебры}
	
	\begin{lemma} \label{inclusions direct-inverse}
		Пусть $ X, Y $ --- множества, $ f \colon X \to Y $ --- отображение множеств.
		Если $ \calA_X $ --- $ \sigma $-алгебра подмножеств $ X $, то $ f^*(f_*(\calA_X)) \subset  \calA_X $.
		Если $ \calA_Y $ --- $ \sigma $-алгебра подмножество $ Y $, то $ \calA_Y \subset f_*(f^*(\calA_Y)) $.
	\end{lemma}
	
	\begin{proof}
		По предложению \ref{direct image of sigma algebra} отображение $ f $ является
		морфизмом измеримых пространств $ f \colon (X, \calA_X) \to (Y, f_*(\calA_X)) $.
		Тогда по предложению \ref{inverse image of sigma algebra} имеем включение
		$ f^*(f_*(\calA_X)) \subset  \calA_X $.
		
		Теперь По предложению \ref{inverse image of sigma algebra} отображение $ f $ является
		морфизмом измеримых пространств $ f \colon (X, f^*(\calA_Y)) \to (Y, \calA_Y) $.
		Тогда по предложению \ref{direct image of sigma algebra} имеем включение
		$ \calA_Y \subset f_*(f^*(\calA_Y)) $.
	\end{proof}
	
	Рассмотрим категорию $ \sigma $-$ \mathfrak{Alg}_X $ всех $ \sigma $-алгебр с единицей $ X $,
	в которой объектами выступают $ \sigma $-алгебры подмножеств $ X $,
	а единственный существующий морфизм из $ \sigma $-алгебры $ \calA $ идёт в $ \sigma $-алгебру $ \calA' $,
	если $ \calA \subset \calA' $. Построенные нами для отображения множеств $ f \colon X \to Y $
	конструкции $ f_* $ и $ f^* $ являются функторами между категориями $ \sigma $-$ \mathfrak{Alg}_X $
	и $ \sigma $-$ \mathfrak{Alg}_Y $ и, более того, что эти функторы сопряжены.
	
	\begin{theorem}
		Пусть $ f \colon X \to Y $ --- отображение множеств.
		Тогда $ f_* \colon \sigma $\text{-}$ \mathfrak{Alg}_X \to \sigma $\text{-}$ \mathfrak{Alg}_Y $
		и $ f^* \colon \sigma $\text{-}$ \mathfrak{Alg}_Y \to \sigma $\text{-}$ \mathfrak{Alg}_X $
		--- функторы.
		Кроме того, функтор $ f^* $ является левым сопряжённым к функтору $ f_* $ 
		(и обратно, функтор $ f_* $ является правым сопряжённым к $ f^* $). 
		Другими словами, для любого объекта $ \calA_X \in \sigma $\text{-}$ \mathfrak{Alg}_X $
		и любого объекта $ \calA_Y \in \sigma $\text{-}$ \mathfrak{Alg}_Y $ и имеется естественный изоморфизм
		множеств $ \Hom(f^*(\calA_Y), \calA_X)) \cong \Hom(\calA_Y, f_*(\calA_X)) $.
	\end{theorem}
	
	\begin{proof}
		Из построения $ f_* $ и $ f^* $ сохраняют отношения включения и равенства (и, следовательно, композицию включений),
		поэтому они функториальны.
		
		Теперь по предложению \ref{inverse image of sigma algebra} имеет место включение
		$ f^*(\calA_Y) \subset \calA_X $ тогда и только тогда, когда $ f \colon (X, \calA_X) \to (Y,\calA_Y) $
		является измеримым отображением.
		По предложению \ref{direct image of sigma algebra} последнее равносильно включению
		$ \calA_X \subset f_*(\calA_Y) $.
		Таким образом, множества $ \Hom(f^*(\calA_Y), \calA_X)) $ и $ \Hom(\calA_Y, f_*(\calA_X)) $
		одновременно либо пусты, либо состоят из одного элемента.
		Тогда пусть данный изоморфизм будет пустым в первом случае 
		и сопоставляет единственный элемент единственному элементу во втором случае.
		
		Если имеются включения $ \calA_{X1} \subset \calA_{X2} $ и $ \calA_{Y1} \subset \calA_{Y2} $,
		то возникает коммутативная диаграмма, индуцированная этими включениями, выражающая естественность
		изоморфизма
		\begin{center}
			\begin{tikzcd}[row sep=small,column sep=tiny]
				& \Hom(f^*(\calA_{Y2}), \calA_{X2})) \ar[dotted]{dd} \ar{rr} & & \Hom(\calA_{Y2}, f_*(\calA_{X2})) \ar{dd} \\
				\Hom(f^*(\calA_{Y2}), \calA_{X1})) \ar{rr} \ar{ur} \ar{dd} & & \Hom(\calA_{Y2}, f_*(\calA_{X1})) \ar{ur} \ar{dd} \\
				& \Hom(f^*(\calA_{Y1}), \calA_{X2})) \ar[dotted]{rr} & & \Hom(\calA_{Y1}, f_*(\calA_{X2})) & \\
				\Hom(f^*(\calA_{Y1}), \calA_{X1})) \ar{rr} \ar[dotted]{ru} & & \Hom(\calA_{Y1}, f_*(\calA_{X1})) \ar{ur} &
			\end{tikzcd}
		\end{center}
	\end{proof}
	
	\begin{lemma}
		Пусть $ X, Y $ --- множества, $ T $ --- система подмножеств $ Y $, $ Y \in T $.
		Пусть $ f \colon X \to Y $ --- отображение множеств.
		Обозначим через $ f^{-1}(T) $ систему множеств $ \defineset{f^{-1}(U)}{U \in T} $.
		Тогда $ R_{\sigma}(f^{-1}(T)) = f^{*}(R_{\sigma}(T)) $.
	\end{lemma}
	
	\begin{proof}
		Так как $ T \subset R_{\sigma}(T) $, то по построению $ f^* $ имеем включение $ f^{-1}(T) \subset f^*(R_{\sigma}(T)) $.
		По предложению \ref{inverse image of sigma algebra} система множеств $ f^*(R_{\sigma}(T)) $ является $ \sigma $-алгеброй.
		Тогда по минимальности (теореме \ref{min ring exists}) 
		имеем включение $ R_{\sigma}(f^{-1}(T)) \subset f^{*}(R_{\sigma}(T)) $.
		
		Далее, система множеств $ f_*(R_{\sigma}(f^{-1}(T))) $ является $ \sigma $-алгеброй по предложению
		\ref{direct image of sigma algebra}. По построению $ f_* $ имеем включение $ T \subset f_*(R_{\sigma}(f^{-1}(T))) $.
		Снова по минимальности (теореме \ref{min ring exists}) имеем включение $ R_{\sigma}(T) \subset f_*(R_{\sigma}(f^{-1}(T))) $.
		
		По лемме \ref{inclusions direct-inverse} имеем включение $ f^*(f_*(R_{\sigma}(f^{-1}(T))) \subset R_{\sigma}(f^{-1}(T)) $.
		Собирая вместе все включения и пользуясь тем, что $ f^*(A) \subset f^*(B) $ для $ A \subset B $ получаем
		$$ f^*(R_{\sigma}(T)) \subset f^*(f_*(R_{\sigma}(f^{-1}(T))) \subset R_{\sigma}(f^{-1}(T)) \subset f^{*}(R_{\sigma}(T)), $$
		откуда следует требуемое.
	\end{proof}
	
	
	\subsubsection{Функтор борелевской $ \sigma $-алгебры}
	
	Пусть $ (X, \tau) $ --- топологическое пространство.
	Минимальная $ \sigma $-алгебра, порождённая системой открытых множеств $ \tau $ 
	называется \defin{борелевской $ \sigma $-алгеброй}{borel-sigma-algebra},
	а её элементы называются \defin{борелевскими множествами}{borel-sets}.
	Мы будем обозначать её через $ \calB(\tau) $, а соответствующее измеримое пространство
	через $ \Bor((X, \tau)) = (X, \calB(\tau)) $.
	Мы докажем, что конструкция $ \Bor \colon \Top \to \Meas $, сопоставляющая топологическому пространству $ (X, \tau) $
	измеримое пространство $ (X, \calB(\tau)) $, а непрерывному отображению $ f $
	его же как отображение множеств, функториальна.
	
	\begin{lemma} \label{measurability of map}
		Пусть $ f \colon X \to Y $ --- отображение множеств,
		$ \theta $ --- топология $ Y $, $ \calA_X $ --- $ \sigma $-алгебра подмножеств $ X $.
		Пусть также $ S $ --- база топологии $ \theta $ такая, 
		что всякое открытое подмножество представляется в виде не более, чем счётного объединения элементов базы, 
		и $ T $ --- предбаза топологии $ \theta $ такая, что всякое открытое множество представляется в виде не
		более, чем счётного объединения конечных пересечений элементов $ T $.
		Тогда следующие утверждения равносильны
		\begin{enumerate}
			\item прообраз всякого элемента борелевской $ \sigma $-алгебры $ \calB(\theta) $ лежит в $ \calA_X; $ \label{measurability of map | all}
			\item прообраз всякого открытого множества лежит в $ \calA_X; $ \label{measurability of map | open}
			\item прообраз всякого элемента базы $ S $ лежит в $ \calA_X; $ \label{measurability of map | base}
			\item прообраз всякого элемента предбазы $ T $ лежит в $ \calA_X. $ \label{measurability of map | prebase}
		\end{enumerate}
	\end{lemma}
	
	\begin{proof}
		Все пункты являются частным случаем пункта \ref{measurability of map | all},
		а пункты \ref{measurability of map | base} и \ref{measurability of map | prebase}
		--- пункта \ref{measurability of map | open}.
		Так как база топологии (с данным дополнительным условие) 
		является частным случаем предбазы топологии (с дополнительным условием),
		то достаточно вывести из пункта $ \ref{measurability of map | prebase} $
		пункт \ref{measurability of map | all}.
		
		Рассмотрим систему множеств $ T_X = f^{-1}(T) \cup \{X\} := \defineset{f^{-1}(U)}{U \in T} \cup \{X\} $.
		По условию $ T_X \subset \calA_X $. Пусть $ R_{\sigma}(T_X) $ --- минимальная $ \sigma $-алгебра, содержащая $ T_X $.
		Так как $ \calA_X $ является $ \sigma $-алгеброй, то $ R_{\sigma}(T_X) \subset \calA_X $.
		Из построения $ f_* $ имеем $ f_*(R_{\sigma}(T_X)) \subset f_*(\calA_X) $.
		Так же из построения $ f_* $ имеем включение $ T \subset f_*(R_{\sigma}(T_X)) $.
		Из условия наложенного на $ T $ следует, что минимальная $ \sigma $-алгебра, порождённая $ T $
		совпадает с $ \calB(\theta) $. Тогда по минимальности (теореме \ref{min ring exists})
		имеем включение $ \calB(\theta) \subset f_*(R_{\sigma}(T_X)) $.
		Следовательно, $ \calB(\theta) \subset f_{*}(\calA_X) $ и по предложению \ref{direct image of sigma algebra}
		$ f $ является морфизмом измеримых пространств $ f \colon (X, \calA_X) \to (Y, \calB(\theta)) $,
		что и утверждается в пункте \ref{measurability of map | all}.
	\end{proof}
	
	\begin{theorem} \label{borel functor}
		Пусть $ (X, \tau), (Y, \theta) $ --- топологические пространства,
		$ f \colon (X, \tau) \to (Y, \theta) $ --- непрерывное отображение.
		Тогда отображение $ f $ является морфизмом измеримых пространств
		$ f \colon (X, \calB(\tau)) \to (Y, \calB(\theta)) $.
	\end{theorem}
	
	\begin{proof}
		Следует из эквивалентности пунктов \ref{measurability of map | all}
		и $ \ref{measurability of map | open} $ леммы \ref{measurability of map} 
		для случая, когда $ \calA_X = \calB(\tau) $.
	\end{proof}
	
	\subsection{Предварительные сведения из анализа Фурье}
	
	\subsection{Предварительные сведения из линейной алгебры}
	
	\subsubsection{Билинейные функции и квадратичные формы}
	
	Пусть $ \Bbbk $ --- некоторое поле (в нашем случае будут рассматриваться только поля вещественных чисел $ \RR $) и $ V $ --- векторное пространство над $ \Bbbk $.
	
	Отображение $ B \colon V \times V \to \Bbbk $ называется \defin{билинейной функцией}{bilinear}, если выполнены следующие аксиомы
	\begin{enumerate}
		\item $ \forall v,u,w \in V \ B(u + v, w) = B(u, w) + B(v, w) $;
		\item $ \forall v,u \in V, \lambda \in \Bbbk \ B(\lambda u, v) = \lambda B(u, v) $;
		\item $ \forall v,u,w \in V \ B(u, v + w) = B(u, w) + B(u, v) $;
		\item $ \forall v,u \in V, \lambda \in \Bbbk \ B(u, \lambda v) = \lambda B(u, v) $.
	\end{enumerate}
	
	Билинейная функция называется \defin{симметрической}{symmetric}, если дополнительно для любых $ u, v \in V $ выполнено $ B(u, v) = B(v, u) $.
	
	\begin{example}
		Пусть $ V = \Bbbk $ и $ B(a, b) = a \cdot b $, где $ \cdot $ --- умножение в поле $ \Bbbk $.
		Тогда $ B $ --- симметрическая билинейная функция.
	\end{example}
	
	\begin{example}
		Пусть в векторном пространстве $ V $
		фиксирован базис $ e_1, \ldots, e_n $. Тогда если $ B(x, y) = \sum\limits_{i = 1}^{n} x_iy_i $, где $ x = \sum\limits_{i = 1}^{n} x_ie_i $ и $ y = \sum\limits_{i = 1}^{n} y_ie_i $, то $ B $ --- также билинейная симметрическая форма.
	\end{example}
	
	\defin{Квадратичной формой}{quadratic} называется отображение $ Q \colon V \to \Bbbk $ такое, что для некоторой билинейной формы и любой вектора $ v \in V $ имеет место равенство $ Q(v) = B(v, v) $.
	Если $ B $ --- билинейная функция, то квадратичная форма $ Q $, заданная формулой $ Q(v) = B(v, v) $
	называется квадратичной формой соответствующей билинейной функции $ B $.
	Пусть $ \Bbbk = \RR $, $ Q $ --- квадратичная форма 
	и для любого ненулевого вектора $ v \in V $ выполнено неравенство $ Q(v) > 0 $.
	Тогда форма $ Q $ называется положительно определённой. 
	Если для любого $ v \in V $ выполнено неравенство $ Q(v) \geqslant 0 $,
	то форма $ Q $ называется неотрицательно определённой.

	Симметрическую билинейную форму с положительно определённой соответствующей квадратичной формой называют 
	\defin{скалярным произведением}{inner-product}. Вместо $ B(u,v) $ часто пишут $ (u, v) $ или $ \left<u, v\right> $.
	
	Примеры. Квадратичные формы, соответствующие билинейным функциям из примеров выше являются положительно определёнными.
	
	\begin{theorem}[Коши, Буняковский, Шварц] \label{Cauchy-real}
		Пусть $ V $ --- векторное пространство над полем $ \RR $ и $ B $ --- скалярное произведение на $ V $.
		Тогда дл любых двух векторов $ u, v \in V $ выполнено равенство
		$$ B(u, v)^2 \leqslant B(u,u)B(v,v), $$
		причём равенство достигается тогда и только тогда, когда $ u $ и $ v $ коллинеарны.
	\end{theorem}
	
	\begin{proof}
		Рассмотрим вектор $ u + tv $, где $ t \in \RR $ и значение квадратичной формы на нём.
		По билинейности, симметричности и положительной определённости имеем 
		$$ B(u + tv, u + tv) = B(u, u) + tB(u, v) + tB(v, u) + t^2B(v,v) = B(u,u) + 2tB(u,v) + t^2B(v,v) \geqslant 0, $$
		причём последнее равенство достигается тогда и только тогда, когда $ u + tv = 0 $.
		
		Многочлен второй степени принимает только неотрицательные (положительные) значения тогда и только тогда, когда его дискриминант меньше или равен 0 (меньше 0).
		Итого $$ D = 4B(u,v)^2 - 4B(u,u)B(v,v) \leqslant 0  \Leftrightarrow B(u,v)^2 \leqslant B(u,u)B(v,v) $$
		и $ D = 0 \Leftrightarrow B(u,v)^2 = B(u,u)B(v,v) $. Последнее равносильно тому, что многочлен имеет корень $ t $
		и $ u + tv = 0 $, то есть $ u $ и $ v $ пропорциональны.
	\end{proof}
	
	Заметьте, что доказательство этого неравенства в случае поля комплексных чисел требует добавления дополнительной <<поправки>> $ \lambda $.
	
	\subsubsection{Полуторалинейные функции}
	
	В этом подразделе будем рассматривать только векторные пространства над полем комплексных чисел.
	
	Отображение $ S \colon V \times V \to \Bbbk $ называется \defin{полуторалинейной функцией (по второму аргументу)}{sesquilinear}, если выполнены следующие аксиомы
	\begin{enumerate}
		\item $ \forall v,u,w \in V \ S(u + v, w) = S(u, w) + S(v, w) $;
		\item $ \forall v,u \in V, \lambda \in \Bbbk \ S(\lambda u, v) = \lambda S(u, v) $;
		\item $ \forall v,u,w \in V \ S(u, v + w) = S(u, w) + S(u, v) $;
		\item $ \forall v,u \in V, \lambda \in \Bbbk \ S(u, \lambda v) = \overline{\lambda} S(u, v) $,
		где надчёркивание означает комплексное сопряжение.
	\end{enumerate}
	
	Полуторалинейная функция называется \defin{эрмитовой}{hermitian}, если для любых векторов $ u $ и $ v $ дополнительно выполнено равенство
	$ S(u, v) = \overline{S(v, u)} $.
	
	Эрмитова функция называется \defin{скалярным произведением}{C-inner-product}, если для любого ненулевого вектора $ v $
	выполнено неравенство $ S(v, v) > 0 $.
	
	
	\begin{theorem}[Коши, Буняковский, Шварц] \label{Cauchy-complex}
		Пусть $ V $ --- векторное пространство над полем $ \CC $ и $ S $ --- скалярное произведение на $ V $.
		Тогда для любых двух векторов $ u, v \in V $ выполнено равенство
		$$ S(u, v)\overline{S(u,v)} \leqslant S(u,u)S(v,v), $$
		причём равенство достигается тогда и только тогда, когда $ u $ и $ v $ коллинеарны.
	\end{theorem}
	
	\begin{proof}
		Если $ S(u, v) = 0 $, то неравенство выполнено. При таком условии $ u $ и $ v $ пропорциональны тогда и только тогда,
		когда один из этих векторов равен 0. Последнее в свою очередь равносильно тому, что правая часть неравенства обращается в нуль. Далее будем считать, что $ S(u,v) \neq 0 $.
		
		Рассмотрим вектор $ u + t\lambda v $, где $ t \in \RR $ и $ \lambda = S(u,v) $.
		Поскольку $ S $ --- скалярное произведение и из условий наложенных на $ 
		\lambda $, то 
		$$ S(u + t\lambda v, u + t\lambda v) 
		= S(u, u) + t\overline{\lambda}S(u, v) + t\lambda S(v, u) + t^2\lambda\overline{\lambda}S(v,v) = $$ 
		$$ = S(u,u) + 2tS(u,v)S(v,u) + t^2S(u,v)S(v,u)S(v,v) \leqslant 0 $$
		причём последнее равенство достигается тогда и только тогда, когда $ u + t\lambda v = 0 $.
		
		Многочлен второй степени принимает только неотрицательные (положительные) значения тогда и только тогда, когда его дискриминант меньше или равен 0 (меньше 0).
		Итого $$ D = 4S(u,v)^2S(v,u)^2 - 4S(u,u)S(v,v)S(u,v)S(v,u) \leqslant 0  \Leftrightarrow S(u,v)S(v,u) \leqslant S(u,u)S(v,v) $$
		и $ D = 0 \Leftrightarrow S(u,v)^2 = S(u,u)S(v,v) $. Последнее равносильно тому, что многочлен имеет корень $ t_0 $
		и $ u + t_0S(u, v)v = 0 $, то есть $ u $ и $ v $ пропорциональны.
	\end{proof}
	
	\section{Элементарная комбинаторика}
	
	\subsection{Классические комбинаторные величины}
	
	Существует ровно $ \tfrac{n!}{k!(n - k)!} =: C_n^k $ кортежей длины $ n $, содержащих ровно $ k $ единиц (<<можно расставить эти $ k $ единицы на данные $ n $ позиций ровно $ \tfrac{n!}{k!(n - k)!} =: C_n^k $ способами>>).
	Действительно, выстроим данные нам $ k $ единицы и $ n - k $ нулей в ряд и переставим его члены.
	Имеется всего $ n! $ перестановок всех членов. При этом, дополнительные перестановки внутри набора из единиц
	и внутри набора из нулей не изменяют их конечной расстановки. Это значит, что все $ n! $ перестановок
	разбиваются на $ k!(n - k)! $ групп одинаковых, а общее их количество равно $ \tfrac{n!}{(n - k)!k!} $.
	
	\subsection{Свойства комбинаторных величин}
	
	\section{Вероятностное пространство, случайные события}
	
	Пусть $ \Omega $ --- некоторое множество, $ \mathfrak{F} $ --- $ \sigma $-алгебра с единицей $ \Omega $
	и $ \prob $ --- $ \sigma $-аддитивная мера на $ \mathfrak{F} $, удовлетворяющая свойству $ \prob(\Omega) = 1 $. 
	Тогда тройка $ (\Omega, \mathfrak{F}, \prob) $ называется \defin{вероятностным пространством}{prob-space}.
	Множество $ \Omega $ называется \defin{пространством элементарных событий (исходов)}{space},
	элементы $ \sigma $-алгебры $ \mathfrak{F} $ называются \defin{событиями}{event},
	а мера $ \prob $ \defin{вероятностной мерой}{probability-measure}.
	
	Иногда мы будем называть вероятностное пространство <<экспериментом>> или <<испытанием>>
	или говорить, что <<эксперименту>> или <<испытанию>> соответствует вероятностное пространство,
	выражая таким образом физический смысл этого понятия: проходит эксперимент (испытание), 
	у которого есть различные элементарные исходы. 
	Эти исходы могут в результате этого эксперимента в разных комбинациях возникнуть с разной вероятностью.
	
	Вероятностное пространство называется \defin{дискретным}{discr}, если множество $ \Omega $ не более, чем счётно.
	
	Для кратности, если множество $ \{\omega\} $ является событием, вместо $ \prob({\omega}) $ будем писать $ \prob(\omega) $.
	
	\begin{example}
		Пусть $ \Omega = [0, 1] $ и $ \events = M $ --- $ \sigma $-алгебра измеримых относительно меры Лебега
		подмножеств отрезка $ [0, 1] $, $ \prob = \mu $ --- классическая мера Лебега.
		Тогда $ \prob((\tfrac{1}{2}, \tfrac{3}{4})) = \tfrac{1}{4} $, $ \prob(\tfrac{2}{9}) = 0 $ 
		и $ \prob((0,\tfrac{1}{2}) \cup (\tfrac{2}{3}, 1)) = \tfrac{5}{6} $.
	\end{example}
	
	\begin{example}
		Пусть теперь $ \Omega = [0, 2] $ и $ \events = M $ --- $ \sigma $-алгебра измеримых относительно меры Лебега
		подмножеств отрезка $ [0, 2] $, $ \prob = \tfrac{1}{2}\mu $ --- мера, пропорциональная мере Лебега (мы выбрали имеено такую меру, чтобы удовлетворить условию $ \prob(\Omega) = 1 $).
		Тогда $ \prob((0, 1)) = \tfrac{1}{2} $, $ \prob(\tfrac{4}{5}) = 0 $ и $ \prob((0,\tfrac{1}{2}) \cup (1, \tfrac{4}{3})) = \tfrac{5}{12} $.
	\end{example}
	
	\begin{example}
		Пусть $ \Omega = \{1,2,3,4,5,6\} $ --- числа, возникающие при броске игральной кости.
		Будем считать, что все элементарные исходы равновероятны, 
		то есть $ \prob(1) = \prob(2) = \prob(3) = \prob(4) = \prob(5) = \prob(6) = \tfrac{1}{6} $.
		Тогда вероятность события $ A = \{2,4,6\} $ --- <<выпало чётное число>> 
		равна $ \prob(A) = \prob(2) + \prob(4) + \prob(6) = \tfrac{1}{6} + \tfrac{1}{6} + \tfrac{1}{6} = \tfrac{1}{2} $.
	\end{example}
	
	Рассмотренный пример мотивирует нас ввести параллельные определения для дискретного пространства.
	\defin{Дискретным вероятностным пространством}{discr-2} мы будем называть  пару $ (\Omega, \prob) $, 
	где $ \Omega = \{\omega_k\}_{k \in \mathbb{N}} $ --- не более чем счётное множество 
	(также называемое \defin{пространством элементарных исходов}{space-discr}),
	а $ \prob \colon \Omega \to \RR $ --- неотрицательная функция, удовлетворяющая свойству
	$ \sum\limits_{k \in \mathbb{N}} \prob(\omega_k) = 1 $.	
	Говорят, что в этом случае на $ \Omega $ \defin{заданы вероятности элементарных событий}{prob-defined} и что функция $ \prob $ 
	\defin{задаёт на $ \Omega $ распределение вероятностей}{disrtib-discr}.
	\defin{Событиями}{event-discr} называются подмножества $ \Omega $. 
	\defin{Вероятностью события}{prob-discr} $ A \subset \Omega $ называется величина
	$$ \prob(A) = \sum\limits_{\omega \in A} P(\omega), $$
	которую мы также будем обозначать буквой $ \prob $. 
	Последнее данное определение корректно, поскольку ряд в правой части сходится абсолютно.
	
	\begin{proposition} \label{}
		Пусть $ (\Omega, \prob) $ --- дискретное вероятностное пространство в смысле \hyperlink{discr-2}{последнего определения}.
		Пусть $ \prob \colon 2^{\Omega} \to \mathbb{R} $ --- функция, сопоставляющая событию его вероятность.
		Тогда тройка $ (\Omega, 2^{\Omega}, \prob) $ является вероятностным пространством в смысле \hyperlink{prob-space}{исходного определения}.
	\end{proposition}
	
	\begin{proof}
		
		Множество $ 2^{\Omega} $ является $ \sigma $-алгеброй, поэтому достаточно проверить, что функция $ \prob $
		удовлетворяет аксиомам вероятностной меры.
		
		Из определения $ \prob $ имеем
		$$ \prob(\Omega) = \sum\limits_{i = 1}^{+\infty} \prob(\omega_i) = 1. $$
		Пусть $ A, B \subset \Omega $ и $ A \cap B = \varnothing $.
		Положим $ A = \{\omega_{i}\}_{i \in I_A} $, $ B = \{\omega_{i}\}_{i \in I_B} $ и 
		$ A \sqcup B = \{\omega_i\}_{i \in I_{A \sqcup B}} $. 
		Поскольку $ A $ и $ B $ не пересекаются, то $ I_A \sqcup I_B =  I_{A \sqcup B} $.
		Тогда, так как ряды в формуле ниже сходятся абсолютно, имеем
		$$ \prob(A \sqcup B) = \sum\limits_{i \in I_{A \sqcup B}} \omega_i = 
		\sum\limits_{i \in I_A} \omega_{i} + \sum\limits_{i \in I_B} \omega_{i}
		= \prob(A) + \prob(B). $$
		
		Пусть теперь $ \{A_k\}_{k \in \mathbb{N}} $ --- счётное семейство непересекающихся подмножеств множества $ \Omega $.
		Положим $ A_k = \{\omega_{i}\}_{i \in I_k} $, $ A = \bigsqcup\limits_{k \in I} A_k $.
		Снова, поскольку $ A_k $ попарно не пересекаются, то $ I = \bigsqcup\limits_{k \in \mathbb{N}} I_k $.
		Поскольку все ряды ниже сходятся абсолютно, то выполнены равенства
		$$ \prob(A) = \sum\limits_{i \in I} \prob(\omega_i) 
		= \sum\limits_{k \in \mathbb{N}} \sum\limits_{i \in I_k} \prob(\omega_i) 
		= \sum\limits_{k \in \mathbb{N}} \prob(A_k). $$
	\end{proof}
	
	Пусть $ A, B \in \events $ --- события. Введём основные операции над событиями 
	и приведём их классические наименования и обозначения в теории вероятностей. 
	
	Событие $ \Omega \setminus A $ называется
	\defin{дополнением к событию $ A $}{event-compl}
	и обозначается $ \overline{A} $ (<<событие $ A $ не произошло>>).
	
	Событие $ A \cup B $ называется
	\defin{суммой событий $ A $ и $ B $}{event-sum}
	и обозначается $ A + B $  (<<произошло событие $ A $ или $ B $>>). 
	В курсе лекций это обозначение использовалось для случаев, когда $ A \cap B = \varnothing $.
	
	Событие $ A \cap B $ называется 
	\defin{произведением событий $ A $ и $ B $}{event-product} 
	и обозначается $ AB $ (<<произошло и событие $ A $ и событие $ B $>>).
	
	События $ \Omega $ и $ \varnothing $ называются \defin{достоверным}{} и \defin{невозможным}{}, соответственно.
	
	Если $ AB = \varnothing $, то события $ A $ и $ B $ называются \defin{несовместными}{}.
	(<<события $ A $ и $ B $ не происходят одновременно>>).
	
	\begin{proposition}[Начальные свойства вероятностной меры]
		Пусть $ A, B, A_k \in \events $ --- события.
		Тогда имеет место следующее:
		\begin{enumerate}
			\item $ \prob(\overline{A}) = 1 - \prob(A); $ \label{prob-prop1}
			\item если $ A \subset B $, то $ \prob(B \setminus A) = \prob(B) - \prob(A); $ \label{prob-prop2}
			\item если $ A \subset B $, то $ \prob(A) \leqslant \prob(B); $ \label{prob-prop2.5}
			\item $ \prob(A \cup B) = \prob(A) + \prob(B) - \prob(AB); $ \label{prob-prop3}
			\item $ \prob(A \cup B) \leqslant \prob(A) + \prob(B) $; \label{prob-prop4}
			\item $ \prob(\bigcup\limits_{k = 1}^{n} A_k) = \sum\limits_{k = 1}^{n} 
			\left(\sum\limits_{i_1 < i_2 < \ldots < i_k} (-1)^{k - 1}\prob(A_{i_1} \ldots A_{i_k})\right); $ \label{prob-prop5}
			\item $ \prob\left(\bigcup\limits_{k = 1}^{+\infty} A_k \right) \leqslant \sum\limits_{k = 1}^{+\infty} \prob(A_k) $ 
			(это свойство называется \defin{субаддитивностью}{subadditivity}). \label{prob-prop6}
		\end{enumerate}
	\end{proposition}
	
	\begin{proof}
		Равенство \ref{prob-prop1} следует из цепочки 
		$$ 1 = \prob(\Omega) = \prob(A \sqcup \overline{A}) = \prob(A) + \prob(\overline{A}). $$
		
		Равенство \ref{prob-prop2} --- из цепочки 
		$$ \prob(B) = \prob(A \cup (B \setminus A)) = \prob(A) + \prob(B \setminus A). $$
		Неравенство \ref{prob-prop2.5} следует из этого равенства и неотрицательности вероятности.
		
		Равенство \ref{prob-prop3} --- из цепочки
		$$ \prob(A \cup B) = \prob((A \setminus B) \sqcup (A \cap B) \sqcup (B \setminus A)) = $$ 
		$$ = \prob(A \setminus B) + \prob(A \cap B) + \prob(B \setminus A) + \prob(A \cap B) - \prob(A \cap B) = $$
		$$ = \prob((A \setminus B) \sqcup (A \cap B)) + \prob((B \setminus A) \sqcup (A \cap B)) - \prob(A \cap B) = $$
		$$ = \prob(A) + \prob(B) - \prob (A \cap B). $$
		Неравенство \ref{prob-prop4} немедленно следует из равенства \ref{prob-prop3}.
		
		Докажем \ref{prob-prop5} по индукции.
		
		База $ n = 2 $ была доказана в пункте 3.
		
		Докажем шаг.
		Положим $ B = \bigcup\limits_{k = 1}^{n - 1} A_k $.
		По базе индукции $$ \prob(B \cup A_{n}) = \prob(B) + \prob(A_n) - \prob(B A_n). $$
		Далее, положим $ B_k = A_kA_n $. Тогда $ BA_n = \bigcup\limits_{k = 1}^{n - 1} B_k $.
		По индукционному предположению вероятность $ \prob(B \cup A_n) $ равна
		$$ \sum\limits_{k = 1}^{n - 1} 
		\left(\sum\limits_{i_1 < i_2 < \ldots < i_k} (-1)^{k - 1}\prob(A_{i_1} \ldots A_{i_k})\right)
		+ \prob(A_n)
		- \sum\limits_{k = 1}^{n - 1} 
		\left(\sum\limits_{i_1 < i_2 < \ldots < i_k} (-1)^{k - 1}\prob(A_{i_1} \ldots A_{i_k} A_{n})\right) = $$
		$$ = \sum\limits_{k = 1}^{n} 
		\left(\sum\limits_{i_1 < i_2 < \ldots < i_k} (-1)^{k - 1}\prob(A_{i_1} \ldots A_{i_k})\right). $$
		
		Докажем неравенство $ \ref{prob-prop6} $.
		Положим $ B_k = A_k \setminus \bigcup\limits_{i = 1}^{k - 1} A_i $.
		Тогда $ \bigcup\limits_{k = 1}^{+\infty} B_k = \bigcup\limits_{k = 1}^{+\infty} A_k $,
		причём  $ B_k $ попарно не пересекаются и $ \prob(B_k) \leqslant \prob(A_k) $ по $ \ref{prob-prop2} $.
		Тогда по $ \sigma $-аддитивности имеем 
		$$ \bigcup\limits_{k = 1}^{+\infty} A_k = \bigcup\limits_{k = 1}^{+\infty} B_k 
		= \sum\limits_{k = 1}^{+\infty} \prob(B_k) \leqslant \sum\limits_{k = 1}^{+\infty} \prob(A_k). $$
	\end{proof}
	
	\section{Условные вероятности, формула Байеса, независимость событий}
	
	\subsection{Условная вероятность}
	
	В задачах бывает полезно рассмотреть вероятность того, что произойдёт некоторое событие $ B $ при условии, 
	что произойдёт событие $ A $. Пусть $ \prob(A) > 0 $. Тогда вероятность $ \prob(B \mid A) = \tfrac{\prob(AB)}{\prob(A)} $
	называется \defin{условной вероятностью события $ B $ при условии того, 
	что событие $ A $ произойдёт с вероятностью $ \prob(A) > 0 $}{conditional}.
	Вероятность $ \prob(B) $ также иногда называется \defin{априорной вероятностью события $ B $}{apriori}.
		
	\begin{proposition}
		Пусть $ (\Omega, \events, \prob) $ --- вероятностное пространство.
		Пусть $ {A \in \events} $ --- событие, удовлетворяющее условию $ \prob(A) > 0 $.
		Тогда тройка $$ (\Omega, \events, \left.\prob\right|_{A}), $$
		где $ \left.\prob\right|_{A}(B) := \prob(B \mid A) = \tfrac{\prob(AB)}{\prob(A)} $,
		является вероятностным пространством.
	\end{proposition}
	
	\begin{proof}
		Достаточно проверить аксиомы вероятностной меры 
		(аксиомы $ \sigma $-аддитивной меры и равенство $ \left.\prob\right|_{A}(\Omega) = 1 $).
		
		Так как обе величины $ \prob(AB) $ и $ \prob(A) $ неотрицательны (а последняя и вовсе положительна),
		то $ \prob(B \mid A) \leqslant 0 $.
		
		Справедливость упомянутого равенства выводится из определения условной вероятности: 
		$$ \left.\prob\right|_{A}(\Omega) = \tfrac{\prob(A\Omega)}{\prob(A)} = \tfrac{\prob(A)}{\prob(A)} = 1. $$
		
		Пусть $ \{B_k\}_{k \in \mathbb{N}} $ --- 
		счётная последовательность попарно не пересекающихся элементов алгебры $ \events $. 
		Тогда элементы последовательности $ \{B_k \cap A\}_{k \in \mathbb{N}} $ также попарно не пересекаются.
		Тогда
		$$ \left.\prob\right|_{A}\left(\bigsqcup\limits_{k = 1}^{+\infty} B_k \right)
		= \tfrac{1}{\prob(A)}\prob\left(A \cap \bigsqcup\limits_{k = 1}^{+\infty} B_k \right)
		= \tfrac{1}{\prob(A)}\prob\left(\bigsqcup\limits_{k = 1}^{+\infty} AB_k \right)
		= \sum\limits_{k = 1}^{+\infty} \tfrac{\prob(AB_k)}{\prob(A)} 
		= \sum\limits_{k = 1}^{+\infty} \left.\prob\right|_{A}(B_k). $$
	\end{proof}
	
	\begin{corollary}
		Пусть $ A \in \events $ --- событие, вероятность которого больше $ 0 $, $ B_1, B_2 \in \events $.
		Тогда справедливы следующие свойства
		\begin{enumerate}
			\item если $ B_1 \supset A $, то $ \prob(B_1 \mid A) = 1; $
			\item $ \prob(B_1 \cup B_2 \mid A) = \prob(B_1 \mid A) + \prob(B_2 \mid A) - \prob(B_1B_2 \mid A); $
			\item если $ B_1 $ и $ B_2 $ несовместны, то $ \prob(B_1 + B_2 \mid A) = \prob(B_1 \mid A) + \prob(B_2 \mid A). $
		\end{enumerate}
	\end{corollary}
	
	\subsection{Формула полной вероятности и формула Байеса}
	
	Теперь мы покажем, как связаны условные вероятности с вероятностями произведений событий, 
	как можно вычислять вероятность события, зная его условные вероятности для несовместных событий (формула полной вероятности)
	и как можно вычислить условную вероятность <<с переставленными причиной и следствием>> (формула Байеса).
	
	\begin{lemma} \label{cond prob}
		Пусть $ A, B \in \events $ --- события и $ \prob(A), \prob(B) > 0 $.
		Тогда имеют место равенства
		$$ \prob(AB) = \prob(A \mid B)\prob(B) = \prob(B \mid A)\prob(A), $$
		$$ \prob(A \mid B) = \tfrac{\prob(B \mid A)\prob(A)}{\prob(B)}. $$
	\end{lemma}
	
	\begin{proof}
		Первое равенство немедленно следует из определения условной вероятности, второе -- немедленно из первого
		и предположения, что $ \prob(B) > 0 $.
	\end{proof}
	
	Второе равенство, доказанное в лемме иногда (особенно в школьных программах), называют формулой Байеса.
	Ниже, пользуясь этим простым свойством, мы докажем более общую формулу и в дальнейшем будем называть формулой Байеса её.
	
	\begin{theorem}[Формула произведения вероятностей] \label{multiplication law}
		Пусть $ A_1, \ldots, A_n \in \events $ --- события.
		Если вероятности событий $ A_2A_3\ldots A_n, \ldots, A_{n - 1}A_n, A_n $ не равны нулю, то имеет место формула
		$$ \prob(A_1A_2\ldots A_n) = \prob(A_1 \mid A_2 \ldots A_n)\prob(A_2 \mid A_3 \ldots A_n) 
		\ldots \prob(A_{n - 1}\mid A_n)\prob(A_n). $$
		Если вероятности событий $ A_1A_2\ldots A_n, A_1A_2\ldots A_{n - 1}, \ldots, A_1 $ не равны нулю, то имеет место формула
		$$ \prob(A_1A_2\ldots A_n) = \prob(A_n \mid A_1\ldots A_{n-1})\prob(A_{n - 1} \mid A_1 \ldots A_{n - 2}) 
		\ldots \prob(A_{2}\mid A_1)\prob(A_1). $$
	\end{theorem}
	
	\begin{proof}
		Докажем первое утверждение индукцией по $ n $, второе получается из первого перестановкой индексов в обратном порядке.
		
		База: $ n = 2 $ есть определение условной вероятности.
		
		Докажем шаг индукции. Пусть для $ n - 1 $ утверждение выполнено.
		Положим $ B = A_2A_3\ldots A_n $.
		Тогда по базе индукции (здесь мы пользуемся тем, что $ \prob(B) > 0 $) 
		и затем по индукционному предположению (а здесь всеми остальными условиями) имеем
		$$ \prob(A_1B) = \prob(A_1 \mid B)\prob (B) = \prob(A_1 \mid A_2A_3 \ldots A_n)\prob(A_2 \mid A_3 \ldots A_n)
		\ldots \prob(A_{n - 1}\mid A_n)\prob(A_n). $$
	\end{proof}
	
	Набор событий $ A_1, \ldots, A_n \in \events $ называется \defin{разбиением пространства $ \Omega $}{exclusive-exhaustive}
	(или просто <<разбиение $ \Omega $>>),
	если $ \prob(A_i) > 0 $ для каждого $ i $, $ A_i $ попарно несовместны ($ A_iA_j = \varnothing $ при $ i \neq j $)
	и $ {A_1 + A_2 + \ldots + A_n = \Omega} $.
	
	\begin{theorem}[Формула полной вероятности] \label{law of total probability}
		Пусть $ A_1, \ldots, A_n \in \events $ --- разбиение $ \Omega $. 
		Тогда для всякого события $ B $ имеет место равенство формула полной вероятности
		$$ \prob(B) = \sum\limits_{i = k}^{n} \prob(B \mid A_k)\prob(A_k). $$
	\end{theorem}
	
	\begin{proof}
		Так как события $ A_k $ попарно несовместны, то события $ A_kB $ также попарно несовместны.
		Имеем цепочку равенств
		$$ \prob(B) = \prob(B\Omega) = \prob(B(A_1 + \ldots + A_n))
		= \prob(BA_1 + \ldots + BA_n) \overset{\text{несовместность}}{=} $$ 
	    $$ \overset{\text{несовместность}}{=} \sum\limits_{k = 1}^{n} \prob(BA_k)
		= \sum\limits_{k = 1}^{n} \prob(B\mid A_k)\prob(A_k). $$
	\end{proof}
	
	Формула полной вероятности остаётся справедливой, если отказаться от требования $ A_1 + A_2 + \ldots + A_n = \Omega $
	и заменить его на условие $ B \subset A_1 + \ldots + A_n $ (сохраняя требования попарной несовместности событий $ A_i $ 
	и $ \prob(A_i) > 0 $).
	
	\begin{theorem}[Формула Байеса] \label{Bayes}
		Пусть события $ A_1, \ldots, A_n \in \events $ образуют разбиение $ \Omega $,
		пусть $ B \in \events $ --- ещё одно событие и $ \prob(B) > 0 $.
		Тогда справедлива формула Байеса
		$$ \prob(A_i \mid B) = \tfrac{\prob(B \mid A_i)\prob(A_i)}{\sum\limits_{k = 1}^{n} \prob(B \mid A_k)\prob(A_k)}. $$
	\end{theorem}
	
	\begin{proof}
		По лемме \ref{cond prob} (<<простейшая формула Байеса>>) имеем равенство
		$$ \prob(A_i \mid B) = \tfrac{\prob(B \mid A_i)\prob(B)}{\prob(B)}. $$
		По формуле полной вероятности имеем
		$$ \prob(B) = \sum\limits_{i = k}^{n} \prob(B \mid A_k)\prob(A_k), $$
		откуда следует искомая формула.
	\end{proof}
	
	\begin{example}
		Приведём стандартный пример на применение простейшего вида формулы Байеса.
		Пусть в популяции заболевание встречается с вероятностью \( \prob(\text{Б}) = 0.1 \).  
		ПЦР-тест на выявление заболевания устроен так, что:
		\begin{itemize}
			\item При наличии заболевания он даёт положительный результат с вероятностью \( \prob(+ \mid \text{Б}) = 0.9 \),
			\item При отсутствии заболевания он даёт ложноположительный результат с вероятностью 
			\( \prob(+ \mid \text{З}) = 0.2 \).
		\end{itemize}
		Найдём вероятность того, что человек действительно болен, если результат теста оказался положительным.
		По формуле Байеса:
		$$ \prob(\text{Б} \mid +) 
		= \frac{\prob(+ \mid \text{Б})\cdot \prob(\text{Б})}
		{\prob(+ \mid \text{Б})\cdot \prob(\text{Б}) + \prob(+ \mid \text{З})\cdot \prob(\text{З})}.
		$$
		Подставим известные значения:
		$$
		 \prob(\text{Б} \mid +) = \frac{0.99 \cdot 0.01}
		 {0.99 \cdot 0.01 + 0.05 \cdot 0.99} = \frac{0.0099}{0.0099 + 0.0495} = \frac{0.0099}{0.0594} \approx 0.1667.
		$$
		
		Формально мы рассматривали дискретное вероятностное пространство 
		$ \Omega = \{+\text{Б},-\text{З},+\text{З},-\text{Б}\} $ с четырьмя элементарными исходами,
		выражающими все возможные комбинации результата тестирования и реального состояния тестируемого человека.
		Здесь событие Б <<болен>> являлось объединением $ \{+\text{Б},-\text{Б} \} $,
		событие <<пцр-тест дал положительный результат>> --- объединением $ \{+\text{Б},+\text{З} \} $
		и так далее.
		\end{example}
	
	\subsection{Независимость событий}
	
	Интуиция говорит нам, что события $ A $ и $ B $ <<независимы>>, когда от того с какой вероятностью произойдёт событие $ A $
	не зависит вероятность того, что произойдёт событие $ B $ и наоборот.
	Математически это выражается формулами $ \prob(B \mid A) = \prob(B) $ и $ \prob(A \mid B) = \prob(A) $.
	Чтобы не ограничиваться случаями, когда вероятности событий больше 0, мы определим независимость
	следствием формул выше. События $ A $ и $ B $ называются \defin{независимыми}{independent}, если справедливо равенство
	$ \prob(AB) = \prob(A)\prob(B) $.
	
	\begin{proposition}[Свойства независимости] \label{indep-prop}
		Имеют место следующие утверждения:
		\begin{enumerate}
			\item если $ \prob(B) > 0 $, 
			то независимость $ A $ и $ B $ равносильна равенству $ \prob(A \mid B) = \prob(A) $; \label{indep-prop-1}
			\item если $ A $ и $ B $ независимы, то $ \overline{A} $ и $ B $ независимы; \label{indep-prop-2}
			\item если события $ B_1 $ и $ B_2 $ несовместны, $ A $ и $ B_1 $ независимы, а также $ A $ и $ B_2 $ независимы, 
			то $ A $ и $ B_1 + B_2 $ независимы. \label{indep-prop-3}
		\end{enumerate}
	\end{proposition}
	
	\begin{proof}
		Проверим $ \ref{indep-prop-1} $.
		Если $ \prob(B) > 0 $, то по независимости имеем 
		$$ \prob(A \mid B) = \tfrac{\prob(AB)}{\prob(B)} = \tfrac{\prob(A)\prob(B)}{\prob(B)} = \prob(A). $$
		Обратно, если $ \prob(A \mid B) = \prob(A) $, то $ \tfrac{\prob(AB)}{\prob(B)} = \prob(A) $, откуда
		$ \prob(AB) = \prob(A)\prob(B) $.
		
		Для доказательства $ \ref{indep-prop-2} $ выпишем цепочку равенств
		$$ \prob(\overline{A}B) = \prob((\Omega \setminus A)B) = \prob(B \setminus AB) 
		= \prob(B) - \prob(AB) \overset{\text{независимость}}{=} $$ 
		$$ \overset{\text{независимость}}{=} \prob(B) - \prob(A)\prob(B) = \prob(B)(1 - \prob(A)) = \prob(B)\prob(\overline{A}). $$
		
		Наконец, для доказательства $ \ref{indep-prop-3} $ заметим, что события $ B_1A $ и $ B_2A $ несовместны.
		Тогда
		$$ \prob((B_1 + B_2)A) = \prob(B_1A + B_2A) = \prob(B_1A) + \prob(B_2A) 
		= \prob(B_1)\prob(A) + \prob(B_2)\prob(A)= $$ 
		$$ = (\prob(B_1) + \prob(B_2))\prob(A) = \prob(B_1 + B_2)\prob(A). $$
	\end{proof}

	Теперь определим независимость для набора событий.
	Пусть $ B_1, \ldots, B_n \in \events $ --- события.
	Будем говорить, что они \defin{попарно независимы}{pairwise-independent}, если для всяких двух индексов $ i \neq j $ выполнено равенство
	$ \prob(B_iB_j) = \prob(B_i)\prob(B_j) $ (то есть $ B_i $ и $ B_j $ независимы). 
	Будем называть эти события \defin{независимыми}{global-independent}, если для всякого набора индексов
	$ i_1 < \ldots < i_k $ (здесь $ 2 \leqslant k \leqslant n $) имеет место равенство
	$$ \prob\left(\bigcap\limits_{s = 1}^{k} B_{i_s}\right) = \prod_{s = 1}^{k} \prob(B_{i_s}). $$
	
	\begin{proposition}
		Если события $ B_1, \ldots, B_n $ независимы, то они попарно независимы.
	\end{proposition}
	
	\begin{example}
		Вообще говоря из попарной независимости не следует независимость, что демонстрируется следующим примером.
		Рассмотрим тетраэдр, три грани которого покрашены в красный, зелёный и синий цвета, соответственно,
		а последняя разбита на три треугольника, покрашенных в те же цвета. 
		Пусть вероятности выпадения граней равны $ \tfrac{1}{4} $. покажем, что события <<выпала грань с цветом $ A $>>,
		где $ A $ --- цвет попарно независимы, но не являются таковыми в совокупности. 
		Формально ситуация выглядит следующим образом $ \Omega = \{\omega_{R}, \omega_{G}, \omega_{B}, \omega_{RGB}\} $
		--- элементарное событие --- выпала грань с данной раскраской.
		По условию $ \prob(\omega_{R}) = \prob(\omega_{G}) = \prob(\omega_{B}) = \prob(\omega_{RGB}) = \tfrac{1}{4} $.
		Обозначим через $ R = \{\omega_R, \omega_{RGB}\} $  $ (G = \{\omega_G, \omega_{RGB}\}, B = \{\omega_B, \omega_{RGB}\}) $
		события <<выпала грань с красным (зелёным, синим) цветом>>, соответственно.
		Тогда $ \prob(R) = \prob(G) = \prob(B) = \tfrac{1}{2} $,
		$ \prob(RG) = \prob(GB) = \prob(BR) = \tfrac{1}{4} = \prob(R)\prob(G) = \prob(G)\prob(B) = \prob(B)\prob(R) $,
		но $ \prob(RGB) = \prob(\omega_{RGB}) = \tfrac{1}{4} \neq \prob(R)\prob(G)\prob(B) = \tfrac{1}{8} $.
	\end{example}

	\begin{example}
		Пользуясь примером выше, можно показать, что условие несовместности в пункте $ \ref{indep-prop-3} $ 
		предложения \ref{indep-prop} нельзя опустить.
		Положим $ A = R $ и $ B_1 = G $ и $ B_2 = B $. Тогда $ \prob((B_1 \cup B_2)A) = \prob(\omega_{RGB}) = \tfrac{1}{4} $,
		но $ \prob(A) = \tfrac{1}{2} $, $ \prob(B_1 + B_2) = \tfrac{3}{4} $ и $ \tfrac{1}{2} \cdot \tfrac{3}{4} \neq \tfrac{1}{4} $.
		Таким образом, события $ B_1 + B_2 $ и $ A $ не являются независимыми.
	\end{example}
	
	\begin{example}
		Покажем, что из условия независимости нельзя удалить ни одно из равенств.
		Более того, мы докажем, что для всякого натурального $ n $ 
		и семейства наборов индексов $ S_{J} = \{(i_{1,j}, \ldots, i_{k_j,j})\}_{j \in J} $
		можно построить пример вероятностного пространства $ (\Omega, \events, \prob) $ 
		и событий $ A_1, \ldots, A_n \in \events $ 
		для которых множество наборов, на которых выполнены равенства
		$$ \prob\left(\bigcap\limits_{s = 1}^{k} A_{i_s}\right) = \prod_{s = 1}^{k} \prob(A_{i_s}) $$
		в точности совпадает с $ J $.
		
		Построим пример для дискретного вероятностного пространства.
		Положим $ \Omega = \defineset{(\varepsilon_1, \ldots, \varepsilon_n)}{\varepsilon_i \in \{0, 1\}} $
		--- множество всех кортежей из нулей и единиц длины $ n $,
		$ \prob((\varepsilon_1, \ldots, \varepsilon_n)) = p_{(\varepsilon_1, \ldots, \varepsilon_n)} $
		--- будущее распределение вероятностей.
		Также положим 
		$ A_k = \defineset{(\varepsilon_1, \ldots, \varepsilon_n)}{\varepsilon_i \in \{0, 1\}, \varepsilon_k = 1} $.
		Рассмотрим отображение 
		$ \varphi \colon \mathbb{R}^{2^n} \to \mathbb{R}^{2^n} $,
		заданное в некоторых фиксированных базисах этих пространств по правилу 
		$$ \varphi\colon 
		\left( \begin{matrix}
			\ldots \\
			p_{(\varepsilon_1, \ldots, \varepsilon_n)} \\
			\ldots
		\end{matrix} \right)
		\mapsto 
		\left( \begin{matrix}
			\ldots \\
			\prob(A_{i_1}\ldots A_{i_k}) \\
			\ldots
		\end{matrix} \right), $$
		где для $ k = 0 $ предполагается, что в матрице стоит $ \prob(\Omega) $.
		Поскольку $ \prob(A_{i_1}\ldots A_{i_k}) = 
		\prob(\defineset{(\varepsilon_1, \ldots, \varepsilon_n)}{\varepsilon_i \in \{0, 1\}, 
			\varepsilon_{i_s} = 1, 1 \leqslant s \leqslant k})
		= \sum\limits_{\varepsilon_{i_s} = 1, 1 \leqslant s \leqslant k} p_{(\varepsilon_1, \ldots, \varepsilon_n)} $,
		то $ \varphi $ --- линейное отображение.
		Можно показать, что $ \varphi $ сюръективно (проверьте с помощью элементарных преобразований, 
		что его матрица имеет ранг $ 2^n $) и, следовательно, биективно.
		Таким образом, достаточно подобрать значения вероятностей все возможных произведений $ A_i $ так, чтобы
		вероятности $ p_{(\varepsilon_1, \ldots, \varepsilon_n)} $ были неотрицательны, в сумме давали 1 ($ \prob(\Omega) = 1 $)
		и при этом выполнялись в точности все желаемые равенства на вероятности произведений событий $ A_i $.
		Положим $ \prob(A_i) = \tfrac{1}{2^{2n}} $, $ \prob(\Omega) = 1 $.
		Если $ (i_1, \ldots, i_k) \in S_{J} $, то положим $ \prob(A_{i_1}\ldots A_{i_k}) = \tfrac{1}{2^{2kn}} $.
		Иначе положим $ \prob(A_{i_1}\ldots A_{i_k}) = \tfrac{1}{2^{2kn + 1}} $.
		Проверим, что имеют место неравенство
		$ \tfrac{1}{2^{2kn + 2}} \leqslant p_{(\varepsilon_1, \ldots, \varepsilon_n)} \leqslant \tfrac{1}{2^{2kn}} $,
		для кортежей с $ k > 0 $ числом единиц.
		Для кортежа $ (1, \ldots, 1) $ неравенство выполнено по построению.
		Докажем неравенства для оставшихся кортежей с данным условием индукцией по числу нулей в кортеже.
		Пусть в текущем кортеже $ (\varepsilon_1, \ldots, \varepsilon_n) $ присутствует $ n - k \geqslant 1 $ нулей.
		Прибавим к $ p_{(\varepsilon_1, \ldots, \varepsilon_n)} $ все остальные значения вероятностей элементарных исходов
		--- кортежей, в которых некоторые нули из данного кортежа заменены на единицы.
		Тогда $ p_{(\varepsilon_1, \ldots, \varepsilon_n)} \leqslant \tfrac{1}{2^{2kn}} $,
		так как по предположению индукции все остальные слагаемые положительны, а сумма не превосходит $ \tfrac{1}{2^{2kn}} $.
		С другой стороны, 
		$ p_{(\varepsilon_1, \ldots, \varepsilon_n)} \geqslant \tfrac{1}{2^{2kn + 1}} - \tfrac{2^k - 1}{2^{2(k + 1)n}}
		= \tfrac{1}{2^{2kn + 1}} - \tfrac{1}{2^{2kn + 2 + (2n - k - 2)}} $.
		Так как $ n - k - 1 \leqslant 0 $ и $ n - 1 \leqslant 0 $, 
		то последнее слагаемое по модулю не превосходит $ \tfrac{1}{2^{2kn + 2}} $.
		Тогда $ p_{(\varepsilon_1, \ldots, \varepsilon_n)} $ не меньше $ \tfrac{1}{2kn + 2} $.
		Остаётся убедиться в том, что $ p_{(0, \ldots, 0)} 
		= 1 - \sum\limits_{(\varepsilon_1, \ldots, \varepsilon_n) \neq (0, \ldots, 0)} p_{(\varepsilon_1, \ldots, \varepsilon_n)}
		\geqslant 1 - \tfrac{2^n - 1}{2^{2n}} > 0 $.
	\end{example}
	
	\begin{proposition} \label{self independent}
		Пусть событие $ A $ независимо с самим собой тогда и только тогда, когда $ \prob(A) = 0 $ или $ \prob(A) = 1 $.
	\end{proposition}
	
	\begin{proof}
		По определению независимости $ \prob(AA) = \prob(A)\prob(A) $, откуда $ \prob(A)^2 - \prob(A) = 0 $
		и либо $ \prob(A) = 0 $, либо $ \prob(A) = 1 $. Обратно, если одно из этих равенств выполнено,
		то $ \prob(A)\prob(A) = \prob(A) = \prob(AA) $.
	\end{proof}
	
	\subsection{Произведение вероятностных пространств}
	
	Если считать, что каждый исход получен в результате отдельного испытания,
	то мы обнаружим, что любое событие, относящееся к фиксированному испытанию
	будет независимым от любого события, относящегося к другим испытаниям.
	В таких случаях говорят о последовательности независимых испытаний.
	
	Формализуем данную ситуацию. Рассмотрим два вероятностных пространства $ (\Omega_1, \events_1, \prob_1) $
	и $ (\Omega_2, \events_2, \prob_2) $. Их \defin{произведением}{product-of-probability-spaces} мы будем называть вероятностное пространство
	$ (\Omega, \events, \prob) $, где $ \Omega = \Omega_1 \times \Omega_2 $ --- декартово произведение,
	$ \events = \events_1 \otimes \events_2 $ --- $ \sigma $-алгебра, порождённая всеми возможными произведениями $ A_1 \times A_2 $,
	где $ A_1 \in \events_1, A_2 \in \events_2 $ --- события в исходных вероятностных пространствах
	(заметьте, что сами по себе таким произведения образуют полукольцо, но вообще говоря не образуют даже кольца),
	а мера $ \prob $ определяется как продолжение меры $ m $ с полукольца произведений, заданной по правилу
	$ m(A_1 \times A_2) := \prob_1(A_1)\cdot \prob_2(A_2) $.
	
	Пусть теперь $ (\Omega, \events, \tilde{\prob}) $ --- вероятностное пространство с теми же $ \Omega $
	и $ \events $, то возможно другой вероятностной мерой. 
	Это можно представлять себе так: происходят два испытания и результаты одного могу <<повлиять>> на результаты другого. 
	Неформальное <<повлиять>> выражается понятием независимости испытаний.
	Мы будем говорить, что испытания, которым соответствуют вероятностные пространства
	$ (\Omega_1, \events_1, \prob_1) $ и $ (\Omega_2, \events_2, \prob_2) $,
	а составному эксперименту, состоящему из двух этих испытаний, \defin{независимы}{independent-experiments},
	если для любых двух событий $ A_1 \in \events_1 $ и $ A_2 \in \events_2 $ выполнено равенство
	$$ \tilde{\prob}(A_1 \times A_2) = \prob_1(A_1)\prob_2(A_2) 
	= \tilde{\prob}(A_1 \times \Omega_2)\tilde{\prob}(\Omega_2 \times A_2). $$

	\section{Случайные величины, их распределения, функции распределения и плотности}
	
	Пусть $ (\Omega, \events, \prob) $ --- вероятностной пространство.
	Функция $ \xi \colon \Omega \to \RR $ называется \defin{случайной величиной}{random-variable},
	если прообраз любого борелевского множества $ B \in \calB $ при отображении $ f $
	лежит в $ \events $. Соотнося это из определениями из действительного анализа, 
	мы можем сказать, что $ \xi $ является \hyperlink{measurable-function}{измеримой функцией}.
	Также, вспомним, что мы получим эквивалентное определение,
	если вместо всех борелевских множеств будем рассматривать все интервалы, все отрезки,
	все интервалы одного из двух видов $ (-\infty, b) $ и $ (a, +\infty) $
	или все полуинтервалы одного из двух видов $ (-\infty, b] $ и $ [a, +\infty) $.
	
	Таким образом, случайные величины удовлетворяют всем тем стандартным свойствам, 
	которым удовлетворяют измеримые функции:
	
	\begin{proposition} \label{random variebles composition with continuous}
		Пусть $ \xi $ --- случайная величины и $ g $ --- непрерывная на $ \Image \xi $ функция.
		Тогда композиция $ g(\xi) $ является случайной величиной.
	\end{proposition}
	
	\begin{proof}
		Переформулировка предложения \ref{composition with continuous}.
	\end{proof}
	
	\begin{proposition} \label{random variables are good}
		Пусть $ \xi, \eta $ --- случайные величины.
		Тогда множество $ A_{\xi \leqslant \eta} = \defineset{\omega \in \Omega}{\xi(\omega) \leqslant \eta(\omega)} $
		измеримо.
		Функции $ a + \xi, a\xi, |\xi|, \xi^2, \xi + \eta $ и $ \xi\eta $, где $ a $ --- константа, являются случайными величинами.
		Если случайная величина $ \eta $ не принимает значения 0, то функции $ \tfrac{1}{\eta} $ и $ \tfrac{\xi}{\eta} $
		также являются случайными величинами.
	\end{proposition}
	
	\begin{proof}
		Переформулировка предложения \ref{measurable functions are good}.
	\end{proof}
	
	Случайная величина называется \defin{дискретной}{discrete-random-variable}, 
	если она принимает не более, чем счётное число различных значений.
	Случайные величины на дискретном вероятностном пространстве всегда дискретны.
	Кроме того, всякая функция на дискретном вероятностном пространстве (в смысле \hyperlink{discr-2}{второго определения})
	является случайной величиной.
	
	\begin{example}
		Пусть $ \Omega = \{\text{ГГ}, \text{ГP}, \text{PГ}, \text{PP}\} $
		--- все возможные результаты выпадения на двух монетках с гербом (Г) на одной стороне и решкой (Р) на другой.
		Будем считать, что вероятности всех элементарных событий равны $ \tfrac{1}{4} $.
		Пусть $ \xi \colon \Omega \to \{0, 1\} $ сопоставляет элементам ГГ, ГР и РГ единицу, а РР --- ноль.
		Тогда $ \xi $ --- дискретная случайная величина.
		Мы можем интерпретировать $ 0 $ как <<неудачу>> в эксперименте <<бросить две монетки и получить хотя бы один герб>> 
		и $ 1 $ --- <<успех>>. 
		В нашем случае $ \xi $ принимает значение $ 1 $ с вероятностью 
		$ \prob(\xi^{-1}(\{1\})) = \prob(\{\text{ГГ}, \text{ГP}, \text{PГ}\}) = \tfrac{3}{4} $ 
		и значение $ 0 $ с вероятностью $ \prob(\xi^{-1}(\{0\})) = \prob(\{\text{PP}\}) = \tfrac{1}{4} $.		
	\end{example}
	
	Вместо $ \prob(\xi^{-1}(\{a\})) $ мы будем пользоваться записью $ \prob(\xi = a) $, отражающей смысл этого выражения
	--- <<вероятность случайной величины $ \xi $ принять значение $ a $>>.
	Также вместо перегруженных скобками выражений 
	$ \prob(\xi^{-1}((-\infty; b))), \prob(\xi^{-1}((a; +\infty))), \\ \prob(\xi^{-1}((a; b))) $ 
	и им подобных мы будем писать $ \prob(\xi \leqslant b), \prob(\xi \geqslant a) $ и 
	$ \prob(a \leqslant \xi \leqslant b) $.
	
	\begin{example}
		В примере выше $ \prob(\xi = 1) = \tfrac{3}{4} $ и $ \prob(\xi = 0) = \tfrac{1}{4} $.
	\end{example}
	
	\begin{example}
		Дискретная случайная велиичина может быть задана и не на дискретном вероятностном пространстве.
		Пусть $ \Omega = [-1;1] $ --- отрезок, $ \events = \calB $ --- сигма алгебра его борелевских подмножеств,
		а вероятностная мера --- это умноженная на $ \tfrac{1}{2} $ классическая мера Лебега $ \mu $.
		Тогда случайная величина $ \xi(\omega) = sgn(\omega) $, заданная функцией знак, 
		является дискретной случайной величиной.
	\end{example}
	
	\begin{example}
		Вернёмся одному из первых рассмотренных примеров. Пусть $ \Omega = [0, 2] $ и $ \events = M $ --- $ \sigma $-алгебра измеримых относительно меры Лебега
		подмножеств отрезка $ [0, 2] $, $ \prob = \tfrac{1}{2}\mu $ --- мера, пропорциональная мере Лебега.
		Пусть $ \xi(\omega) = \omega^2 $. Эта функция непрерывна, поэтому является случайной величиной (измеримой функцией).
		Найдём $ \prob(\tfrac{1}{4} \leqslant \xi < \tfrac{16}{9}) $.
		Так как неравенства $ \tfrac{1}{4} \leqslant \xi(\omega) < \tfrac{16}{9} $ выполнены тогда и только тогда, когда
		$ \tfrac{1}{2} \leqslant \omega < \tfrac{4}{3} $. Тогда $ \prob(\tfrac{1}{4} \leqslant \xi < \tfrac{16}{9}) = \tfrac{1}{2}(\tfrac{4}{3} - \tfrac{1}{2}) = \tfrac{5}{12} $.
	\end{example}
	
	Введём некоторые объекты, характеризующие случайную величину $ \xi $ через её вероятность принять некоторые значения.
	\defin{Функцией распределения случайной величины $ \xi $}{distribution-function} 
	будем называть функцию $ F_\xi \colon \RR \to \RR $, заданную по правилу 
	$ F_\xi(x) = \prob(\xi \leqslant x) $.
	Заметим, что построенная функция $ F_\xi $ корректно определена на всей числовой прямой,
	так как $ \xi $ --- случайная величина и прообраз бесконечного полуинтервала относительно $ \xi $
	является событием (измерим).
	
	\begin{example}
		Функция распределения дискретной случайной величины $ \xi $ имеет <<ступенчатую форму>>.
		Пусть $ \xi $ принимает конечное число значений $ x_1 < x_2 < \ldots < x_n $
		и $ \prob(\xi = x_i) = p_i $.
		Тогда функция распределения $ F_\xi $ может быть задана как 
		$$ F_\xi(x) = 
		\begin{cases}
			0, & x < x_1; \\
			p_1, & x_1 \leqslant x < x_2; \\
			p_1 + p_2, & x_2 \leqslant x < x_3; \\
			\ldots \\
			p_1 + p_2 + \ldots + p_n = 1, & x \geqslant x_n.
		\end{cases} $$
		График $ F_\xi $ будет выглядеть следующим образом.
		% Спасибо нейронке за картинку
		$$ \begin{tikzpicture}[scale=2]
			
			% Параметры
			\def\xone{-2}
			\def\xtwo{1}
			\def\xnminone{4}
			\def\xn{5}
			
			\def\pone{0.4}
			\def\ptwo{0.3}
			\def\poneptwo{0.7}
			\def\pnminone{0.8}
			\def\psum{1.5}
			
			% Оси
			\draw[->, thin] (-2.5,0) -- (5.5,0) node[right] {$x$};
			\draw[->, thin] (0,0) -- (0,2.2) node[above] {$F_\xi(x)$};
			
			% Подписи x_i
			\foreach \x/\label in {\xone/{$x_1$}, \xtwo/{$x_2$}, \xnminone/{$x_{n-1}$}, \xn/{$x_n$}} {
				\draw (\x,0) node[below] {\label};
			}
			
			% График функции распределения
			\draw[very thick] (-2.5,0) -- (\xone,0);
			\draw[very thick] (\xone,\pone) -- (\xtwo,\pone);
			\draw[very thick] (\xtwo,\poneptwo) -- (\xnminone,\poneptwo);
			\draw[very thick] (\xnminone,\psum) -- (\xn,\psum);
			\draw[very thick] (\xn,2.0) -- (5.5,2.0);
			
			% Точки скачков (пустой снизу, закрашенный сверху)
			\draw[fill=white] (\xone,0) circle (0.035);
			\fill (\xone,\pone) circle (0.035);
			
			\draw[fill=white] (\xtwo,\pone) circle (0.035);
			\fill (\xtwo,\poneptwo) circle (0.035);
			
			\draw[fill=white] (\xnminone,\poneptwo) circle (0.035);
			\fill (\xnminone,\psum) circle (0.035);
			
			\draw[fill=white] (\xn,\psum) circle (0.035);
			\fill (\xn,2.0) circle (0.035);
			
			% Красные пунктирные линии уровня от графика до оси Oy
			\foreach \x/\y/\label in {
				\xone/\pone/{$p_1$},
				\xtwo/\poneptwo/{$p_1 + p_2$},
				\xnminone/\psum/{$p_1 + \ldots + p_{n-1}$},
				\xn/2.0/{$1$}
			} {
				\draw[red, dashed, thin] (\x,\y) -- (0,\y);
				\node[left, red] at (0,\y) {\label};
			}
			
			% Значение 0 на оси Oy
			\node[left, red] at (0,0) {$0$};
			
			% Многоточие между x_2 и x_{n-1}
			\node at (2.5, -0.05) {$\cdots$};
			
		\end{tikzpicture} $$
	\end{example}
	
	Выделим некоторые свойства этой функции.
	
	\begin{proposition} \label{distribution function properties}
		Пусть $ \xi $ --- случайная величина и $ F_\xi $ --- её функция распределения.
		Тогда выполнены следующие свойства
		\begin{enumerate}
			\item функция $ F_\xi $ неубывает$ ; $ \label{distribution function properties | monotonic}
			\item функция $ F_\xi $ непрерывна справа$ ; $ \label{distribution function properties | right continuous}
			\item $ F_\xi(+\infty) := \lim\limits_{x \to +\infty} F_\xi(x) = 1; $ 
			\label{distribution function properties | 1 at +inf}
			\item $ F_\xi(-\infty) := \lim\limits_{x \to -\infty} F_\xi(x) = 0. $
			\label{distribution function properties | 0 at -inf}
 		\end{enumerate}
	\end{proposition}
	
	\begin{proof}
		Для доказательства неубывания \ref{distribution function properties | monotonic}
		воспользуемся свойствами вероятностной меры. Пусть $ x < y $.
		Тогда
		$$ F_\xi(y) = \prob(\xi \leqslant y) = \prob(\xi \leqslant x) + \prob(x < \xi \leqslant y) 
		\geqslant \prob(\xi \leqslant x) = F_\xi(x). $$
		
		Проверим непрерывность справа \ref{distribution function properties | right continuous}. 
		Пусть последовательность $ \{x_n\} $ сходится к $ x_0 $ справа (можно считать, что последовательность возрастает).
		Тогда $ \bigcap\limits_{n = 1}^{+\infty} (-\infty; x_n] = (-\infty; x_0] $
		и по непрерывности вероятностной меры 
		$$ \lim\limits_{x \to x_0+0} F_{\xi}(x) = \lim\limits_{n \to +\infty} F_{\xi}(x_n) 
		= \lim\limits_{n \to +\infty} \prob(\xi \leqslant x_n) = \prob(\xi \leqslant x_0) = F_{\xi}(x_0). $$
		
		Свойства \ref{distribution function properties | 1 at +inf}
		и \ref{distribution function properties | 0 at -inf} также следуют из непрерывности
		$$ \lim\limits_{x \to +\infty} F_{\xi}(x) = \lim\limits_{n \to +\infty} F_{\xi}(x_n) 
		= \lim\limits_{n \to +\infty} \prob(\xi \leqslant x_n) = \prob(\xi \in \RR) = 1; $$
		$$ \lim\limits_{x \to -\infty} F_{\xi}(x) = \lim\limits_{n \to +\infty} F_{\xi}(x_n) 
		= \lim\limits_{n \to +\infty} \prob(\xi \leqslant x_n) = \prob(\xi \in \varnothing) = 0. $$
		Здесь в первом случае $ \bigcup\limits_{n = 1}^{+\infty} (-\infty; x_n] = (-\infty; +\infty) = \RR $,
		а во втором $ \bigcap\limits_{n = 1}^{+\infty} (-\infty; x_n] = (-\infty; -\infty] = \varnothing $.
	\end{proof}
	
	Будем обозначать через $ F_\xi(x-) $ или $ F_\xi(x-0) $ левосторонний предел функции $ F_\xi $ в точке $ x $.
	Тогда $ \prob(\xi = x) = F_\xi(x) - F_\xi(x-) $. 
	
	Случайная величина $ \xi $ называется \defin{непрерывной}{continuous-random-variable},
	если её функция распределения $ F_\xi $ непрерывна на $ \RR $.
	
	\begin{proposition}
		Случайная величина $ \xi $ непрерывна тогда и только тогда, когда для любого числа $ a \in \RR $
		вероятность того, что $ \xi $ примет значение $ a $ равна нулю: $ \prob(\xi = a) = 0 $.
	\end{proposition}
	
	\begin{proof}
		Функция $ F_\xi $ монотонна и поэтому у неё существуют левые пределы во всех точках.
		Так как $ F_\xi $ ещё и непрерывна справа, то непрерывность в точке $ a $
		для неё равносильна равенству $ F_\xi(a) = F_\xi(a-) $,
		что в свою очередь равносильно равенству $ \prob(\xi = a) = F_\xi(a) - F_\xi(a-) = 0 $.
	\end{proof}
	
	\begin{theorem}
		Пусть функция $ F $ удовлетворяет свойствам из предложения \ref{distribution function properties}.
 		Тогда существует единственная вероятностная мера $ \prob_F $ на борелевской $ \sigma $-алгебре $ \calB $
 		подмножеств $ \RR $ такая, что функция $ F $ является функцией распределения случайной величины
 		$ \id \colon (\RR, \calB, \prob_F) \to (\RR, \calB) $.
	\end{theorem}
	
	\begin{proof}
		Существование следует из конструкции меры Лебега-Стилтьеса $ \mu_F $ для функции $ F $.
		Для неё функцией распределения тождественного отображения автоматически становится функция $ F $
		(так как буквально теми же формулами определяется эта мера по полуинтервалах).
		
		Пусть теперь $ \prob $ --- некоторая вероятностная мера 
		и функция $ F $ оказывается функцией распределения для тождественного отображения $ \id \colon x \mapsto x $
		(далее мы будем также обозначать его через $ x $).
		Тогда имеем
		$$ \prob((a, b]) = \prob(a < x \leqslant b) = \prob(x \leqslant b) - \prob(x \leqslant a) = F(b) - F(a); $$
		$$ \prob((-\infty, b]) = \prob(x \leqslant b) - 0 = F(b) - F(-\infty); $$
		$$ \prob((a, +\infty)) = \prob(x > a) = 1 - \prob(x \leqslant a) = 1 -  F(a) = F(+\infty) - F(a). $$
		Мы показали, что меры $ \prob $ и $ \mu_F $ совпадают на полуинтервалах,
		следовательно, они совпадают на минимальной алгебре $ \calA $, порождённой ими.
		Так как борелевская $ \sigma $-алгебра $ \calB $ является минимальной $ \sigma $-алгеброй, содержащей $ \calA $,
		то по теореме Каратеодори \ref{uniqueness of extention to the minimal sigma algebra}
		меры $ \prob $ и $ \mu_F $ совпадают на ней.
	\end{proof}
	
	Для вероятностной меры $ \prob $ на измеримом пространстве $ (\RR, \calB) $ функцию распределения случайной величины
	тождественного отображения $ \id $ мы будем иногда называть 
	\defin{функцией распределения вероятностной меры $ \prob $}{distribution-function-of-measure}.
	
	Следующая теорема не формулируется в основном курсе, но даёт ещё одно описание для меры $ \prob_F $. 
	
	Прямой образ меры $ \xi_*\prob $ (напомним, что он определяется правилом $ \xi_*\prob(B) = \prob(\xi^{-1}(B)) = \prob(\xi \in B) $) 
	на борелевской $ \sigma $-алгебре $ \calB $ подмножеств $ \RR $
	называется \defin{распределением (вероятностей) случайной величины $ \xi $}{distribution}.
	
	\begin{theorem}
		Пусть $ \xi $ --- случайная величина и $ F_\xi $ её функция распределения.
		Пусть $ \mu_\xi $ --- мера Лебега-Стилтьеса на подмножествах $ \RR $, порождённая функцией $ F_\xi $.
		На \hyperlink{borel-sigma-algebra}{борелевской} $ \sigma $-алгебре с  мера $ \mu_\xi $ совпадает 
		с распределением $ \xi_*\prob $ вероятностной меры $ \prob $.
	\end{theorem}
	
	\begin{proof}
		Достаточно проверить, что эти меры совпадают на минимальном кольце, порождённом всеми полуинтвералами вида $ (a, b] $,
		$ (-\infty; b] $ и $ (a; +\infty) $. Тогда равенство на борелевской $ \sigma $-алгебре будет следовать
		из теоремы Каратеодори \ref{uniqueness of extention to the minimal sigma algebra}.
		
		Поскольку мера на минимальном кольце, порождённом данным полукольцом, однозначной определяется
		по мере на этом полукольце, то достаточно проверить равенства на самих полуинтервалах.
		
		Все равенства немедленно следуют из определений функции распределения, меры Лебега-Стилтьеса и прямого образа меры,
		а также свойств меры и функции распределения:
		$$ \mu_f((a, b]) = F_\xi(b) - F_\xi(a) = \prob(\xi \leqslant b) - \prob(\xi \leqslant a)
		= \prob(a < \xi \leqslant b) = \prob(\xi^{-1}((a, b])) = \xi_*\prob((a, b]); $$
		$$ \mu_f((a, +\infty)) = F_\xi(+\infty) - F_\xi(a) = 1 - \prob(\xi \leqslant a)
		= \prob(\xi > a) = \prob(\xi^{-1}((a, +\infty))) = \xi_*\prob((a, +\infty)); $$
		$$ \mu_f((-\infty; b]) = F_\xi(b) - F_\xi(-\infty) = \prob(\xi \leqslant b) - 0
		= \prob(\xi \leqslant b) = \prob(\xi^{-1}((-\infty; b])) = \xi_*\prob((-\infty; b]). $$
	\end{proof}
	
	В условиях последних двух теорем $ \mu_F = \xi_*\prob_F $.
	Если случайная величина $ \xi $ имеет распределение $ \Gamma $ мы будем писать
	$ \xi \sim \Gamma $.
	
	Распределение дискретной случайной величины $ \xi $ называется \defin{дискретным}{discrete-distribution}.
	Если случайная величина $ \xi $ может принимать только значения $ x_1, x_2, \ldots $
	с вероятностями $ p_i = \prob(\xi = x_i) $ (при этом мы будем считать, что вероятности $ \prob(\xi = x_i) $ положительны, чтобы не рассматривать вырожденные случаи), то можно удобно изобразить это с помощью таблицы
	$$\begin{tabular}{|c|c|c|c}
		\hline
		$ \xi $ & $ x_1 $ & $ x_2 $ & \ldots \\
		\hline
		& $ p_1 $ & $ p_2 $ & \ldots \\
		\hline
	\end{tabular}
	\text{ или }
	\xi \sim
	\begin{tabular}{ccc}
		$ x_1 $ & $ x_2 $ & \ldots \\
		$ p_1 $ & $ p_2 $ & \ldots
	\end{tabular} $$
	Поскольку $ \sum\limits_{i = 1}^{+\infty} p_i = \sum\limits_{i = 1}^{+\infty} \prob(\xi = x_i)
	= \prob(\Omega) = 1 $, то дискретное распределение $ \{p_i\} $ можно задать на дискретном вероятностном пространстве. 
	
	Распределение случайной величины $ \xi $ и она сама называются \defin{абсолютно непрерывными}{absolutely-continuous},
	если существует функция $ p \colon \mathbb{R} \to \mathbb{R} $ такая, что
	для любого борелевского множества $ B \in \calB $ выполнено равенство
	$$ \prob(\xi \in B) = \xi_*\prob(B) = \int\limits_{B} f\diff\mu, $$
	где $ \mu $ --- классическая мера Лебега.
	С точки зрения теории меры и интеграла Лебега это определение эквивалентно тому,
	что мера $ \xi_*\prob $ абсолютно непрерывна относительно классической меры Лебега $ \mu $ на прямой
	(это значит, что случайная величина $ \xi $ попадает в множество меры нуль с вероятностью нуль).
	Теорема Радона-Никодима даёт формулу для $ \xi_*\prob $, принятую нами за определение.
	
	
	В силу доказанных выше теорем мы можем дать эквивалентное определение в терминах функции распределения.
	Будем говорить, что распределение случайной величины $ \xi $ и она сама называются \defin{абсолютно непрерывными}{absolutely-continuous-2}, если функция распределения $ F_\xi $ может быть выражена как интеграл
	$$ F_\xi(x) = \int\limits_{(-\infty; x]} p\diff\mu. $$
	В действительном анализе такая функция $ F_\xi $ называется абсолютно непрерывной,
	почти всюду существует производная $ F_\xi' $, которая почти всюду совпадает с $ p(x) $,
	а также имеет место аналог формулы Ньютона-Лейбница: $ F_\xi(b) - F_\xi(a) = \int\limits_{a}^{b} p\diff\mu $.
	Можно доказать, что функция абсолютно непрерывна тогда и только тогда,
	когда она непрерывна, имеет ограниченную вариацию и переводим множества меры нуль по Лебегу в множества меры нуль.
	
	\section{Классические примеры распределений} 
	
	В этом разделе мы рассмотрим классические примеры распределений,
	повсеместно встречающихся в теории вероятностей и в её приложениях.
	
	Важно помнить, что мы описываем распределение случайной величины, а не непосредственно её саму 
	(случайные величины с одинаковыми распределениями могут быть заданы на разных вероятностных пространствах).
	Вместо данной случайной величины всегда можно рассмотреть случайную величину $ \id $ на вероятностном пространстве
	$ (\RR, \calB, P) $ для некоторого $ P $ 
	(или на пространстве $ (U, \calB \cap U, P) $ для некоторого подмножества $ U \subset \RR $),
	то есть распределение некоторой вероятностной меры на $ \RR $ или подмножестве $ \RR $.
	Случайную величину, обладающую распределением $ \Gamma $ также называют $ \Gamma $-ой (бернуллиевской, геометрической, биномиальной и т.д.) случайной величиной.
	
		
	\TODO{вписать описания для всех классических распределений}
	
	\subsection{Дискретные распределения.}
	
	Приведём примеры дискретных распределений. Каждый пример мы будем снабжать описывающей его таблицей.
	
	\subsubsection{Распределение константы}
	
	Пусть случайная величина $ \xi $ принимает значение $ C $ с вероятность $ 1 $: $ \prob(\xi = C) = 1 $.
	Тогда распределение такой случайной величины называется \defin{распределением константы}{distribution-of-constant}
	или <<вырожденным распределением>>. Используя табличку можем записать
	$$ \xi \sim 
	\begin{tabular}{c}
		$ C $ \\
		$ 1 $ 
	\end{tabular}.
	 $$
	
	\subsubsection{Распределение Бернулли}
	
	Пусть случайная величина $ \xi $ принимает значения $ 1 $ и $ 0 $ с вероятностями, соответственно
	$ p $ и $ q $ ($ p + q = 1 $, $ p, q \geqslant 0 $).
	Её распределение называется \defin{распределением Бернулли}{Bernoulli-distribution}
	и обозначается $ \mathrm{Be}(p) $. 
	И снова в виде таблицы
	$$ \xi \sim 
	\begin{tabular}{cc}
		$ 0 $ & $ 1 $ \\
		$ q $ & $ p $
	\end{tabular}
	\sim \mathrm{Be}(p).
	$$
	
	\begin{example}
		Рассмотрим дискретное вероятностное пространство $ \Omega = \{1,2,3,4,5,6\} $ результатов броска игральной кости.
		Будем считать, что наша игральная кость <<иделальная>>, 
		то есть вероятности всех элементарных событий равны $ \tfrac{1}{6} $.
		Пусть $ \xi \colon \Omega \to \RR $ --- случайная величина, принимающая значение $ 1 $ на числах множестве $ \{1,3\} $
		и $ 0 $ иначе и выражающая смысл <<выпала степень тройки>>.
		Тогда $ \xi $ обладает биномиальным распределением и
		$$ \xi \sim 
		\begin{tabular}{cc}
			$ 0 $ & $ 1 $ \\
			$ \tfrac{2}{3} $ & $ \tfrac{1}{3} $
		\end{tabular}.
		$$
	\end{example}
	 
	
	\subsubsection{Дискретное равномерное распределение}
	
	Рассмотрим случайную величину $ \xi $ принимающую значения из набора $ x_1, x_2, \ldots, x_n $, каждое
	с вероятностью $ \tfrac{1}{n} $. Распределение случайной величины $ \xi $ называется 
	\defin{дискретным распределением}{discrete-uniform-distribution} и обозначается $ \mathrm{R}\{1,\ldots,n\} $. В виде таблицы:
	$$ \xi \sim 
	\begin{tabular}{ccc}
		$ x_1 $ & $ \ldots $ & $ x_n $ \\
		$ \tfrac{1}{n} $ & \ldots & $ \tfrac{1}{n} $
	\end{tabular}
	\sim 
	\mathrm{R}\{1,\ldots,n\}.
	$$
	
	\begin{example}
		В качестве примера случайной величины с равномерным распределением можно взять
		случайную величину $ \xi $ на дискретном пространстве $ \Omega = \{1,2,3,4,5,6\} $
		(вероятности всех элементарных событий снова равны $ \tfrac{1}{6} $), сопоставляющую числу из $ \Omega $ его же само.
	\end{example}
	
	\subsubsection{Биномиальное распределение}
	
	Представим, что происходит $ n $ последовательных \hyperlink{independent-experiments}{независимых испытаний}, 
	в каждом из которых с вероятностью $ 0 < p < 1 $ происходит <<успех>>, а с вероятностью $ q = 1 - p $ <<неудача>>.
	Формально, имеется пространство элементарных исходов $ \Omega = \{0, 1\}^{n} $ с заданным на нём распределением вероятностей, 
	полученное как \hyperlink{product-of-probability-spaces}{произведение} вероятностных пространств
	$ (\{0, 1\}, 2^{\{0, 1\}}, \prob) $, где $ \prob(1) = p, \prob(0) = q $.
	Из построения получаем вероятность элементарного исхода $ (\varepsilon_1, \ldots, \varepsilon_n) $
	равной $ p^kq^{n - k} $, где $ k $ --- это <<число успехов>>, 
	то есть число единиц в кортеже $ (\varepsilon_1, \ldots, \varepsilon_n) $.
	
	Рассмотрим случайную величину $ \xi $ принимающую на $ \Omega $ значения, равные <<числу успехов>> (сумме единиц в кортеже). 
	Тогда $ \xi $ принимает значения от $ 0 $ до $ n $, причём значение $ k $ принимается ей с вероятностью
	$ C_{n}^{k}p^kq^{n - k} $.
	Распределение такой случайной величины $ \xi $ называется 
	\defin{биномиальным распределением}{binomial-distribution} и обозначается $ \mathrm{B}(n,p) $.
	В виде таблицы:
	$$ \xi \sim 
	\begin{tabular}{cccccc}
		$ 0 $ & $ 1 $ & $ \ldots $ & $ k $ & $ \ldots $ & $ n $ \\
		$ q^n $ & $ npq^{n - 1} $ & \ldots & $ C_n^kp^kq^{n - k} $ & $ \ldots $ & $ p^n $
	\end{tabular}.
	$$
	
	\subsubsection{Распределение Пуассона}
	
	Теперь мы введём счётный аналог биномиального распределение.
	Воспользуемся разложением экспоненты в ряд
	$$ e^{\lambda} = \sum\limits_{k = 1}^{+\infty} \tfrac{\lambda^k}{k!}. $$
	Пусть случайная величина $ \xi $ принимает целые неотрицательные значения 
	и принимает значение $ k \in \mathbb{Z}_{\geqslant 0} $
	с вероятностью $ \tfrac{\lambda^k}{k!}e^{-\lambda} $, $ \lambda > 0 $ 
	(это условие нужно для того, чтобы все вероятности были положительными).
	Распределение такой случайной величины называется \defin{распределение Пуассона}{Poisson-distribution} 
	и обозначается $ \Pi(\lambda) $. Выписывая это в виде таблицы получаем
	$$ \xi \sim 
	\begin{tabular}{cccccc}
		$ 0 $ & $ 1 $ & $ 2 $ & $ \ldots $ & $ k $ & $ \ldots $ \\
		$ e^{-\lambda} $ & $ \lambda e^{-\lambda} $ & $ \tfrac{\lambda^2}{2}e^{-\lambda} $ 
		& $ \ldots $ & $ \tfrac{\lambda^k}{k!}e^{-\lambda} $ & $ \ldots $
	\end{tabular}
	\sim
	\Pi(\lambda).
	$$
	
	\subsubsection{Геометрическое распределение}
	
	Рассмотрим эксперимент с двумя исходами <<успех>> и <<неудача>>, 
	который повторяется до достижения <<успеха>>.
	Будем считать, что в каждом отдельном испытании <<успех>> достигается с вероятностью $ 0 < p < 1 $,
	а <<неудача>> с вероятностью $ q = 1 - p $.
	Рассмотрим вероятностное пространство $ \Omega = \mathbb{N} $ с распределением 
	$ \prob(n) = pq^{n - 1} $ --- вероятность того эксперимент закончится на $ n $-м этапе
	(произошли $ n - 1 $ <<неудача>> и затем <<успех>>).
	Пусть случайная величина $ \xi $ сопоставляет числу из $ \Omega $ его же само.
	Таким образом, $ \xi $ несёт в себя смысл <<количество испытаний, проведённых до первого успеха>>.
	Распределение такой случайной величины называется \defin{геометрическим распределением}{geometric-distribution}
	и обозначается $ \mathrm{G}(p) $.
	Изобразим таблицу:
	$$ \xi \sim 
	\begin{tabular}{cccccc}
		$ 1 $ & $ 2 $ & $ 3 $ & $ \ldots $ & $ k $ & $ \ldots $ \\
		$ p $ & $ pq $ & $ pq^2 $ & $ \ldots $ & $ pq^{k - 1} $ & $ \ldots $
	\end{tabular}
	\sim
	\mathrm{G}(p).
	$$
	
	\begin{example}
		Пусть $ \Omega = \{0,1\}^{\mathbb{N}} $ --- множество всех последовательностей из нулей и единиц.
		Пусть $ \events $ --- $ \sigma $-алгебра, полученная лебеговским продолжением с полукольца последовательностей,
		имеющих фиксированное общее начало 
		(в нём содержатся, например, последовательности, начинающиеся с $ 1 $, с $ 1101 $ и т.д., 
		также будем включать в полукольцо всё $ \Omega $), 
		по мере $ m $ заданной по правилу $ m(S) = p^kq^m $ для множества $ S $ последовательностей с общим началом
		$ \varepsilon_1\varepsilon_2\ldots\varepsilon_{k + m} $. Здесь $ k $ равно числу единиц в начальном префиксе, 
		а $ m $ --- числу нулей. 
		Мы опустим проверку того, что это последовательности с общим фиксированным началом образуют полукольцо,
		а заданная на них функция $ m $ является $ \sigma $-аддитивной мерой.
		Теперь пусть случайная величина $ \xi $ равна номеру позиции на которой встретилась первая единица последовательности
		и $ 1 $ на последовательности из одних нулей.
		Тогда $ \xi $ обладает биномиальным распределением.
	\end{example}
	
	\subsubsection{Гипергеометрическое распределение}
	
	Теперь рассмотрим эксперимент с шарами. Пусть в мешке лежат $ N $ шаров (уникальных), 
	каждый из которых покрашен в чёрный или в белый цвет.
	Путь число белых шаров в мешке равно $ M $. Вытаскивается $ 1 \leqslant n \leqslant N $ шаров
	(считается, что вероятности вытащить все возможные наборы шаров одинаковы).
	Обозначим через $ \xi $ случайную величину, равную числу вытащенных белых шаров и выясним, чему равна вероятность
	вытащить $ m $ белых шаров среди всех $ n $ шаров.
	Формально пространство элементарных исходов $ \Omega $ состоит из всевозможных подмножеств мощности $ n $
	множества всех шаров, а $ \xi $ принимает значение равное количеству белых шаров в этом подмножестве.
	Вероятность будет равна $ 0 $, если $ m > n $ или $ m > M $ или $ n - m > N - M $. Разберём остальные случаи.
	Имеем $ |\Omega| = C_N^n $. Число способов выбрать $ m $ белых шаров из общего числа $ M $
	и $ n - m $ чёрных шаров из общего числа $ N - M $ равно $ C_M^m $ и $ C_{N - M}^{n - m} $, соответственно.
	Поскольку вероятность выбрать данные $ n $ шаров равна $ \tfrac{1}{C_N^n} $,
	то $ \prob(\xi = m) = \tfrac{C_M^m \cdot C_{N - M}^{n - m}}{C_N^n} $ при всех наложенных ограничениях.
	Распределение такой случайной величины называется \defin{гипергеометрическим распределением}{hypergeometric-distribution}
	
	\subsubsection{Отрицательное биномиальное распределение}
	
	Кроме подсчёта числа <<успехов>>, как это делается биномиальной случайной величиной,
	и подсчёта числа попыток, после которых достигается <<успех>>, как в случае с геометрической случайной величиной,
	можно считать число попыток, после которых достигается $ r $ успехов.
	Снова будем считать, что в каждом отдельном испытании <<успех>> достигается с вероятностью $ p $,
	а неудача происходит с вероятностью $ q $.
	В качестве вероятностного пространства можно снова взять пространство $ \Omega = \mathbb{N} $ с распределением 
	$ \prob(k) = C_{k - 1}^{r - 1}p^rq^{k - r} $ для $ k \geqslant r $ --- вероятность того эксперимент закончится на $ k $-м этапе 
	(то есть в этот момент будет достигнут $ r $-й <<успех>>, 
	а среди предыдущих $ k - 1 $ будут как-то разбросаны ещё $ r - 1 $ <<успех>>).
	Пусть случайная величина $ \xi $ сопоставляет числу из $ \Omega $ его же само.
	Таким образом, $ \xi $ несёт в себя смысл <<количество испытаний, проведённых до $ r $-го успеха>>.
	Распределение такой случайной величины называется \defin{геометрическим распределением}{geometric-distribution}
	и обозначается $ \mathrm{NB}(r,p) $.
	Изобразим таблицу:
	$$ \xi \sim 
	\begin{tabular}{cccccc}
		$ r $ & $ r + 1 $ & $ r + 2 $ & $ \ldots $ & $ k $ & $ \ldots $ \\
		$ p^r $ & $ C_{r}^{r - 1}p^rq $ & $ C_{r + 1}^{r - 1}p^rq^{2} $ & $ \ldots $ & $ C_{k - 1}^{r - 1}p^rq^{k - r} $ & $ \ldots $
	\end{tabular}
	\sim
	\mathrm{NB}(r,p).
	$$
	
	\begin{example}
		Пусть как и в примере с геометрической случайной величиной 
		$ \Omega = \{0,1\}^{\mathbb{N}} $ --- множество всех последовательностей из нулей и единиц.
		Пусть $ \events $ --- $ \sigma $-алгебра, полученная лебеговским продолжением с полукольца последовательностей,
		имеющих фиксированное общее начало 
		(в нём содержатся, например, последовательности, начинающиеся с $ 1 $, с $ 1101 $ и т.д., 
		также будем включать в полукольцо всё $ \Omega $), 
		по мере $ m $ заданной по правилу $ m(S) = p^kq^m $ для множества $ S $ последовательностей с общим началом
		$ \varepsilon_1\varepsilon_2\ldots\varepsilon_{k + m} $. Здесь $ k $ равно числу единиц в начальном префиксе, 
		а $ m $ --- числу нулей. 
		Мы опустим проверку того, что это последовательности с общим фиксированным началом образуют полукольцо,
		а заданная на них функция $ m $ является $ \sigma $-аддитивной мерой.
		Теперь пусть случайная величина $ \xi $ равна номеру позиции на которой встретилась $ r $-я единица последовательности
		и $ r $, если в последовательности содержится меньше $ r $ единиц.
		Тогда $ \xi $ обаладает отрицательным биномиальным распределением.
	\end{example}
	
	\subsection{Абсолютно непрерывные случайные величины}
	
	\subsubsection{Равномерное распределение}
	
	Абсолютно непрерывным аналогом дискретного равномерного распределения выступает равномерное распределение.
	Пусть $ \Omega = [a,b] $ --- отрезок, $ \events = \calB \cap [a,b] $ --- борелевская $ \sigma $-алгебра
	и вероятностная мера задана как $ \prob = \tfrac{1}{b - a}\mu $ --- нормированная мера Лебега.
	Рассмотрим распределение случайной величины $ \xi $, которая сопоставляет точке на отрезке её координату.
	Можно думать об этом, как об эксперименте в котором на отрезок <<бросается>> точка
	и изучаются её вероятности попасть в разные подмножества отрезка.
	Распределение случайной величины $ \xi $ называется \defin{равномерным распределением}{uniform-distribution}
	и обозначается $ \mathrm{R}[a,b] $.
	Функция распределения $ F_\xi $ это случайной величины может быть задана кусочно.
	\begin{center}
		\begin{minipage}{0.45\textwidth}
			\[
			F_\xi(x) =
			\begin{cases}
				0, & x < a, \\
				\dfrac{x - a}{b - a}, & x \in [a, b], \\
				1, & x > b.
			\end{cases}
			\]
		\end{minipage}
		\def\a{1}
		\def\b{4}
		\begin{minipage}{0.5\textwidth}
			\begin{tikzpicture}
				\begin{axis}[
					axis lines=middle,
					axis line style={very thin},
					xlabel={$x$}, ylabel={$F_\xi(x)$},
					ymin=0, ymax=1.2,
					xmin=0, xmax=5,
					xtick={1,4}, xticklabels={$a$, $b$},
					ytick={0,1},
					width=7cm,
					height=4.5cm,
					]
					\addplot[domain=0:\a, black, ultra thick] {0};
					\addplot[domain=\a:\b, black, thick] {(x - \a)/(\b - \a)};
					\addplot[domain=\b:5, black, thick] {1};
				\end{axis}
			\end{tikzpicture}
		\end{minipage}
	\end{center}
	Данна случайная величина обладает функцией плотности.
	\begin{center}
		\begin{minipage}{0.45\textwidth}
			\[
			p_\xi(x) =
			\begin{cases}
				\dfrac{1}{b - a}, & x \in [a, b], \\
				0, & x \notin [a,b].
			\end{cases}
			\]
		\end{minipage}
		\def\a{1}
		\def\b{4}
		\def\h{0.3333}
		\begin{minipage}{0.5\textwidth}
			\begin{tikzpicture}
				\begin{axis}[
					axis lines=middle,
					axis line style={very thin},
					xlabel={$x$}, ylabel={$p_\xi(x)$},
					ymin=0, ymax=0.5,
					xmin=0, xmax=5,
					xtick={1,4}, xticklabels={$a$, $b$},
					ytick={0,\h},
					yticklabels={$0$, $\frac{1}{b - a}$},
					domain=0:5,
					samples=100,
					width=7cm,
					height=4.5cm,
					]
					\addplot [
					black,
					thick,
					domain=\a:\b,
					] {1 / (\b - \a)};
					\addplot[domain=0:\a, black, ultra thick] {0};
					\addplot[domain=\b:5, black, ultra thick] {0};
				\end{axis}
			\end{tikzpicture}
		\end{minipage}
	\end{center}
	
 	
	\subsubsection{Экспоненциальное (показательное) распределение}
	
	Как равномерное распределение выступает аналогом дискретного равномерного распределения,
	так и экспоненциальное распределение является абсолютно непрерывным аналого геометрического распределения.
	Пусть $ \Omega = [0,+\infty) $ --- бесконечный полуинтервал, $ \events = \calB \cap [0,+\infty) $ 
	--- борелевская $ \sigma $-алгебра
	и вероятностная мера задана на бесконечных полуинтервалах как $ \prob((u; +\infty)) = e^{-\alpha u} $,
	где $ \alpha > 0 $ --- некоторое число.
	Рассмотрим распределение случайной величины $ \xi $, которая сопоставляет точке на отрезке её координату.
	В реальной жизни данная конструкция соответствует, например, ситуации, когда измеряется время до первой поломки
	электроприбора или до первого прихода автобуса, 
	если такие события происходят часто и независимо, а частота их наступления (<<интенсивность>> $ \alpha $) постоянна.
	Распределение случайной величины $ \xi $ называется 
	\defin{экспоненциальным (или показательным) распределением}{exponential-distribution}
	и обозначается $ \mathrm{E}(\alpha) $.
	Экспоненциальная случайная величина обладает функцией плотности,
	которую, как и функцию распределения можно задать кусочно.
	Формулы для функции распределения и плотности приведены ниже. 
	\begin{center}
		\begin{minipage}{0.45\textwidth}
			\[
			F_\xi(x) =
			\begin{cases}
				0, & x < 0, \\
				1 - e^{-\alpha x}, & x \geqslant 0.
			\end{cases}
			\]
		\end{minipage}
		\begin{minipage}{0.5\textwidth}
			\begin{tikzpicture}
				\begin{axis}[
					axis lines=middle,
					axis line style={very thin},
					xlabel={$x$}, ylabel={$F_\xi(x)$},
					ymin=0, ymax=1.1,
					xmin=-1, xmax=6,
					xtick={0},
					ytick={0,1},
					width=7cm,
					height=4.5cm,
					]
					\addplot[
					black,
					thick,
					domain=0:6,
					samples=200
					] {1 - exp(-1 * x)}; % alpha = 1
					
					\addplot[domain=-1:0, black, ultra thick] {0};
				\end{axis}
			\end{tikzpicture}
		\end{minipage}
	\end{center}
	
	\begin{center}
		\begin{minipage}{0.45\textwidth}
			\[
			p_\xi(x) =
			\begin{cases}
				\alpha e^{-\alpha x}, & x \geqslant 0, \\
				0, & x < 0.
			\end{cases}
			\]
		\end{minipage}
		\begin{minipage}{0.5\textwidth}
			\begin{tikzpicture}
				\begin{axis}[
					axis lines=middle,
					axis line style={very thin},
					xlabel={$x$}, ylabel={$p_\xi(x)$},
					ymin=0, ymax=1.2,
					xmin=-1, xmax=6,
					xtick={0},
					ytick={0},
					domain=0:6,
					samples=200,
					width=7cm,
					height=4.5cm,
					]
					\addplot[
					black,
					thick,
					domain=0:6,
					] {1 * exp(-1 * x)}; % здесь alpha = 1
					
					\addplot[domain=-1:0, black, ultra thick] {0};
				\end{axis}
			\end{tikzpicture}
		\end{minipage}
	\end{center}
	
	\begin{example}
		Будем рассматривать переменную $ x $ на прямой как переменную означающую <<прошедшее время>>.
		Для абсолютно непрерывной случайной величины $ \xi $ введём 
		\defin{функцию надёжности}{reliability-function} $ \overline{F}_\xi(x) := 1 - F_\xi(x) = \prob(\xi > x) $
		выражающую вероятность того, что нечто произойдёт (приедет автобус) или не произойдёт (поломка электроприбора)
		в течении времени $ x $. Введём также функцию \defin{интенсивности отказов}{failure-rate}
		полагаемую равной $ \lambda(x) = 
		\lim\limits_{\Delta x \to 0} \tfrac{\overline{F}_\xi(x + \Delta x) - \overline{F}_\xi(x)}{\overline{F}_\xi(x)\Delta x} $.
		Интенсивность отказов выражает <<долю отказавших в единицу времени приборов>>.
		Почти всюду функции $ \lambda(x) $ и 
		$ - \tfrac{\overline{F}_\xi'(x)}{\overline{F}_\xi(x)} = \tfrac{p_\xi(x)}{\overline{F}_\xi(x)} $
		совпадают, поэтому будем также писать $ \lambda(x) = \tfrac{p_\xi(x)}{\overline{F}_\xi(x)} $.
		
		Пусть теперь интенсивность отказов постоянна на полуинтервале $ [0;+\infty) $ и равна $ \alpha > 0 $.
		Из формулы выше получаем дифференциальное уравнение $ \overline{F}_\xi' = -\alpha\overline{F}_\xi $,
		откуда $ \overline{F}_\xi(x) = Ce^{-\alpha x} $ на $ [0;+\infty) $. 
		Если считать, что $ \overline{F}_\xi(0) = 1 $ (<<изначально всё работает>>), 
		то  $ C = 1 $ и $ F_\xi(x) = 1 - e^{-\alpha x} $.
		
		В общем случае, решив такую же задачу Коши с начальным условием $ \overline{F}_\xi(0) = 1 $,
		получаем формулу для функции распределения:
		$$ F_\xi(x) = 1 - e^{-\int\limits_{0}^{x} \lambda(x)\diff x}. $$
	\end{example}
	
	\subsubsection{Нормальное распределение (распределение Гаусса)}
	
	Распределение, которое мы рассмотрим сейчас возникает 
	как предел распределения суммы большого количества малых, 
	независимых случайных величин (см. Центральную предельную теорему ниже).
	Нормальное распределение $ \mathcal{N}(a, \sigma^2) $ задаётся двумя параметрами $ a $ и $ \sigma > 0 $,
	которые, как мы выясним позже равны её математическому ожиданию и квадратному корню из дисперсии, соответственно.
	Если функция распределения случаной величины $ \xi $ имеет вид, указанный ниже,
	то её распределение называется \defin{нормальным распределением}{normal-distribution}.
	Если к тому же параметры имеют особый вид: $ a = 0 $ и $ \sigma^2 = 1 $,
	то распределение называется \defin{стандартным нормальным распределением}{standard-normal-distribution}.
	\begin{center}
		\begin{minipage}{0.45\textwidth}
			\[
			\Phi_{a, \sigma^2}(x) := F_\xi(x) = \frac{1}{\sqrt{2\pi}\sigma} \int_{-\infty}^x e^{-\tfrac{(x - a)^2}{2\sigma^2}} \diff t;
			\]
			{\color{orange}
			\[
			\Phi(x) := F_\xi(x) = \frac{1}{\sqrt{2\pi}} \int_{-\infty}^x e^{-\tfrac{x^2}{2}} \diff t.
			\]}
		\end{minipage}
		\begin{minipage}{0.5\textwidth}
			\begin{tikzpicture}
				\begin{axis}[
					axis lines=middle,
					axis line style={very thin},
					xlabel={$x$}, ylabel={$F_\xi(x)$},
					ymin=0, ymax=1.05,
					xmin=-4, xmax=4,
					xtick={0,1},
					ytick={0,1/2,1},
					samples=200,
					width=7cm,
					height=4.5cm,
					]
					% Функция распределения стандартного нормального распределения
					\addplot [
					orange,
					thick,
					domain=-4:4
					] {0.5 * (1 + erf(x / sqrt(2)))};
					\addplot [
					black,
					thick,
					domain=-4:4
					] {0.5 * (1 + erf((x - 1) / (sqrt(2)* 1.5)))};
				\end{axis}
			\end{tikzpicture}
		\end{minipage}
	\end{center}
	
	\begin{center}
		\begin{minipage}{0.45\textwidth}
			\[
			p_\xi(x) = \frac{1}{\sqrt{2\pi}\sigma} e^{-\tfrac{(x - a)^2}{2\sigma^2}};
			\]
			{\color{orange}
			\[
			p_\xi(x) = \frac{1}{\sqrt{2\pi}} e^{-\tfrac{x^2}{2}}.
			\]}
		\end{minipage}
		\begin{minipage}{0.5\textwidth}
			\begin{tikzpicture}
				\begin{axis}[
					axis lines=middle,
					axis line style={very thin},
					xlabel={$x$}, ylabel={$p_\xi(x)$},
					ymin=0, ymax=1.05,
					xmin=-4, xmax=4,
					xtick={0,1},
					ytick={0,1/2,1},
					samples=200,
					width=7cm,
					height=4.5cm,
					]
					% Функция распределения стандартного нормального распределения
					\addplot [
					orange,
					thick,
					domain=-4:4
					] {1 / (sqrt(2*pi))*exp(-x*x/2)};
					\addplot [
					black,
					thick,
					domain=-4:4
					] {1 / (sqrt(2*pi)*1.5)*exp(-(x - 1)*(x - 1)/(2*1.5*1.5))};
				\end{axis}
			\end{tikzpicture}
		\end{minipage}
	\end{center}
	
	Функция распределения $ \Phi_{a, \sigma^2} $ выражается через $ \Phi $ по формуле 
	$ \Phi_{a,\sigma^2}(x) = \Phi(\tfrac{x - a}{\sigma}) $.
	
	\subsubsection{Распределение Коши}
	
	Ещё одно абсолютно непрерывное распределение, которое мы рассмотрим, похоже на нормальное распределение.
	Как мы увидим позже, между ними есть существенные различия --- 
	распределение Коши является классическим примером распределения, которое не имеет матожидания.
	Если случайная величина имеет описываемую формулой ниже функцию распределения,
	то её распределение называется \defin{распределением Коши}{Cauchy-distribution}.
	\begin{center}
		\begin{minipage}{0.45\textwidth}
			\[
			K_{a, \sigma^2}(x) := F_\xi(x) = \frac{1}{\pi} \left(\arctan\left(\tfrac{x - a}{\sigma}\right) + \tfrac{\pi}{2}\right);
			\]
			{\color{orange}
			\[
			K_{0, 1}(x) := F_\xi(x) = \frac{1}{\pi} \left(\arctan(x) + \tfrac{\pi}{2}\right).
			\]}
		\end{minipage}
		\begin{minipage}{0.5\textwidth}
			\begin{tikzpicture}
				\begin{axis}[
					axis lines=middle,
					axis line style={very thin},
					xlabel={$x$}, ylabel={$F_\xi(x)$},
					ymin=0, ymax=1.05,
					xmin=-4, xmax=4,
					xtick={0,1},
					ytick={0,1/2,1},
					samples=200,
					width=7cm,
					height=4.5cm,
					]
					% Функция распределения стандартного нормального распределения
					\addplot [
					orange,
					thick,
					domain=-4:4
					] {1 / pi * rad(atan((x - 1)/1.5)) + 0.5};
					\addplot [
					black,
					thick,
					domain=-4:4
					] {1 / pi * rad(atan(x)) + 0.5};
				\end{axis}
			\end{tikzpicture}
		\end{minipage}
	\end{center}
	
	\begin{center}
		\begin{minipage}{0.45\textwidth}
			\[
			p_\xi(x) = \frac{1}{\pi} \tfrac{1}{1 + \left(\tfrac{x - a}{\sigma}\right)^2};;
			\]
			{\color{orange}
			\[
			p_\xi(x) = \frac{1}{\pi} \tfrac{1}{1 + x^2}.
			\]}
		\end{minipage}
		\begin{minipage}{0.5\textwidth}
			\begin{tikzpicture}
				\begin{axis}[
					axis lines=middle,
					axis line style={very thin},
					xlabel={$x$}, ylabel={$p_\xi(x)$},
					ymin=0, ymax=1.05,
					xmin=-4, xmax=4,
					xtick={0,1},
					ytick={0,1/2,1},
					samples=200,
					width=7cm,
					height=4.5cm,
					]
					% Функция распределения стандартного нормального распределения
					\addplot [
					orange,
					thick,
					domain=-4:4
					] {1 / pi * 1 / (((x - 1)*(x - 1)) / 1.5 / 1.5 + 1)};
					\addplot [
					black,
					thick,
					domain=-4:4
					] {1 / pi * 1 / (x*x + 1)};
				\end{axis}
			\end{tikzpicture}
		\end{minipage}
	\end{center}
	
	Аналогично нормальному распределению имеет место формула
	$ K_{a,\sigma^2}(x) = K_{0, 1}(\tfrac{x - a}{\sigma}) $.
	
	\subsection{Сингулярные распределения}
	
	Распределение случайной величины $ \xi $ называется \defin{сингулярным распределение}{singular-distribution},
	если её функция распределения $ F_\xi $ непрерывна и множество её точек роста 
	$$ \defineset{x \in \RR}{\forall \varepsilon > 0 \ F_\xi(x - \varepsilon) < F_\xi(x + \varepsilon)} $$
	имеет меру нуль по Лебегу.
	
	\begin{example}
		Стандартным примером сингулярного распределения служит распределение, соответствующее канторовой лестнице 
		$ c \colon [0, 1] \to [0, 1] $.
		Рассмотрим канторово множество 
		$$ K = \bigcap\limits_{i = 1}^{+\infty} 
		\bigcup\limits_{j = 0}^{2^i - 1} \left[\tfrac{k_j}{3^i}, \tfrac{k_j + 1}{3^i}\right]
		= \bigcap\limits_{i = 1}^{+\infty} 
		\bigcup\limits_{j = 0}^{3^{i - 1} - 1} 
		\left[\tfrac{3j}{3^i}, \tfrac{1 + 3j}{3^i}\right] \cup \left[\tfrac{2 + 3j}{3^i}, \tfrac{3 + 3j}{3^i}\right]
		= [0, 1] \setminus \bigcup\limits_{i = 1}^{+\infty} 
		\bigcup\limits_{j = 0}^{3^{i - 1} - 1} \left(\tfrac{1 + 3j}{3^i}, \tfrac{2 + 3j}{3^i}\right), $$ 
		где $ k_j $ --- $ j $-е число от $ 0 $ до $ 3^i $, в троичной записи которого присутствуют только нули и двойки.
		Всякая точка канторова множеств в троичной системе
		счисления имеет вид бесконечной троичной дроби с нулями и двойками.
		Тогда канторова лестница задаётся следующими формулами
		$$ c(x)
		=
		\begin{cases}
			\tfrac{j}{2^i}, & x \in \left(\tfrac{1 + 3j}{3^i}, \tfrac{2 + 3j}{3^i}\right); \\
			\sum\limits_{i = 1}^{+\infty} \tfrac{a_j / 2}{2^i}, & x \in K, x = \sum\limits_{i = 1}^{+\infty} \tfrac{a_j}{3^j}.
		\end{cases} $$
		Можно показать, что заданная таким образом функция непрерывна. 
		Множество точек роста $ c $ совпадает канторовым множеством,
		и поэтому имеет меру нуль по Лебегу.
		
		В нашем примере оказалось так, что канторова лестница $ c $ имеет производную почти всюду, 
		как и абсолютно непрерывные функции. Однако, в отличии от абсолютно непрерывных функций
		эта производна равна 0 во всех точках своего определения и поэтому канторову лестницу нельзя вычислить 
		как интеграл от своей производной.		
	\end{example}
	
	\section{Совместные распределения случайных величин}
	
	
	
	
	\section{Численные характеристики случайных величин}
	
	\subsection{Математическое ожидание, моменты и абсолютные моменты}
	
	\subsection{Ковариация, дисперсия и корреляция}
	
	\subsection{Мода, медиана}
	
	\subsection{Квантили, асимметрия и эксцесс}
	
	\TODO{записать определения и свойства, описать ковариацию как скалярное произведение}
	
	\TODO{доказать формулы для вычисленя матожидания через интегралы Лебега, Лебега-Стилтьеса и интеграл Римана для абсолютно непрерывной случайно величины}
	
	
	
	\section{Сходимости случайных величин}
	
	\subsection{Сходимость почти наверное}
	
	\subsection{Сходимость по вероятности}
	
	\subsection{Пространство $ \mathcal{L}_p $ и сходимость в нём}
	
	\subsection{Сходимость по распределению}
	
	\subsection{Связь сходимостей}
	
	\TODO{записать определения всех сходимостей и вывод одних сходимостей из других}
	
	\section{Производящие функции}
	
	\TODO{записать определение}
	
	\section{Характеристические функции}
	
	\begin{theorem}[Бохнер, Хинчин]
		
	\end{theorem}
	
	\section{Предельные теоремы}
	
	\TODO{дописать ниже доказательства теорем}
	
	\subsection{Неравенства}
	
	\subsection{Закон больших чисел}
	
	\begin{theorem}[Закон больших чисел в форме Бернулли]
		
	\end{theorem}
	
	\begin{theorem}[Закон больших чисел в форме Чебышёва]
		
	\end{theorem}
	
	\begin{theorem}[Усиленный закон больших чисел]
		
	\end{theorem}
	
	\begin{theorem}[Закон больших чисел в форме Хинчина]
		content
	\end{theorem}
	
	\subsection{Теорема Муавра-Лапласа}
	
	\begin{theorem}[Теорема Пуассона]
		
	\end{theorem}
	
	\begin{theorem}[Формула Стирлинга]
		
	\end{theorem}
	
	\begin{theorem}[Муавр, Лапласа]
		
	\end{theorem}
	
	\subsection{Закон нуля или единицы}
	
	\begin{lemma}[Борель, Кантелли]
		
	\end{lemma}
	
	\begin{lemma}[Борель, Кантелли]
		
	\end{lemma}
	
	\begin{theorem}[Закон нуля или единицы Колмогорова]
		
	\end{theorem}
	
	\subsection{Закон повторного логарифма}
	
	\begin{theorem}[Закон повторного логарифма]
		
	\end{theorem}
	
	\subsection{Закон арксинуса}
	
	\begin{theorem}[Закон арксинуса]
		
	\end{theorem}
	
	\subsection{Правило трёх сигм}
	
	\begin{theorem}[Правило трёх сигм]
		
	\end{theorem}
	
	\subsection{Центральная предельная теорема}
	
	\begin{theorem}[Центральная предельная теорема]
		
	\end{theorem}
	
	\begin{theorem}[Оценка Берри-Эссена]
		
	\end{theorem}
	
	\section{Свёртки случайных величин}
	
	d
	
	\section{Указатель терминов}
	
	d
	
	\section{Указатель теорем}
	
	d
	
	\begin{thebibliography}{2}
		
		\bibitem{KolmogorovFomin} Колмогоров~А.~Н., Фомин~С.~В. {\it Элементы теории функций и функционального анализа}, Физматлит, 2004, 572с.
		
		\bibitem{DiyachenoUliyanov} Дьяченко~М.~И., Ульянов~П.~Л. {\it Мера и интеграл}, Факториал, 1998, 160с.
	
		\bibitem{Borovkov} Боровков~А.~А. {\it Теория вероятностей}, Физматлит, 1986, 432с.
	
	\end{thebibliography}
	
	
\end{document}

