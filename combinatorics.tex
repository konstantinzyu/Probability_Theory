\section{Элементарная комбинаторика}
\subsection{Введение}
Элементарная комбинаторика изучает методы подсчёта числа способов выбора и расположения объектов. В теории вероятностей она служит основой для вычисления числителей при расчёте вероятностей различных событий.

\subsection{Основные принципы подсчёта}
\paragraph{Принцип сложения.} Если несколько (
например, два) взаимоисключающих способов выполнения операции имеют $m$ и $n$ вариантов соответственно, то общее число вариантов равно
$$m + n.$$ 
\paragraph{Принцип умножения.} Если операция состоит из двух последовательных шагов, имеющих $m$ и $n$ независимых вариантов выполнения, то общее число вариантов равно
$$m \times n.$$ 

\subsection{Комбинаторные числа}
\begin{description}
	\item[Перестановки] Число перестановок $n$ различных объектов:
	\begin{equation*}
		P_n = n!.
	\end{equation*}
	\item[Размещения без повторений] Число упорядоченных выборок из $k$ разных элементов по порядку:
	\begin{equation*}
		A_n^k = \frac{n!}{(n-k)!}.
	\end{equation*}
	\item[Размещения с повторениями] Число последовательностей длины $k$, каждый элемент выбирается из $n$ вариантов с повторениями:
	\begin{equation*}
		{A'}_n^k = n^k.
	\end{equation*}
	\item[Сочетания без повторений] Число подмножеств мощности $k$ из $n$:
	\begin{equation*}
		C_n^k = \binom{n}{k} = \frac{n!}{k!\,(n-k)!}.
	\end{equation*}
	\item[Сочетания с повторениями] Число неупорядоченных выборок из $k$ элементов с возвращением из $n$ типов:
	\begin{equation*}
		{C'}_n{}^k = \binom{n + k -1}{k} = \frac{(n + k -1)!}{k!\,(n-1)!}.
	\end{equation*}
\end{description}

\subsection{Доказательства формул}
\begin{itemize}
	\item Для $A_n^k$ применяем принцип умножения:
	$$n\times(n-1)\times\cdots\times(n-k+1)=\frac{n!}{(n-k)!}.$$ 
	\item Для $C_n^k$ замечаем, что выбрать $k$ и упорядочить их можно $A_n^k$ способами, причём каждый набор упорядочен $k!$ раз, поэтому
	$$C_n^k=\frac{A_n^k}{k!}=\frac{n!}{k!\,(n-k)!}.$$
	\item Для сочетаний с повторениями используем метод "звёзд и палочек": нужно расположить $k$ звёздочек и $n-1$ разделителей, всего $(n+k-1)$ позиций, выбираем $k$ из них под звёздочки.
\end{itemize}

\subsection{Примеры}
\begin{enumerate}
	\item Сколько различных перестановок букв в слове "ТЕСТ": $P_4=4!=24$.
	\item Сколькими способами можно выбрать комитет из 3 человек из 10 без учёта порядка: $C_{10}^3=\frac{10!}{3!7!}=120$.
	\item Сколько PIN-кодов длины 4 можно составить из десяти цифр (0--9): ${A'}_{10}^4=10^4=10000$.
	\item Сколькими способами можно распределить 5 одинаковых яблок между 3 детьми: $C'_{3}{}^5=\binom{3+5-1}{5}=\binom{7}{5}=21$.
\end{enumerate}

\subsection{Свойства и тождества}
\begin{itemize}
	\item \textbf{Симметрия:} $\displaystyle C_n^k = C_n^{n-k}.$
	\item \textbf{Рекуррентность (тожд. Паскаля):} $\displaystyle C_n^k = C_{n-1}^{k-1} + C_{n-1}^k.$
	\item \textbf{Сумма биномиальных коэффициентов:} $\displaystyle \sum_{k=0}^n C_n^k = 2^n.$
	\item \textbf{Бином Ньютона:} $(1 + x)^n = \sum_{k=0}^n C_n^k \,x^k.$
\end{itemize}

\subsection{Доказательства свойств}
\begin{itemize}
	\item Симметрию можно увидеть, выбирая $k$ из $n$ либо сразу, либо сначала отмечая невключенных.
	\item Для тождества Паскаля разделяем выбор на те, где фиксированный элемент включён ($C_{n-1}^{k-1}$) и не включён ($C_{n-1}^k$).
	\item Сумму $\sum C_n^k$ считают как общее число подмножеств множества из $n$ элементов, всего $2^n$.
	\item Бином Ньютона выводится из разложения произведения $n$ сомножителей $(1+x)$ по правилу умножения.
\end{itemize}
