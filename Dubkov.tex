\documentclass{article}
\usepackage{amsmath, amssymb}
\usepackage[utf8x]{inputenc} % Включаем поддержку UTF8  
\usepackage[russian]{babel} % Required for inserting images

\title{Отрицательноe биноминальное распределение}
\author{Семён Дубков}
\date{May 2025}

\begin{document}
\maketitle
\fontsize{13}{15}\selectfont
\section{Определение}
Говорят что случайная велечина $\xi$ имеет отрицательное биноминальное распределение с парамeтрами $p$ и $r$, если :
\[
p_k = \mathbb{P}( \xi = k) = C_{k-1}^{r-1} p^rq^{k-r},\ k=r,r+1 \dots ,r+k, \ q = 1-p
\]
Обозначается так: $\xi \sim NB(p,r) $\\
Например число испытаний в схеме Бернулли необходимых для $r$ успехов имеет отрицательное биноминальное распределение

\section{Математическое ожидание и дисперсия}
Найдем математическое ожидание с помощью производящей функции
\begin{align*}
\varphi(x)=\sum_{k=0}^{\infty}p_kx^k = \sum_{k=r}^{\infty}C_{k-1}^{r-1}p^rq^{k-r}x^k=p^r\sum_{k=0}^{\infty}C_{k+r-1}^{r-1}q^kx^{k+r}=\\
p^rx^r\sum_{k=0}^{\infty}C_{k+r-1}^{r-1}(qx)^k=p^rx^r(1-qx)^{-r}=\left(\frac{px}{1-qx}\right)^r
\end{align*}
Последний переход следует из разложения в ряд Тейлора функции:
\begin{align*}
    (1+t)^{\alpha}=1+\alpha t+\frac{\alpha(\alpha-1)}{2}t^2+\dots
\end{align*}
Так как $\mathbb{M}\xi^{[k]}=\varphi^{(k)}(1-) $, где $\xi^{[k]}\overset{def}{=}\xi(\xi-1)\dots(\xi-k+1)$, то при $k=1$ получаем $\mathbb{M}\xi=\varphi'(1-)$
\begin{align*}
    \varphi'(x) = p^rr(\frac{x}{1-qx})^{r-1}\frac{1-qx+qx}{(1-qx)^2}=p^rr\frac{x^{r-1}}{(1-qx)^{r+1}}
\end{align*}
\begin{align*}
    \varphi'(1-)=p^rr\frac{1^{r-1}}{(1-q)^{r+1}}=\frac{p^rr}{p^{r+1}}=\frac{r}{p}
\end{align*}
Получаем ответ:
\[
\boxed{\mathbb{M}\xi=\frac{r}{p}}
\]
Найдем дисперсию $\mathbb{D\xi}=\mathbb{M}\xi^{2}-(\mathbb{M\xi})^2$
\[
\mathbb{M}\xi^{[2]}=\varphi''(1-),\ \mathbb{M}\xi=\varphi'(1-)
\]
\[
\mathbb{M}\xi^{[2]}=\mathbb{M}\xi(\xi-1)=\mathbb{M}\xi^2-\mathbb{M}\xi
\]
Выражаем $\mathbb{M}\xi^2 = \mathbb{M}\xi^{[2]}+\mathbb{M}\xi=\varphi''(1-)+\varphi'(1-)$, и подставляем в дисперсию $\mathbb{D\xi}=\varphi''(1-)+\varphi'(1-)-(\varphi'(1-))^2$. Получили связь между дисперсией и производящей функцией. Найдем $\varphi''(1-)$
\begin{center}
$\varphi''(x)=\left(p^rr\frac{x^{r-1}}{(1-qx)^{r+1}}\right)'=p^rr\frac{(r-1)x^{r-2}(1-qx)^{r+1}+q(r+1)(1-qx)^rx^{r-1}}{(1-qx)^{2r+2}}$    
\end{center}


\begin{center}
$\varphi''(1-)=p^rr\frac{(r-1)(1-q)^{r+1}+q(r+1)(1-q)^r}{(1-q)^{2r+2}}= 
p^rr\frac{(r-1)p^{r+1}+q(r+1)p^r}{p^{2r+2}}=
p^{2r}r\frac{(r-1)p+q(r+1)}{p^{2r+2}}=r\frac{rp-p+qr+q}{p^2}=r\frac{r-p+q}{p^2}
$    
\end{center}
\[
\mathbb{D\xi}=\varphi''(1-)+\varphi'(1-)-(\varphi'(1-))^2=
\]
\[
r\frac{r-p+q}{p^2}+\frac{r}{p}-\frac{r^2}{p^2}=\frac{r^2-pr+qr+pr-r^2}{p^2}=\frac{qr}{p^2}
\]
\[
\boxed{\mathbb{D}\xi=\frac{qr}{p^2}}
\]

\end{document}